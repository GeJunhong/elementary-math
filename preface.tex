
\chapter*{序}
\addcontentsline{toc}{chapter}{前言}

%\begin{verse}
%无材可去补苍天,枉入红尘若许年。\\
%此系身前身后事,倩谁记去作奇传。\\
%-- 曹雪芹
%\end{verse}

\section{本书的由来}

中学时代的我热爱数学,对一些感兴趣的数学问题写过一些笔记,然而那时没有积累意识,有些稿件在之后的几年里就再也没看到过了\footnote{尽管大部分在现在看来只是小儿科。}。工作后的业余时间里,我还是喜欢偶尔想些数学问题,有时给网友解答问题也会有一些数学主题,并产生了一些新的笔记。在接触了LaTeX之后,就萌发了把这些文字材料整理成册的想法,也为了以后新写的数学材料能得到长期积累和妥善保管,而Emacs编辑器近乎无所不能的威力,更让我的写作欲望越来越强,这便是本书的起源。

本书的内容仅限于初等数学,即以高中数学为核心,额外可能会有一些平面几何和初等数论的内容,但是基本上不会涉及到极限以上的数学理论,这意味着不会有导数这样的内容。

本的源代码托管在全球最大的代码托管网站GitHub上\footnote{注:本人的其它项目也都在 \href{https://github.com/zhcosin}{https://github.com/zhcosin} 下。}:
\begin{center}
\href{https://github.com/zhcosin/elementary-math}{https://github.com/zhcosin/elementary-math}
\end{center}

\section{作者简介}
%\begin{center}
\begin{quotation}
zhcosin, A programmer love math in depth.
\end{quotation}
%\end{center}

鄙人对数学的兴趣始于初三,其时,代数学中的抛物线和几何学中的圆,让我一改对数学死板僵硬的印象,从此与数学结下不解之缘。高中时期,数学单科年级第一的地位从来无可撼动,在2005年全国高中数学联赛中摘获四川省第46名,引得学校领导发现我们这个穷县的县中还是能够出几个人才的,于是从下一届开始了层层加码的数学竞赛培训,也算是改写了母校的竞赛历史(虽然个人觉得这并非是一件好事),后来高考数学也是单科满分。然而长期数学思维的训练使得文科头脑退化,惨不忍睹的英语分数使得我只能勉强上一个二本学校,还是找遍厚厚的挑学校指南才选了个叫做大学而不是学院的东北某高校\footnote{英语很烂的情况在考研时期得到极大改观,学习方法是:根本不背单词,每天只练习阅读和翻译,坚持了一年,结果英语分数大大超过国家线。英语水平的改善为之后我学习计算机科学奠定了基础。}。大学期间贪玩,微积分与高等代数也不曾好生学,成为大学期间的一大遗憾,导致我现在不得不下决心利用业余时间系统性的重新学习。

数学专业硕士毕业后选择了软件开发行业,说的好听是软件工程师,说的不好听就是程序员。然而我喜欢计算机科学,它是一门实践性强的科学,而数学是一门理论性非常强的科学。我喜欢编译器,因为可以打造自己的设计程序语言,可以自己实现计算器\footnote{曾经实现过,但没有按照编译器的理论来写,所以不满意,打算重新实现。},如果你有足够兴趣和精力,还能实现像mathematics那样的符号计算系统。我会打造一门自己的程序设计语言。另外还打算学习算法,这应该是对数学能力要求较高的计算机领域,因此有兴趣深入学习,不过目前并无计划。

\section{读者对象}

\begin{enumerate}
\item 我自己。
\item 爱好数学并且有大把时间的高中学生,通常这时间是牺牲别的科目作代价的。
\end{enumerate}

%%% Local Variables:
%%% mode: latex
%%% TeX-master: "book"
%%% End:
