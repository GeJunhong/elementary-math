
\chapter*{序}
\addcontentsline{toc}{chapter}{序}

这份笔记开工于2016年4月6日,其实它在这之前几年就应该动笔的,自从2005年高考后的十年间,几乎没有认真学习过数学,几年下来越来越对自己的数学能力不满意,所以打算以写作这份笔记为契机,给自己一个重新学习的机会。虽然有时也有一些数学上的锻炼,但终究没有积累下来,没有形成自己的知识体系,这份笔记算是在这方面的一个尝试。

其实我不能算是这份笔记的作者,只能算是整理者,因为其中属于我自己原创的内容并不多,就这份笔记所涉及的内容而言,那些定理和结论都是前人几百年前甚至早在公元前就研究透彻了,我辈能在没接触前人成果的情况下独立发现一些结论就已经是非常不容易了,何况我是在早已经接触过这些结论甚至还大体记得推导证明过程的基础上进行了重新推导而已。这份笔记主要内容的来源,基本上是参考文献所列出的那几本书籍,这基本上也是我这几年所读过的书,实际上没有哪一部是认真读完过的,还有极少数的内容是我自己在没接触过前人结论的情况下自己推导所得(例如伯努利信封问题,我一直称为错位排列问题),这些内容主要以例子的形式出现。

这份笔记的写作受到了前苏联数学家菲赫金哥尔茨所著《微积分学教程》的影响,这是我最推崇的一部巨著,书中取材之广泛,讨论之深度和广度超乎我的想象能力,我也从这部书中受益良多,这份笔记也在取材和广度和深度方面甚至内容的组织方面都受到它的影响。

这份笔记目前没有成书之日,也没有什么计划之类,受限于自己的数学能力和工作闲暇,不定期的更新而已。

我最佩服的几何学家是古希腊的 Apollonius,中文译作阿波罗尼奥斯,他生活在公元前的古希腊,他所著的《圆锥曲线论》将圆锥曲线的性质几乎一网打尽,以致于后人在长达两千年间没能在这个领域有所建树,直到笛卡尔坐标几何的创立,他所采用的还是纯几何理论,当然他也是在一些前人的研究成果上结合自己的研究写出了这部巨著,有时间会认真读一读这部书的部分内容。让我惊吧的是这竟然是在公元前的希腊完成的,古希腊的数学到底是有多发达,何以古希腊没能在我们今天所称四大文明古国之列呢。

%%% Local Variables:
%%% mode: latex
%%% TeX-master: "book"
%%% End:
