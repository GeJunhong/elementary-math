
\section{几个原理}
\label{sec:some-principle}

\subsection{容斥原理}
\label{subsec:inclusion-exclusion-principle}

\begin{principle}[容斥原理]
  用$|A|$表示集合$A$的元素个数,则多个集合的并集的元素个数是:
  \begin{eqnarray}
    \label{eq:inclusion-exclusion-principle}
    \left| \bigcup_{i=1}^nA_i \right| & = & \sum_{i=1}^n|A_i|-\sum_{1\leqslant i <j \leqslant n} \left| A_i \cap A_j \right| +\sum_{1 \leqslant i <j <k \leqslant n} \left| A_i\cap A_j \cap A_k \right| \nonumber \\
 & & -\cdots+(-1)^{n+1} \left| \bigcap_{i=1}^n A_i \right|
  \end{eqnarray}
\end{principle}
容斥原理的含义借助韦恩图是显而易见的,其证明用数学归纳法即可,这里从略。

\begin{figure}[htbp]
  \centering
\includegraphics{content/prepare/pic/inclusion-exclusion-principle.pdf}
\caption{容斥原理示意图}
\label{fig:inclusion-exclusion-principle}
\end{figure}


\begin{example}[伯努利(Bernoulli)信封问题]
  作为容斥原理的一个直接应用,这里讨论一下伯努利信封问题: 有相同数目的信封和信件若干,将这些信件装进这些信封,使得没有任何一封信件与信封搭配正确,问题是有多少种装法。

  记信件数目是$n$,并把信件和信封依次编号为$1,2,\ldots,n$,信件与所属的信封编号相同。那么要计算所有信件都搭配错误的组装数,可以考虑其反面即至少有一封信件搭配正确的组装数目,再从总数$n!$中减去它即可,根据容斥原理,构造出集合$A_i$表示第$i$封信件搭配正确的组装方案,则$|A_i|=(n-1)!$,那么至少有一封信件搭配正确的方案集合就是这些$A_i$的并集,其数目是
  \begin{eqnarray*}
      &    \sum_{i=1}^n|A_i|-\sum_{1\leqslant i <j \leqslant n}|A_i \cap A_j|+\sum_{1 \leqslant i <j <k \leqslant n}|A_i\cap A_j \cap A_k| \\
      & -\cdots+(-1)^{n+1}|\cap_{i=1}^nA_i| \\
      = & n (n-1)! - C_n^2(n-2)! + C_n^3(n-3)! - \cdots + (-1)^{n+1}C_n^n0! \\
      = & n!\sum_{i=1}^n(-1)^{i+1}\frac{1}{i!}
  \end{eqnarray*}
  所以从总数$n!$中减去它就得到最后的结果,在$0!=1$的约定下,这结果可以写为:
  \begin{equation}
    \label{eq:bernoulli-envelope-problem-solution}
    \sum_{i=0}^n(-1)^i \frac{1}{i!}
  \end{equation}

  欧拉对此问题有一个借助数列递推的解法,记信封数目为$n$时的结果为$a_n$,那么$a_1=0$,来看下这个数列的递推情况,在有$n+1$个信封时,考虑编号为1的那封信件,除1号信封外它有$n$个信封可以装入,假定它装入的信封编号是$r$,那再考虑编号为$r$的信件,它此时有两个选择,一是它可以装入1号信封,这时其它的$n-1$个信件的装法是$a_{n-1}$,它的另一个选择是不装入1号信封,这时由于1号信封等同于$r$号信封,所以除1号信件以外的$n$封信件有$a_n$种装法,于是得到该数列的递推公式为:
  \begin{equation*}
    a_{n+1}=n(a_n+a_{n-1})
  \end{equation*}
  两边同除以$(n+1)!$并记$b_n=\frac{a_n}{n!}$可得
  \begin{equation*}
    b_{n+1}=\frac{n}{n+1}b_n+\frac{1}{n+1}b_{n-1}
  \end{equation*}
  即
  \begin{equation*}
    b_{n+1}-b_n=-\frac{1}{n+1}(b_n-b_{n-1})
  \end{equation*}
  所以
  \begin{equation*}
    b_n-b_{n-1}= \sum_{i=2}^{n}(b_i-b_{i-1})= \sum_{i=2}^n (-1)^{i}\frac{1}{i!}
  \end{equation*}
  于是
  \begin{equation*}
    a_n= n!b_n = n!(b_1 + \sum_{i=2}^n(b_i-b_{i-1})) = n! \sum_{i=2}^n(-1)^{i}\frac{1}{i!}
  \end{equation*}
  同样在$0!=1$的约定下即有
  \begin{equation*}
    a_n = n!\sum_{i=0}^n(-1)^{i}\frac{1}{i!}
  \end{equation*}
\end{example}

\subsection{数学归纳法}
\label{subsec:mathematical-induction}

除了常用的第一数学归纳法以外,我们还有以下的:
\begin{principle}[第二数学归纳法]
如果与正整数有关的命题$P(n)$满足:
  \begin{enumerate}
  \item $P(1)$成立;
  \item 由$P(1),P(2),\dots,P(k)$成立能够推证出$P(k+1)$成立;
  \end{enumerate}
那么该命题对于一切正整数成立.
\end{principle}
有些命题在递推过程中,依赖的是前面的所有结论而非仅仅依赖前一个结论,此时第二数学归纳法就非常适用。
\begin{principle}[倒推归纳法]
如果与正整数有关的命题$P(n)$满足:
  \begin{enumerate}
  \item 有无穷多个正整数$n$使命题成立;
  \item 由$P(k)$成立能够推证$P(k-1)$成立;
  \end{enumerate}
那么该命题对于一切正整数成立。
\end{principle}
倒推归纳法的原理也是显而易见的,对于任何一个给定的正整数$n$,由于不超过$n$的正整数只有有限个,所以必然存在一个大于$n$的正整数$N$,使得命题$P(N)$成立,再倒推回来,知$P(n)$成立。

\begin{example}[均值不等式]
作为一个例子,我们用倒推归纳法来证明均值不等式,这比用通常的第一数学归纳法来得更加容易:

  均值不等式的内容是: 对任意$n(n\geqslant2)$个正实数$a_i$,有下面不等式成立:
\[ \frac{1}{n}\sum_{i=1}^na_i \geqslant \left( \prod_{i=1}^na_i \right)^{\frac{1}{n}} \]

\begin{proof}[证明]\footnote{这个倒推归纳法的证明来自于参考文献\cite{the-secret-of-inequality}.}
  对$n=2$的情形,有$\frac{1}{2}(a_1+a_2)-\sqrt{a_1a_2}=\frac{1}{2}(\sqrt{a_1}-\sqrt{a_2})^2\geqslant 0$知不等式成立。

 反复使用$n=2$的结论,我们就可以得到当$n$是2的幂的时候不等式是成立的,然而2的幂是无穷多的,所以只要证明,不等式如果对$n+1$个正实数成立就必然对$n$个正实数也成立就可以了。

对于任意$n$个正实数,我们再添加一个正实数$a_{n+1}=\frac{1}{n}\sum_{i=1}^na_i$构成$n+1$个正实数,由假设,不等式对$n+1$个正实数是成立的,所以有
\[
\frac{1}{n+1}\sum_{i=1}^{n+1}a_i \geqslant \left( \prod_{i=1}^{n+1}a_i \right)^{\frac{1}{n+1}}
\]
而由于$a_{n+1}$正好等于其它$n$个实数的平均数,所以
\[ \frac{1}{n+1}\sum_{i=1}^{n+1}a_i = \frac{1}{n}\sum_{i=1}^{n}a_i \]
因此前一式即为:
\[
\frac{1}{n}\sum_{i=1}^{n}a_i \geqslant \left( \prod_{i=1}^{n}a_i \right)^{\frac{1}{n+1}} \cdot \left( \frac{1}{n}\sum_{i=1}^na_i \right)^{\frac{1}{n+1}}
\]
化简即得
\[
\frac{1}{n}\sum_{i=1}^na_i \geqslant \left( \prod_{i=1}^na_i \right)^{\frac{1}{n}} 
\]
即得证。
\end{proof}
\end{example}

\begin{principle}[跳跃数学归纳法]
  如果与正整数有关的命题$P(n)$满足:
  \begin{enumerate}
  \item $P(1)$,$P(2)$,$\ldots$,$P(m)$成立;
  \item 由$P(k)$成立能够推证$P(k+m)$成立;
  \end{enumerate}
那么该命题对于一切正整数成立。
\end{principle}

数学归纳法还有其他形式,比如解决对偶性问题(如正余弦)的螺旋归纳法,解决同时与两个正整数相关的命题的二重归纳法等,但它们的道理都是类似的。

数学归纳法的奠基也并非一定要从1开始。


\subsection{抽屉原理}
\label{sec:drawer-principle}



%%% Local Variables:
%%% mode: latex
%%% TeX-master: "../../book"
%%% End:
