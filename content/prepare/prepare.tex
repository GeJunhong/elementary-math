
\chapter{绪论}
\label{chap:general-topic}

本章讲述一些与其它各章节没有太大直接关联的主题。

\section{一般主题}

\subsection{数学归纳法的变种}
\label{sec:mathematical-induction}
除了常用的第一数学归纳法以外,我们还有以下的:
\begin{principle}[第二数学归纳法]
如果与正整数有关的命题$P(n)$满足:
  \begin{enumerate}
  \item $P(1)$成立;
  \item 由$P(1),P(2),\dots,P(k)$成立能够推证出$P(k+1)$成立;
  \end{enumerate}
那么该命题对于一切正整数成立.
\end{principle}
\begin{principle}[第三数学归纳法]
如果与正整数有关的命题$P(n)$满足:
  \begin{enumerate}
  \item 有无穷多个正整数$n$使命题成立;
  \item 由$P(k)$成立能够推证$P(k-1)$成立;
  \end{enumerate}
那么该命题对于一切正整数成立。
\end{principle}
\begin{principle}[跳跃数学归纳法]
  如果与正整数有关的命题$P(n)$满足:
  \begin{enumerate}
  \item $P(1)$,$P(2)$,$\ldots$,$P(m)$成立;
  \item 由$P(k)$成立能够推证$P(k+m)$成立;
  \end{enumerate}
那么该命题对于一切正整数成立。
\end{principle}


\subsection{容斥原理}
\label{sec:inclusion-exclusion-principle}

\begin{principle}[容斥原理]
  用$|A|$表示集合$A$的元素个数,则多个集合的并集的元素个数是:
  \begin{equation}
    \label{eq:inclusion-exclusion-principle}
    \begin{split}
    |\cup_{i=1}^nA_i|= & \sum_{i=1}^n|A_i|-\sum_{1\leqslant i <j \leqslant n}|A_i \cap A_j|+\sum_{1 \leqslant i <j <k \leqslant n}|A_i\cap A_j \cap A_k| \\
 & -\cdots+(-1)^{n+1}|\cap_{i=1}^nA_i|
    \end{split}
  \end{equation}
\end{principle}
其证明用数学归纳法即可,这里从略。

\subsection{转轴公式}
\label{sec:codinatie-axias-rotation}
假如在平面直角坐标系中,有一个点的坐标是$P(x_0,y_0)$,现在我们把坐标轴绕着原点逆时针方向转动一个角度$\theta$,我们来寻求这个点在新的坐标系下的坐标(这个点不跟着坐标轴旋转)。

设向量$\overrightarrow{OP}$在与原坐标系$x$正向成角$\alpha$,则该向量与新坐标系$x$正向成角$\alpha-\theta$,记向量$\overrightarrow{OP}$长度为$r$,则
\begin{equation*}
  x_0=r\cos{\alpha},y_0=r\sin{\alpha}
\end{equation*}
假设点$P$在新坐标系下的坐标为$P(x_1,y_1)$,则
\begin{equation*}
  x_1=r\cos{\alpha-\theta},y_1=r\sin{\alpha-\theta}
\end{equation*}
于是就有
\begin{equation}
  \label{eq:formulas-rotation-axias}
  \begin{split}
  x_1 & = x_0\cos{\theta} + y_0\sin{\theta} \\
  y_1 & = -x_0\sin{\theta} + y_0\cos{\theta}
  \end{split}
\end{equation}
这个旋转方式也可以看作坐标系不动,点$P$绕原点顺时针旋转。



%%% Local Variables:
%%% mode: latex
%%% TeX-master: "../../book"
%%% End:
