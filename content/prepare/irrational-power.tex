
\section{无理指数幂的定义问题}
\label{sec:irrational-power}

在中学数学的课本中,没有定义指数为无理数的幂,然而却提出了定义在全实数域上的指数函数,这是因为无理指数幂的定义要用到极限,本文的目的是为了在初等数学范围内给无理指数幂作一个解释,以解答中学生对此问题可能的疑惑。

在高等数学中,对无理指数幂的定义是,对于一个无理数$r$和一个实数$a>0$,用任意一个以$r$为极限的有理数的序列$r_i(i=1,2,\ldots)$去逼近它,无理指数幂$a^r$的定义为$a^r=\lim_{n \to \infty}a^{r_n}$.

这个极限值是与序列$r_n$无关的,用任何一个以$r$为极限的有理数序列,所得的那个极限都是相同的。这就是高等数学中无理指数幂的定义。

接下来我们尝试在初等数学范围内解释一下无理指数幂定义。

先回顾一下在中学数学范围内是如何定义出一个实数的。

对于字面上能够写出来的数,比如整数2与小数3.2,我们清楚它的每一个数位上的数字是多少,我们就认为我们定出了一个数。比如3.2的个位是3,十分位为2,其余为零。

有些数我们是不可能把它的每一位都写出来的,比如说无限循环小数$\frac{1}{3}$,但是我们清楚它的小数部分每一位都是3,这样我们也认为我们定出了一个数。

圆周率$\pi$是一个自然界中存在的常数,这个无限不循环小数我们也不能把它的每一位都写出来,甚至我们为了确定它的小数部分第100位数是几都得经过一番计算,但无论如何,它的每一位数我们也是能够确定的,只是这需要一些计算而已\footnote{这计算需要利用高等数学中的级数展开式}。

还有一些数,比如$\ln{2}$,我们是用它所满足的一些性质来刻画它的,如果要问$ln2$的某一数位上的数字是几,我们也需要通过一些计算步骤才能得出\footnote{这同样需要利用高等数学中的级数展开式}。

所以我们可以说,我们认为我们定出了一个数,当且仅当我们能够回答出来这个数的每一个数位上的数字是多少,无论这个回答是立刻就能作出的,还是需要经过一系列的运算。

有了这点认识,我们要定义如像$2^{\sqrt{3}}$这样的无理指数幂,我们只要能够按照某种规则确定出它的每一个数位上的数值就行了。

%%% 先说明一下,所有实数都可以用无限小数表示,这并不是简单的把有限小数后面补无穷个零,而是把有限小数的最后一个不为零的数位减一,这一位之后的所有的数位全部改为9,比如说,整数1就成了$0.\dot{9}$,$3.2$就成了$3.1\dot{9}$,这个表示法不会改变数值的值\footnote{这要利用等数列的无穷求和才能证明。}。这样表示的好处是:对任何两个实数比较大小时,可以直接按数位比较就行了,按数位从高到低的顺序\footnote{双端无穷序列。},找出第一个不相同的数位,谁的这一位大,谁就大。这种表示法就避免了当1与$0.\dot{9}$比较大小时所带来的特例问题。本文以下提到的实数,都是用这种表示法来表示。

把所有实数都用小数的形式来表示,即
\begin{equation}
  \label{eq:decimal-expression-of-real}
x=\cdots x_2x_1x_0.x_{-1}x_{-2}\cdots(0 \leqslant x_i \leqslant 9)
\end{equation}
在这种表示下,对于两个实数$x$和$y$,如果从数从高到低的顺序(双端无穷序列),第一个不相同的数位上,$x$的该数位大于$y$的该数位,则有$x \geqslant y$,注意到此处有个等号,这可以从1与$0.\dot{9}$的比较中看出这个等号是不能舍去的\footnote{至于1为何与$0.\dot{9}$相等需要用到收敛等比数列的无穷和。},如果所有数位都相同,则必有$x=y$。

为了避免产生1与$0.\dot{9}$所导致的问题,我们约定,如果一个无限循环小数的循环部分是9,我们就把它收上来,也就是说$1.12\dot{9}=1.13$,在这样的约定下,刚才比较大小时如果第一个不相同的数位上谁大,谁的值就大。

提一下一个实数的不足近似值和过剩近似值的概念,对于任何一个实数,它的$n$位 \emph{不足近似值} 是
\begin{equation}
  \label{eq:lower-approximate-value-nth}
\underline{x_n}=\cdots x_2x_1x_0.x_{-1}x_{-2}\cdots x_{-n}
\end{equation}
它的$n$位 \emph{过剩近似值} 是
\begin{equation}
  \label{eq:upper-approximate-value-nth}
\overline{x_n}=\cdots x_2x_1x_0.x_{-1}x_{-2}\cdots x_{-n} + 10^{-n}
\end{equation}
通俗的说法就是,$n$位不足近似值是舍掉小数点后第$n$位以后的数位,而$n$位过剩近似值则是在把这以后的数位收上来。易知序列$\underline{x_n}$单调不减而序列$\overline{x_n}$单调不增。

从定义可以看出,对于任何一个实数$x$,不等式$\underline{x_n} \leqslant x \leqslant \overline{x_n}$永远成立,但是两边的等号不能同时成立。更进一步,对于任意两个正整数$m$和$n$,实际上有$\underline{x_m} < \overline{x_n}$ 成立,这个很容易说明,假定$m<n$,就有$\underline{x_m} \leqslant \underline{x_n} \leqslant x \leqslant \overline{x_n}$,而式中的后两个等号又不能同时取到,所以有此结论。

现在来考虑无理指数幂,对于一个实数$a>0$和一个无理数$r$,先假定底数$a>1$,我们作出$r$的不足近似值序列$\underline{r_n}$和过剩近似值序列$\overline{r_n}$,因为$\underline{r_n}$和$\overline{r_n}$都是有理数,我们作两个序列$L_n=a^{\underline{r_n}}$和$M_n=a^{\overline{r_n}}$,在$a>1$的假定下对任意两个正整数$m$和$n$就有$L_m<M_n$。这就是说,任何一个$L_m$都小于所有的$M_n$,任何一个$M_n$都大于所有的$L_m$。

现在我们来证明,存在一个实数$P$,对于任意正整数$n$都有$L_n<P<M_n$。

我们作一个实数$P$,它的整数部分同$L_0$,它的第$n$位小数同$L_n$的第$n$位小数,因为对于任意正整数$n$,$L_n$都是确定的,所以这样确实可以确定出一个实数。

根据作法可以得出$L_n \leqslant P$,




%%% Local Variables:
%%% mode: latex
%%% TeX-master: "../../book"
%%% End:
