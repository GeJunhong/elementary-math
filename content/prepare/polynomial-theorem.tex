
\section{多项式乘幂定理}
\label{sec:polynomial-power-theorem}

从中学数学教材中熟知有如下的二项式定理
\begin{theorem}[二项式定理]
  对于任意两个实数$x$和$y$,以及任意的正整数$n$,有如下等式:
  \begin{equation}
    \label{eq:binomial-theorem}
    (x+y)^n = \sum_{i=0}^n C_n^i x^iy^{n-i}
  \end{equation}
  其中每一项的系数$C_n^i=\frac{n!}{i!(n-i)!}$称为 \emph{二项式系数}。
\end{theorem}

利用数学归纳法,证明是很容易的,此处略去。

把这定理推广到多个数相加的情况,就有如下的多项式乘幂定理
\begin{theorem}[多项式乘幂定理]
  对于任意$m$个实数$x_i(i=1,2,\ldots,m)$,以及任意的正整数$n$,有如下等式:
  \begin{equation}
    \label{eq:polynomial-theorem}
    \left( \sum_{i=1}^m x_i \right)^n = \sum_{r_{i} \geqslant 0,\sum_{i=1}^{m}r_{i}=n} \frac{n!}{r_1!r_2!\cdots r_m!}x_1^{r_1}x_2^{r_2}\cdots x_m^{r_m}
  \end{equation}
  其中的指数组合$(r_1,r_2,\ldots,r_{m})$要遍及方程$\sum_{i=1}^{m}r_{i}=n$的所有非负整数解,每一项的系数$\frac{n!}{r_1!r_{2}!\cdots r_{m}!}$称为 \emph{多项式系数}。
\end{theorem}

对加数的个数$m$使用数学归纳法,并利用二项式定理,便可以证明此定理,此处同样略去。

这公式随着次数和加数个数的增加会迅速变长,它的项数就是方程$\sum_{i=1}^mr_i=n$的非负整数解的个数,这个值是$C_{n+m-}^{m-1}$(参见\ref{sec:solve-number-of-equation}小节),比如说,对于$n=3,m=3$的情况,把这公式写出来就是:
\begin{equation*}
  (a+b+c)^3 = a^3+b^3+c^3+3a^2b+3a^2c+3b^2a+3b^2c+3c^2a+3c^2b+6abc
\end{equation*}

为了以后书写方便,我们引入所谓轮换求和和对称求和的符号,对于三个实数$a,b,c$而言,轮换求和是
\begin{equation*}
  \sum_{cyc} f(a,b,c) = f(a,b,c) + f(b, c, a) + f(c, a ,b)
\end{equation*}
而对称求和是
\begin{equation*}
  \sum_{sym} f(a,b,c) = f(a, b, c) + f(a, c, b) + f(b, c, a) + f(b, a, c) + f(c, a, b) + f(c, b, a)
\end{equation*}
即轮换求和是将序列$a,b,c$首尾相接后轮换进行求和,而对称求和是要对所有可能的排列进行求和。

在这种符号下,$(a+b+c)^3$ 的展开可以写成如下这样:
\begin{equation*}
  (a+b+c)^3 = \sum_{sym}a^3+3\sum_{sym}a^2b+6abc
\end{equation*}
而$(a+b+c+d)^4$的展开则是
\begin{equation*}
  (a+b+c+d)^4 = \sum_{sym} a^4 + 4 \sum_{sym}a^3b + 6\sum_{sym}a^2b^2+12\sum_{sym}a^2bc+24abcd
\end{equation*}

现在利用这些简化的符号来表达多项式乘幂定理,可以看出,定理公式中因子次数组合相同的项都具有相同的系数,把系数提出来就是对称求和,所以这定理可以改写为
\begin{equation*}
  \left( \sum_{i=1}^m x_i \right)^n = \sum_{r_i \geqslant 0, \sum_{k=1}^m r_k=n} \frac{n!}{r_1!r_2!\cdots r_n!} \sum_{sym} x_1^{r_1}x_2^{r_2}\cdots x_n^{r_n}
\end{equation*}

\begin{example}[均值不等式]
  作为多项式乘幂定理的一个应用,我们来证明均值不等式: 对任意$n$个正实数$x_i(i=1,2,\ldots,n)$,有下面不等式成立:
  \begin{equation*}
    \sqrt[n]{x_1x_2\cdots x_n} \leqslant \frac{x_1+x_2+\cdots x_n}{n}
  \end{equation*}
  由于证明过程写起来比较晦涩,先写出$n=3$的过程示例,以帮助理解,以$a,b,c$标记这三个数,由$(a^k+b^k)-(a^{k-1}b+ab^{k-1})=(a-b)(a^{k-1}-b^{k-1}) \geqslant 0$得$a^k+b^k \geqslant a^{k-1}b+ab^{k-1}$,于是
  \begin{equation*}
    \sum_{sym}a^2b = \frac{1}{2}\sum_{sym}(a^2b+bc^2) = \frac{1}{2} \sum_{sym}b(a^2+c^2) \geqslant \frac{1}{2}\sum_{sym}2bac = 6abc
  \end{equation*}
  同样
  \begin{equation*}
    \sum_{sym}a^3 = \sum_{sym}\frac{a^3+b^3}{2} \geqslant \sum_{sym}\frac{a^2b+ab^2}{2}=\frac{1}{2}\sum_{sym}a^2b \geqslant 3abc
  \end{equation*}
  所以最终
  \begin{equation*}
    (a+b+c)^3=\sum_{sym}a^3+3\sum_{sym}a^2b+6abc \geqslant 3abc+18abc+6abc=27abc
  \end{equation*}
  于是三元均值不等式得证。

  可以看出,这就是一个不断平衡各个因子次数的过程,它基于$a^k+b^k\geqslant a^{k-1}b+ab^{k-1}$这个基本的不等式,下面我们把这个过程一般化。

  只需要证明如下的不等式
  \begin{equation*}
    (x_1+x_2+\cdots+x_n)^n \geqslant n^n x_1x_2\cdots x_n
  \end{equation*}
  我们考虑左边按照多项式乘幂定理展开后的对称求和的通项
  \begin{equation*}
    \frac{n!}{r_1!r_2!\cdots r_n!}\sum_{sym}x_1^{r_1}x_2^{r_2}\cdots x_n^{r_n}
  \end{equation*}
  这个对称求和是针对$r_1,r_2,\ldots,r_n$这个组合的各种可能的排列的,把右边的对称求和(不要系数)记作$\sigma(r_1,r_2,\ldots,r_n)$,我们来平衡这些次数。

  理想的次数平衡是每个因子的次数$r_i$都是1,如果不是如此的话,因为这$\sum_{i=1}^nr_i=n$,所以必然存在某两个$r_i$和$r_j$使得$r_i-1 \geqslant r_j+1$即$r_i - r_j \geqslant 2$,我们把这两个次数平衡一次,变成$r_i-1$和$r_j+1$,基本不等式是
  \begin{equation*}
  (x_1^{r_i}y_2^{r_j}+x_2^{r_i}x_1^{r_j})-(x_1^{r_i-1}x_2^{r_j+1}+x_2^{r_i-1}x_1^{r_j+1})=x_1^{r_j}x_2^{r_j}(x_1-x_2)(x_1^{r_i-r_j-1}-x_2^{r_i-r_j-1}) \geqslant 0
  \end{equation*}
  所以
  \begin{eqnarray*}
    \sigma(r_1,\ldots,r_i,\ldots,r_j,\ldots,r_n) & = & \sum_{sym}x_i^{r_i}x_j^{r_j}\cdots \\
 & = & \frac{1}{n-1} \sum_{sym}(x_i^{r_i}x_j^{r_j}+x_i^{r_j}x_j^{r_i})\cdots \\
 & \geqslant & \frac{1}{n-1} \sum_{sym} (x_1^{r_i-1}x_2^{r_j+1}+x_2^{r_i-1}x_1^{r_j+1})\cdots \\
                               & = & \frac{1}{2(n-1)} \sum_{sym}x_1^{r_i-1}x_2^{r_j+1}\cdots \\
    & = & \frac{1}{2(n-1)} \sigma(r_1,\ldots,r_i-1,\ldots,r_j+1,\ldots,r_n)
  \end{eqnarray*}
\end{example}

%%% Local Variables:
%%% mode: latex
%%% TeX-master: "../../book"
%%% End: