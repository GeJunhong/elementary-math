
\section{坐标变换}
\label{sec:codinatie-axias-rotation}

在高级中学数学教材中,三角函数的伸缩变换已经为人所熟知,而本文要讨论的是坐标变换的一般性理论。

\subsection{平移}

\subsection{旋转}
假如在平面直角坐标系中,有一个点的坐标是$P(x_0,y_0)$,现在我们把坐标轴绕着原点逆时针方向转动一个角度$\theta$,我们来寻求这个点在新的坐标系下的坐标(这个点不跟着坐标轴旋转)。

设向量$\overrightarrow{OP}$在与原坐标系$x$正向成角$\alpha$,则该向量与新坐标系$x$正向成角$\alpha-\theta$,记向量$\overrightarrow{OP}$长度为$r$,则
\begin{equation*}
  x_0=r\cos{\alpha},y_0=r\sin{\alpha}
\end{equation*}
假设点$P$在新坐标系下的坐标为$P(x_1,y_1)$,则
\begin{equation*}
  x_1=r\cos{(\alpha-\theta)},y_1=r\sin{(\alpha-\theta)}
\end{equation*}
于是就有
\begin{equation}
  \label{eq:formulas-rotation-axias}
  \begin{split}
  x_1 & = x_0\cos{\theta} + y_0\sin{\theta} \\
  y_1 & = -x_0\sin{\theta} + y_0\cos{\theta}
  \end{split}
\end{equation}
这个旋转方式也可以看作坐标系不动,点$P$绕原点顺时针旋转。

\subsection{伸缩}


%%% Local Variables:
%%% mode: latex
%%% TeX-master: "../../book"
%%% End:
