
\section{行列式}
\label{sec:determinant}

\subsection{奇排列与偶排列}
\label{sec:odd-permutation-and-even-permutation}

为了讨论行列式的每一项的符号,这一小节来讨论下奇排列与偶排列。

我们来考虑正整数$1,2,\ldots,n$的任一排列,称其为一个$n$级排列,交换其中两个元素的位置的操作称为一个 \emph{对换}, 我们先指出如下定理
\begin{theorem}
  \label{theorem:all-rangement-can-be-producted-by-exchange}
  正整数$1,2,\ldots,n$的任一排列$a_1,a_2,\ldots,a_n$均可由原始排列$1,2,\ldots,n$经过有限次对换得到,同样它也可以经过有限次对换得出原始排列$1,2,\ldots,n$。
\end{theorem}

\begin{proof}[证明]
  我们给出一个实际的对换方案,如果$a_1 \neq 1$,则必然有某个$a_i=1(i>1)$,于是我们先对换$a_1$和$a_i$,对换之后就可以使得新的$a_1=1$了,然后再看$a_2$是否等于2,如果不是则做类似处理,依次这样下去,最多对换$n-1$次,便可以得出原始排列$1,2,\ldots,n$,这是一个对换方案,然后把这个对换方案做一个逆对换,便可以由原始排列$1,2,\ldots,n$得出排列$a_1,a_2,\ldots,a_n$.
\end{proof}

显然由这定理可知,任意两个不同的$n$级排列都可以通过对换操作互相转换。

现在我们准备引入排列的奇偶性,为此先给出如下结论
\begin{theorem}
  如果一个排列是经由原始排列$1,2,\ldots,n$通过奇数次对换得到,则它也只能经由奇数次对换得到,同样,如果它是由原始排列经过偶数次对换得到,则它也只能通过偶数次对换得到。
\end{theorem}

\begin{proof}[证明]
  我们对排列的级数$n$作数学归纳法。显然$n=1,2$时结论均成立,假定结论对小于等于$n$级的排列都成立,我们来考虑$n+1$级排列的情况。

  假定一个$n+1$级排列$a_1,a_2,\ldots,a_n,a_{n+1}$是由原始排列$1,2,\ldots,n,n+1$经由奇数次对换得到的,
\end{proof}

由这定理可知对于一个确定的$n$级排列,由原始排列得到它的对换次数的奇偶数是确定的,根具体的对换方案无关,仅由排列本身确定,为此我们把只能由原始排列通过奇数次对换而得到的排列称为\emph{奇排列},而把只能由偶数次对换而得到的排列称为\emph{偶排列},根据上面两个定理可知,一个排列,必然要么是奇排列要么是偶数列,不能两者都是或者两者都不是。

由上述定理立即可得:
\begin{theorem}
  对排列作一次对换将改变其奇偶性。
\end{theorem}

虽然我们定义了排列的奇偶性,但对于一个具体的排列,要指出它是奇排列还是偶排列却还是不容易的,最简单的办法就是按照\autoref{theorem:all-rangement-can-be-producted-by-exchange}的证明过程中的步骤构成出一个对换方案来,并由其对换次数来确定排列的奇偶性,但我们还有更可行的方法,为此提出 \emph{正序} 与 \emph{逆序} 的概念。
\begin{definition}
  对于正整数$1,2,\ldots,n$的任一排列$a_1,a_2,\ldots,a_n$,如果其中两个元素$a_i$和$a_j$满足$i<j$且$a_i<a_j$,则称它们是这排列中的一个 \emph{正序},如果它们满足$i<j$而$a_i>a_j$,则称其为这排列中的一个 \emph{逆序},一个排列中所有正序的个数称为 \emph{正序数},所有逆序的个数称为 \emph{逆序数}。
\end{definition}

显然,一个排列中的正序数与逆序数之和是$C_n^2=n(n-1)/2$.

我们指出
\begin{theorem}
  \label{theorem:odevity-of-rangement-equivalent-to-its-inversion-odvity}
  一个排列的逆序数,与排列本身具备相同的奇偶性。
\end{theorem}

为了证明这个定理,我们先证如下引理
\begin{lemma}
  \label{lemma:all-rangement-can-be-producted-by-adjacent-exchange}
  正整数$1,2,\ldots,n$的任一排列都可以由原始排列$1,2,\ldots,n$通过有限次相邻元素的对换得到。
\end{lemma}

\begin{proof}[证明]
  我们仍然实际构造出一个方案,假设这排列是$a_1,a_2,\ldots,n$,如果$a_1 \neq 1$,则在原始排列中,把数$a_i$依次跟它前一个元素对换,直到把它移动第一个元素的位置,同样的方法逐个应用到$a_2,a_3,\ldots,a_n$,即可在有限步内将原始排列$1,2,\ldots,n$变成$a_1,a_2,\ldots,a_n$。
\end{proof}

有了这引理,我们回到\autoref{theorem:odevity-of-rangement-equivalent-to-its-inversion-odvity}的证明上
\begin{proof}[证明]
   由\autoref{lemma:all-rangement-can-be-producted-by-adjacent-exchange},排列可以由原始排列通过相邻元素的对换而得来,而这对换的次数的奇偶性又是排列自身确定的,而显然相邻元素的对换必然改变逆序数的奇偶性,所以经过多少次相邻元素的对换,逆序数的奇偶性就改变多少次,而原始排列$1,2,\ldots,n$的逆序数是零,所以逆序数的奇偶性,与对换次数的奇偶性相同,即与排列的奇偶性相同。
\end{proof}

显然,利用逆序数是容易求得一个具体排列的奇偶性的。

\subsection{行列式的概念}
\label{sec:determinant-concept}

我们把二级行列式和三级行列式的概念推广到任意$n\times n$的数阵上,对于一个$n \times n$的数阵
\[
  \begin{pmatrix}
    a_{11} & a_{12} & \cdots & a_{1n} \\
    a_{21} & a_{22} & \cdots & a_{2n} \\
    \vdots & \vdots & \vdots & \vdots \\
    a_{n1} & a_{n2} & \cdots & a_{nn}
  \end{pmatrix}
\]
它的行列式是所有位于不同行且不同列的元素的乘积的代数和,也就是说,求和中的每一项如果不考虑符号都具备形式
\[ a_{i_1j_1}a_{i_2j_2} \cdots a_{i_nj_n} \]
其中$i_1,i_2,\ldots,i_n$和$j_1,j_2,\ldots,j_n$都是$1,2,\ldots,n$的一个排列,而这求和中将会遍及所有可能的排列。我们把上式按$i_k$的顺序重新排列成
\[ a_{1j_1}a_{2j_2} \cdots a_{nj_n} \]
这时$j_1,j_2,\ldots,j_n$已经不再是原来的排列了,我们只是仍使用相同的符号,然后这一项的符号是这样定义的:当$j_1,j_2,\ldots,j_n$是奇排列是带负号,为偶排列时带正号,这样所有这样的项的和便是由数阵所决定的行列式的值。
\[
  \begin{vmatrix}
    a_{11} & a_{12} & \cdots & a_{1n} \\
    a_{21} & a_{22} & \cdots & a_{2n} \\
    \vdots & \vdots & \vdots & \vdots \\
    a_{n1} & a_{n2} & \cdots & a_{nn}
  \end{vmatrix}
  = \sum_{j_1,j_2,\ldots,j_n} (-1)^{\tau (j_1,j_2,\ldots,j_n)} a_{1j_1}a_{2j_2} \cdots a_{nj_n}
\]

\subsection{行列式的性质与计算}
\label{sec:determinant-properties}

\subsection{行列式按一行(列)展开}
\label{sec:determinant-expand-by-row-or-column}

\subsection{Cramer法则}
\label{sec:cramer-rule}

\subsection{Laplace定理}
\label{sec:laplace-theorem}




%%% Local Variables:
%%% mode: latex
%%% TeX-master: "../../book"
%%% End:
