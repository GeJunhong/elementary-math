
\section{函数方程}
\label{sec:function-equation}

函数方程,顾名思义是关于函数的方程,它是以函数为未知量的方程,例如方程$f(x+y)=f(x)+f(y)$,这个方程中的未知量是函数$f(x)$. 如果某个函数能够满足某一函数方程,则称它是这函数方程的一个解,求出函数方程的全部解的过程称为解函数方程。

\subsection{柯西方程}
\label{sec:cauchy-function-equality}

这一节介绍柯西研究过的一个函数方程:定义在实数集$R$上的$f(x)$对于任意两个实数$x$和$y$(可以相等)都成立$f(x+y)=f(x)+f(y)$,我们要讨论的问题不外乎几点:这样的函数存在吗?如果存在,是否唯一?函数表达式是否有通式?能否求出?

显然任意正比例函数$f(x)=ax$都满足这函数方程,所以问题是还有没有其它形式的解。

首先取$x=y=0$得出$f(0)=0$,于是$0=f(0)=f(x+(-x))=f(x)+f(-x)$,所以$f(x)$是奇函数。

再令$x=y$得$f(2x)=2f(x)$,不难根据数学归纳法得到,对任意正整数$n$和任意实数$x$,有$f(nx)=nf(x)$,再由$f(x)$是奇函数,知这等式对于负整数也同样成立。

在等式$f(nx)=nf(x)$中取$x=1$,便知对于任意整数$n$,有$f(n)=nf(1)$,也就是说,整个整数集上的函数值,全由$x=1$处的函数值决定。

接着转而讨论有理数的情形,对于整数$p$,有
\[ f(1)=f \left( p\cdot \frac{1}{p} \right) = p f \left( \frac{1}{p} \right) \]
所以
\[ f \left( \frac{1}{p} \right) = \frac{1}{p} f(1) \]
进一步,对于任意有理数$\dfrac{q}{p}$,有
\[ f \left( \frac{q}{p} \right) = f \left( q \cdot \frac{1}{p} \right) = q f \left( \frac{1}{p} \right) = \frac{q}{p} f(1) \]
所以,对于一切有理数$x$,都有
\[ f(x) = x f(1) \]
于是全体有理数集上的函数值,都由$f(1)$所唯一确定。

到目前为止,事情都很顺利,但是当讨论到无理数时,情况就不同了。因为前面之所以顺利,是因为由0和1,经过有限次加减,可得出全体整数,再经过乘除,可以得出全体有理数,但是无理数,却不能这么简单的得出,讨论的函数方程对于开方都无能为力,更不用说对于圆周率$\pi$这样的无理数了。

任意待定一个无理数$\pi$(为了突出它的无理性,选用了圆周率符号,但并不表示圆周率),定义集合$Q_{\pi}={x|x=a+b\pi,a,b\in Q}$,称为由无理数$\pi$生成的无理数集,显然如果$x_1,x_2 \in Q_{\pi}$,则$x_1+x_2$以及$\lambda x_1 (\lambda \in Q)$也都属于这集合,设$x=a+b\pi \in Q_{\pi}$,称$a$为它的有理部分,$b$为它的无理部分(的系数)。

在此定义下,我们将看到,原来的函数方程对于有理数上的取值和无理数上的取值是互相独立的,比如说规定当$x \in Q_{pi}$时,它的函数值等于它的有理部分的函数值,即$f(a+b\pi)=f(a)$,现在来验证,这样定义的函数符合原来的函数方程,这只要验证$x$和$y$中有无理数就行了。

设$x$为有理数,而$y=a+b\pi$为无理数,则$f(x+y)=f((x+a)+b\pi)=f(x+a)=f(x)+f(a)$,而$f(x)+f(y)=f(x)+f(a)$,因此$f(x+y)=f(x)+f(y)$.由交换律,$x$为无理数而$y$为有理数的情形也是成立的。

再设$x_1=a_1+b_1\pi$,$x_2=a_2+b_2\pi$,则$f(x_1+x_2)=f((a_1+a_2)+(b_1+b_2)\pi)=f(a_1+a_2)=f(x_1)+f(x_2)$,这就证明了当$x_1$与$x_2$都是无理数且都从属于同一个$Q_{\pi}$时,函数方程是成立的。

如果$x_1$与$x_2$分属于不同的$Q_{\pi}$.

\subsection{指数方程与对数方程}
\label{sec:exponent-equation-and-logarithm-equation}

\subsection{三角方程}
\label{sec:triangle-function-equation}

三角方程是根据三角函数所满足的关系式所提出来的,根据正余函数和余弦函数的和角公式,有
\begin{eqnarray*}
  \cos{(x+y)} & = & \cos{x}\cos{y} - \sin{x}\sin{y} \\
  \sin{(x+y)} & = & \sin{x}\cos{y} + \cos{x}\sin{y}
\end{eqnarray*}

所以很自然的提出如下函数方程:两个函数$C(x)$和$S(x)$,对于任何两个实数$x$和$y$都成立
\begin{eqnarray*}
  C{(x+y)} & = & C{x}\C{y} - S{x}\S{y} \\
  S{(x+y)} & = & S{x}\C{y} + C{x}\S{y}
\end{eqnarray*}
显然正弦函数和余弦函数是其一解,所以问题是,是否是唯一解。




%%% Local Variables:
%%% mode: latex
%%% TeX-master: "../../book"
%%% End:
