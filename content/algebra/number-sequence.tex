
\section{数列}
\label{sec:number-sequence-general}

\subsection{数列概念}

数列就是一个数的序列,可以是有限序列也可以是无限序列,按下标记法可以写成$a_1,a_2,\ldots$,有时为了方便也可以让下标从零开始$a_0,a_1,a_2,\ldots$,甚至作为无穷序列时,还可以把下标扩展到负整数:$\cdots,a_{-2},a_{-1},a_0,a_1,a_2,\cdots$,成为一个双向无穷数列,如果有一个公式可以把数列的每一项表成它的下标的函数,就称此公式为该数列的\emph{通项公式}。数列本质上是定义在整数集或者正整数集上的函数。

\begin{example}
  任一实数在十进制下可以表为
  \[ a_na_{n-1}\cdots a_1a_0.a_{-1}a_{-2}\cdots \]
  其中$a_i$取值范围是$0,1,\ldots,9$,这便可以视为一个数列,而且是单侧无穷数列,而这个实数的值是
  \[ a_n\cdot 10^n + a_{n-1} \cdot 10^{n-1} + \cdots + a_1 \cdot 10 + a_0 \cdot 10^0 + a_{-1} \cdot 10^{-1} + a_{-2} \cdot 10^{-2} +\cdots \]
\end{example}

除了通项公式之外,\emph{递推公式}也是研究数列的一个重要手段,所谓递推公式是指一个联系着数列中相邻若干项的一个公式,例如可以这样定义一个数列,约定$a_1=1$,$a_2=1$,以后的每一项都由它前面的两项之和来定义,即$a_{n+2}=a_{n+1}+a_n(n \geqslant 1)$,显然这公式唯一的确定了数列的每一项,也就确定了数列本身,它就是有名的 \emph{斐波那契(Fibonacci)数列},后文还会讨论它。

\subsection{等差数列与等比数列}

等差数列和等比数列是最常见的两种数列。

\begin{definition}
 如果一个数列$\ldots,a_{-2},a_{-1},a_0,a_1,a_2,\ldots$中,任一项减去它前一项所得差值都相同,即存在常数$d$,对任意整数$n$都有$a_n-a_{n-1}=d$,则称此数列为 \emph{等差数列},而常数$d$称为此等差数列的\emph{公差}.
\end{definition}

显然,正整数数列$\ldots,-2,-1,0,1,2,\ldots$是公差为1的等差数列,任何项都取相同值的\emph{常数数列}则是公差为零的等差数列。

等差数列的递推公式也可以用$a_{n+1}+a_{n-1}=2a_n$来刻画,而且其中不出现公差,仅仅通过项的关系来就定义出等差数列,这也表明,数列的递推公式可以不唯一。

\begin{theorem}
对于等差数列中的任意两项(无论顺序)有
\[ a_n-a_m=(n-m)d \]
\end{theorem}

\begin{proof}[证明]
证明是容易的,只要证明$n\geqslant m$的情况即可(否则可以反过来相减),首先可以验证$n$与$m$相等时该等式成立,而后有
\[ a_n-a_m=\sum_{k=m}^{n-1}(a_{k+1}-a_k)=(n-m)d \]
当然,对差值$n-m$施行归纳法也是可行的。
\end{proof}

假定我们知道$a_0$和公差$d$,在上式中取$m=0$,我们就得到等差数列的通项公式
\begin{theorem}
  等差数列的每一项可以由$a_0$及公差$d$表为
\[ a_n=a_0+nd \]
这便是等差数列的通项公式.
\end{theorem}

当然通项并不一定非得使用$a_0$来表示,$a_n=a_1+(n-1)d$,$a_n=a_{100}+(n-100)d$也都是通项。

易见这通项是关于$n$的一次函数,实际上容易证明,如果数列的通项可表成关于下标的一次函数,则它必然是等差数列。

现在考虑对等差数列进行某种求和,如果从$a_0$开始,向正下标或者负下标进行累加,$n$是任意一个整数(无论正负),记$S_n=\sum_{i=0}^{n}a_n$,则
\[ S_n=\sum_{i=0}^na_i=\sum_{i=0}^n(a_0+nd)=(n+1)a_0+d\sum_{i=0}^ni\]
于是这个求和就归结为对自然数序列$0,1,2,\ldots,n$进行求和,高斯曾经口算出$1+2+3+\cdots+100=5050$,他的计算方法是,将1与100相加得101,2与99相加得101,如此这般,最后50与51相加得101,于是$1+2+\cdots+100=50 \times 101=5050$,把这个方法略加改造便得出如下的计算$1+2+\cdots+n$的方法,为了避免$n$的奇偶性带来分组分不完从而还需要分奇偶讨论的问题,再拿一个式子$n+(n-1)+\cdots+2+1$与$1+2+\cdots+(n-1)+n$进行相加,则这两个式子对应位置上的两数之和都是$n+1$,所以最终得公式
\[ 1+2+\cdots+n = \frac{1}{2}n(n+1) \]
不必怀疑右端会有出现小数的可能,因为作为连续的两个自然数$n$和$n+1$中必定一奇一偶,所以右端永不会为小数。

把这公式应用到上面的式子便得
\begin{theorem}
  在公差为$d$的等差数列$a_n$中,有
  \[ S_n=\sum_{i=0}^na_i=(n+1)a_0+\frac{1}{2}n(n+1)d \]
  其中$n$可以是任意整数(包括负整数)。
\end{theorem}

上面求$1+2+\cdots+n$的方法称为\emph{倒序相加法},当然也可以直接把这方法应用到$S_n=a_0+a_1+\cdots+a_n$上,同样可得出上面的结果.

\begin{definition}
  如果一数列$\ldots,a_{-2},a_{-1},a_0,a_1,a_2,\ldots$中,任意一项与前一项的比值都是同一非零常数,即存在非零常数$q$,使得对于任意整数$n$都有$a_n / a_{n-1} = q$成立,则称此数列是 \emph{等比数列},而比值常数$q$称为它的 \emph{公比}.
\end{definition}

对于等比数列,如果它的各项都是正的,取对数即可变身为一个等差数列。对于所有的等比数列,同样可以得出类似于等差数列的结论来,即是

\begin{theorem}
对于公比为$q$的等比数列$\cdots,a_{-2},a_{-1},a_0,a_1,a_2,\cdots$,对任意整数$n$和任意整数$m$都成立
\[ \frac{a_n}{a_m}=q^{n-m} \]
\end{theorem}

并由此得出通项
\begin{theorem}
  公比为$q$的等比数列$\ldots,a_{-2},a_{-1},a_0,a_1,a_2,\ldots$的通项可表为
 \[ a_n=a_0q^n \]
\end{theorem}

关于等比数列的求和
\[ S_n = a_0+a_1+\cdots+a_n = a_0(1+q+\cdots+q^{n}) \]
这就归结为对$1+q+q^2+\cdots+q^{n}$求值,

由数学归纳法可得出下面的乘法公式,它也是平方差公式和立方差公式的自然推广(见\autoref{example:a-power-n-b-power-n-factoring})
\begin{equation}
  \label{eq:a-power-n-substract-b-power-n}
 a^n-b^n = (a-b)(a^{n-1}+a^{n-2}b+\cdots+ab^{n-2}+b^{n-1}) 
\end{equation}
式中$n$是任意正整数,

在式中把$n$换成$n+1$并取$a=1,b=q$便可得出,对于任意正整数$n$,成立
\[ 1+q+\cdots+q^{n} = \frac{1-q^{n+1}}{1-q} \]
若$n$为负整数,则上式中以$1/q$代$q$,便知上式仍然成立,于是对于等比数列就有
\begin{theorem}
  在以$q(q \neq 1)$为公比的等比数列$\ldots,a_{-2},a_{-1},a_0,a_1,a_2,\ldots$中,对于任意整数$n$都有
  \[ S_n = \sum_{i=0}^na_i = \frac{a_0(1-q^{n+1})}{1-q} \]
  而在$q=1$时,$S_n=(n+1)a_0$.
\end{theorem}

还可以以另外一种方式得出这公式,因为
\[ S_n = a_0+a_1+\cdots+a_n \]
两端同乘以公比$q$,便可得出
\[ qS_n = q(a_0+a_1+\cdots+a_n) = a_1+a_2+\cdots+a_n+a_{n+1} \]
于是
\[ (1-q)S_n = a_0-a_{n+1} = a_0(1-q^{n+1}) \]
所以得出
\[ S_n = \frac{a_0-a_{n+1}}{1-q} = \frac{a_0(1-q^{n+1})}{1-q} \]
当然这过程中要求$q \neq 1$,而对于$q=1$的情况,$S_n=(n+1)a_0$.

这个方法称为 \emph{错位相减法},下面这个例子提示了这个方法的一些用途。

\begin{example}[错位相减法的一个用途]
  考虑对数列$a_n = n q^n(n=1,2,3,\ldots)$进行求和,这里要求$q \neq 1$,否则就变成早已解决过的问题了。同样有
  \[ S_n = a_1+a_2+\cdots+a_n = q+2q^2+3q^3 + \cdots + (n-1)q^{n-1} + nq^n \]
  两端同乘以$q$得
  \[ qS_n = q^2+2q^3+3q^4 + \cdots + (n-1)q^n + nq^{n+1} \]
  把以上两式相减,得
  \[ (1-q)S_n = (q + q^2 + \cdots + q^n) - n q^{n+1} \]
  显然上式右端括号中的部分是等比数列的求和,于是$S_n$便可以求出。

  接下来再考虑数列$a_n=n^mq^n(n=1,2,3,\ldots)$,其中$m$是个固定的正整数,记
  \[ S_m(n) = \sum_{i=1}^na_i = \sum_{i=1}^n n^m q^n = q + 2^mq^2 + \cdots + (n-1)^mq^{n-1} + n^m q^m \]
  进行同样的过程有
  \[ qS_m(n) = q^2 + 2^mq^3 + \cdots + (n-1)^m q^n + n^m q^{n+1} \]
  两式相减
  \[ (1-q)S_m(n) = q +(2^m-1)q^2 + (3^m-2^m)q^3 + \cdots + (n^m-(n-1)^m)q^m - n^mq^{n+1}  \]
  而利用下面的公式(它可以由二项式定理得出)
  \[ (1+k)^n-k^n = 1+nk+C_n^2k^2 + \cdots + C_n^{n-1}k^{n-1} \]
  上式可化为
  \begin{eqnarray*}
    (1-q)S_m(n) & = & \sum_{i=1}^n(i^m-(i-1)^m)q^i - n^mq^{n+1} \\
                & = & \sum_{i=0}^{n-1}((1+i)^m-i^m)q^{i+1} - n^mq^{n+1} \\
                & = & \sum_{i=0}^{n-1}\sum_{j=0}^{m-1}C_m^ji^jq^{i+1} - n^mq^{n+1} \\
                & = & \sum_{j=0}^{m-1}C_m^j\sum_{i=0}^{n-1}i^jq^{i+1} - n^mq^{n+1} \\
    & = & q\sum_{j=0}^{m-1}C_m^jS_j(n-1) - n^mq^{n+1}
  \end{eqnarray*}
  这意味着$S_m(n)$可以用$S_0(n-1),S_1(n-1),\ldots,S_{m-1}(n-1)$来表示,即具有递推性,而$S_0(n-1)$显然就是等比数列求和,于是$S_m(n)$便可以求出来。

  进一步,便可以求形如$a_n=p(n)q^n$这样的数列的和,其中$p(n)$是关于$n$的多项式,只是这过程将越来越繁琐,如此机械化的计算方法,由计算机程序来进行是再合适不过了。
\end{example}

\begin{example}[自然数的幂和]
  \label{example:sum-of-power-of-integer}
  在上面得出了公式
  \[ 1+2+\cdots+n = \frac{1}{2}n(n+1) \]
  现在就来讨论下一般的$S_m(n)=\sum_{i=1}^ni^m$的求和公式,仍然由
  \[ (1+k)^m-k^m = 1 + mk + C_m^2k^2 + \cdots + C_m^{m-1}k^{m-1} \]
  对$k=1,2,\ldots,n$进行累加,便得
  \[ (1+n)^m-1 = S_0(n)+C_m^1S_1(n)+C_m^2S_2(n) + \cdots C_m^{m-1}S_{m-1}(n) \]
  显然这便是联系着诸$S_m(n)$的递推关系,把它写成更美观的形式
  \[ \sum_{i=0}^mC_m^iS_i(n) = (1+n)^{m+1}-1 \]
  且有初始公式
  \[ S_0(n) = n \]
  由此出发便能求出任何$S_m(n)$的表达式来,比如说取$m=2$便得
  \[ S_0(n)+2S_1(n)+S_2(n) = (1+n)^3-1 \]
  于是得出
  \[ S_2(n) = 1^2+2^2+\cdots+n^2 = \frac{1}{6}n(n+1)(2n+1) \]
  同样不必怀疑右端的整数性,因为$n$与$n+1$作为相邻两个正整数,必然有一个偶数,故$2 \mid n(n+1)$,还需证明$3 \mid n(n+1)(2n+1)$,若是$n$与$n+1$中有3的倍数,则自然不消说,若是$n$与$n+1$都不是3的倍数,则它俩被3除所得余数必然一个是1,另一个是2,于是$2n+1=n+(n+1)$便必然是3的倍数,于是2和3都能整除$n(n+1)(2n+1)$且2与3互素,所以6也能整除它。

  同样再取$m=3$,便可得出
  \[ S_3(n)= 1^3+2^3+\cdots+n^3 = \frac{1}{4}n^2(n+1)^2 \]
  一般的可以知道,$S_m(n)$是关于$n$的$m+1$次多项式。
\end{example}

\subsection{差分与高阶等差数列}
\label{sec:difference-and-high-level-common-difference-sequence}

\begin{definition}
  对于无穷数列$\ldots,a_{-2},a_{-1},a_0,a_1,a_2,\ldots$,定义它的 \emph{一阶差分数列} 为
  \[ \Delta a_n = a_n-a_{n-1} \]
  进一步递归的定义它的$m$阶差分数列是
  \[ \Delta^ma_n = \Delta^{m-1}a_n - \Delta^{m-1}a_{n-1} \]
  即$m$阶差分是$m-1$阶差分的差分,这称为 \emph{高阶差分},为了方便,可以把数列本身看作它自己的 \emph{零阶差分数列}.
\end{definition}

差分将一个数列变换为另一个数列,这与平方运算把一个数变成另一个数,求导运算把一个函数变成另一个函数没什么差别,所以差分是在数列这种数学对象上的一种运算。

为了讨论的方便,用$a_n^{(m)}$代表数列$a_n$的$m$阶差分数列,并规定$a_n^{(0)}=a_n$,上标加了括号,以与幂相区别,在这定义就有
\[ a_n^{(m)} = a_n^{(m-1)} - a_{n-1}^{(m-1)} \]

接下来有高阶等差数列的定义
\begin{definition}
  如果一个数列$a_n$的$m$阶差分数列是等差数列,则该数列称为 \emph{$m$阶等差数列}.
\end{definition}

显然等差数列本身即是零阶等差数列,同时,如果一个数列是$m$阶等差数列,则它也必然是$m+1$阶等差数列,$m+2$阶等差数列,等等。

易见
\begin{theorem}
  数列$a_n$是$m$阶等差数列的充分必要条件是$\Delta a_n$是$m-1$阶等差数列.
\end{theorem}

假定$a_n$是一个$m$阶等差数列,那么存在常数$d$,使得对任意整数$n$成立下式
\[ a_n^{(m)} - a_{n-1}^{(m)} = d \]
把式中的$m$阶差分数列用$m-1$阶差分的差分来替代,就得到
\[ a_n^{(m-1)}-2a_{n-1}^{(m-1)}+a_{n-2}^{(m-1)} = d \]
同样的手段再施行两次,得
\begin{eqnarray*}
 a_n^{(m-2)}-3a_{n-1}^{(m-2)}+3a_{n-2}^{(m-2)}-a_{n-3}^{(m-2)} & = & d  \\
 a_n^{(m-3)}-4a_{n-1}^{(m-3)}+6a_{n-2}^{(m-3)}-4a_{n-3}^{(m-3)}+a_{n-4}^{(m-3)} & = & d 
\end{eqnarray*}
易见各项式的系数正好便是二项式系数并交错的带上正负号,并且$m-k$阶差分数列的公式中便含有相邻$k+2$项,照此规律,当差分的阶数降为零时,公式成为
\[ C_{m+1}^0a_n-C_{m+1}^1a_{n-1}+\cdots+(-1)^{m+1}C_{m+1}^{m+1}a_{n-m-1} = d \]
这便是$m$阶等差数列的一个递推公式,但是公式中带有未知常数$d$,现在想办法去掉它,因为对于$m$阶等差数列而言,它的$m$阶差分数列成为等差数列,那么它的$m+1$阶差分数列就成为一个常数数列,也就是公差为零的等差数列,因此按照上面的公式,原来数列的递推公式可以表为
\[ C_{m+2}^0a_n-C_{m+2}^1a_{n-1}+\cdots+(-1)^{m+2}C_{m+2}^{m+2}a_{n-m-2} = 0 \]
于是得到以下重要结果
\begin{theorem}
  \label{theorem:recursive-for-high-level-common-difference-sequence}
  $m-2$阶等差数列的递推公式是
\[ C_m^0a_n-C_m^1a_{n-1}+\cdots+(-1)^mC_m^ma_{n-m} = 0 \]
\end{theorem}
这只要把上面的推导,改用数学归纳法写出来就可以证明它。

这结果对于等差数列即零阶等差数列而言就成为$a_n-2a_{n-1}+a_{n-2}=0$.

现在讨论一下高阶等差数列的通项问题,先考虑一阶等差数列$a_n$,按定义,它的一阶差分数列是等差数列,于是有
\[ a_n - a_{n-1} = rn+s \]
于是对于整数$n$(包括负整数),就有
\begin{eqnarray*}
  a_n & = & a_0 + \sum_{i=1}^n(a_i-a_{i-1}) \\
      & = &  a_0 + \sum_{i=1}^n(ri+s) \\
      & = &  a_0 + ns + r\sum_{i=1}^n i \\
  & = & a_0 + ns + \frac{1}{2}rn(n+1)
\end{eqnarray*}
这表明它的通项是关于$n$的二次多项式,再重复同样的过程并利用自然数的幂和公式(\autoref{example:sum-of-power-of-integer})便可得出二阶等差数列的通项是关于$n$的三次多项式,如此等等,于是得出如下重要结果
\begin{theorem}
  \label{theorem:common-formular-for-high-level-common-difference-sequence}
  无穷数列$\ldots,a_{-2},a_{-1},a_0,a_1$是$m$阶等差数列的充分必要条件是,它的通项能表成关于下标的$m+1$次多项式.
\end{theorem}

这结果表明,高阶等差数列本质上就是多项式数列。

\begin{proof}[证明]
  对$m$施行数学归纳法,$m=0$的情况是显然的,假设必要性对于小于$m$的正整数都成立,那么对于一个$m$阶等差数列$a_n$,它的一阶差分数列是$m-1$阶等差数列,因而通项为关于下标的$m$次多项式,于是
  \[ a_n-a_{n-1} = \sum_{j=0}^m b_jn^j \]
  累加得
  \begin{eqnarray*}
    a_n & = & a_0 + \sum_{i=0}^n(a_i-a_{i-1}) \\
        & = & a_0 + \sum_{i=0}^n\sum_{j=0}^mb_ji^j \\
    & = & a_0 + \sum_{j=0}^mb_j\sum_{i=0}^ni^j \\
  \end{eqnarray*}
  在\autoref{example:sum-of-power-of-integer}中,我们就已经知道$\sum_{i=1}^ni^j$是$j+1$次多项式,于是$a_n$便是关于$n$的$m+1$次多项式,所以必要性成立.

  再证充分性,同样施行归纳法,如果数列的通项是关于下标的一次多项式,显然它必然是(零阶)等差数列,即$m=0$时充分性成立,假若充分性对于小于$m$的正整数都成立,现在设数列$a_n$的通项是关于$n$的$m+1$次多项式,即
  \[ a_n = \sum_{i=0}^{m+1}b_in^i \]
  则它的差分数列
  \[ a_{n+1}-a_n = \sum_{i=0}^{m+1}b_i((1+n)^i-n^i) \]
  显然,$n^{m+1}$被抵销,这成为一个次数低于$m+1$的多项式,按归纳假设,这差分数列必然是某一阶的等差数列,且阶数小于$m$,于是原来的数列$a_n$也是某一阶的等差数列,且阶数小于等于$m$,自然也是$m$阶等差数列.
\end{proof}

\subsection{线性递推数列的通项}
\label{subsec:linear-recurrence-sequence}

本节讨论常系数线性递推数列的通项求法问题\footnote{线性递推数列的通项求法这部分内容,主要参考了文献\cite{olympic-math}.},这个都是有固定结论的内容,本文只是粗略转叙一番而已。

所谓常系数线性递推数列,是指它的递推公式形如$\sum_{k=1}^n \lambda _k a_k = c$的数列,例如斐波那契数列$a_{n+2}=a_{n+1}+a_n$。

先看最简单的一种,递推式为$a_{n+1}=pa_n+q$的数列,当$p=1$时它成为等差数列,当$q=0$则成为等比数列,所以此处限定$p \neq 1, q \neq 0$。

要求它的通项,只要在它两端同时除以$p^{n+1}$,就有
\[ \frac{a_{n+1}}{p^{n+1}} = \frac{a_n}{p^n} + \frac{q}{p^{n+1}} \]
因此有
\begin{eqnarray*}
\frac{a_n}{p^n} & = & \frac{a_1}{p} + \sum_{k=2}^{n}\left( \frac{a_k}{p^k} - \frac{a_{k-1}}{p^{k-1}} \right) \\
& = & \frac{a_1}{p} + \sum_{k=2}^{n}\frac{q}{p^k}
\end{eqnarray*}
剩下的就是对一个等比数列进行求和了。

另外一种方法比较巧妙,假定存在一个实数$\lambda$,使得$a_{n+1}+\lambda=p(a_n+\lambda)$,展开与原递推式比较即得$\lambda=\frac{q}{p-1}$,于是数列$a_n-\lambda$就成为一个等比数列了。

现在看二阶的情形,每一项需要它前面两项才能确定:$a_{n+2}=pa_{n+1}+qa_n$,假想有两个实数$r$和$s$能够使得
\begin{equation}
  \label{eq:two-level-linear-recurrence-sequence-1}
a_{n+2}-ra_{n+1}=s(a_{n+1}-ra_n)
\end{equation}
展开后与原递推式比较可得
\begin{align*}
  r+s  =  p \\
  rs  =  -q
\end{align*}
因此$r$和$s$是方程$x^2=px+q$的两个根,在复数范围内,它必有两个解(可以相等,重根按重数计算),于是数列$a_{n+1}-ra_{n}$成为等比数列,求出它的通项$a_{n+1}-ra_n=f(n)$后,只要两端同时除以$r^{n+1}$即可求出$a_n$的通项,其结果如下

如果方程有两个等根$x=r$,那么
\[ a_n = \left( \frac{a_2-ra_1}{r^2} - \frac{a_2-2ra_1}{r^2} \right)r^n \]
如果方程有两个不相等的根$x_1=r,x_2=s$,那么
\[ a_n = \frac{a_2-sa_1}{r(r-s)} r^n - \frac{a_2-ra_1}{s(r-s)} s^n \]

如果这两个根不相等,则还有另一种求法,因为$r$和$s$都是方程$x^2=px+q$的根,因此既然有\ref{eq:two-level-linear-recurrence-sequence-1}成立,也就必然有
\begin{equation}
  \label{eq:two-level-linear-recurrence-sequence-2}
a_{n+2}-sa_{n+1}=r(a_{n+1}-sa_n)
\end{equation}
成立,递推下去,便有
\begin{align*}
  a_{n}-ra_{n-1}  =  s^{n-2}(a_2-ra_1) \\
  a_{n}-sa_{n-1}  =  r^{n-2}(a_2-sa_1) 
\end{align*}
从中解出$a_n$来:
\[ a_n=\frac{(a_2-sa_1)r^{n-1}-(a_2-ra_1)s^{n-1}}{r-s} \]
易见这公式具有如下形式
\[ a_n=c_1r^n+c_2s^n \]
其中$c_1$和$c_2$是常数,这常数跟$a_1$、$a_2$以及$r$和$s$有关,跟$r$和$s$有关实际上就是跟$p$与$q$有关,可以直接将$a_1$和$a_2$的值代入上式中定出$c_1$和$c_2$,这只需求解一个二元一次方程组就可以了。

而对于有两个等根的情形,通项具有形式
\[ a_n = (c_1+c_2n)r^n \]
的情形,同样可以利用待定系数法求出$c_1,c_2$.

更一般的情形是
\begin{theorem}
对于线性递推数列$\sum_{k=1}^n\lambda_k a_{p+k}=c$,称方程$\sum_{k=1}^n\lambda _k x^k = c$为它的特征方程,在复数范围内这个特征方程必有$n$个根(重根按重数计算),假定这些根是 $x_i(i=1,2,\cdots,m, m \leq n)$,相应根的重数是$r_i(i=1,2,\cdots,m, \sum_{i=1}^mr_i=n)$,则它的通项是:
\[ a_n=\sum_{i=1}^mP_{r_i-1}(n)x_i^n \]
上式中$P_{r_i-1}(n)$表示一个关于$n$的次数是$r_i-1$的多项式,如果哪个根是单重根,则它的系数是常数。
\end{theorem}

此定理便是说,线性递推数列,其实就是多项式作系数的指数的组合,以后在微分方程中还会看到,常系数线性微分方程的解也是指数函数的组合,这两个结果表明,差分与微分存在某种类似,差分可以视为微分的离散化。

\begin{example}
作为一个例子,现在来求斐波那契数列的通项,递推公式为$a_{n+2}=a_{n+1}+a_n$,特征方程是$x^2=x+1$,其两个根是$x_{1,2}=\frac{1}{2}(1 \pm \sqrt{5})$,于是通项应为:
\[ a_n=\alpha _1 x_1^n+ \alpha _2 x_2^n \]
将$a_1=1$和$a_2=1$带入求出两个系数,最后得:
\[ a_n= \frac{1}{\sqrt{5}}\left[ \left( \frac{1+\sqrt{5}}{2} \right)^n - \left( \frac{1-\sqrt{5}}{2} \right)^n \right] \]
令人惊讶的是一个所有项都是正整数的数列,其通项居然出现了无理数,事实上,利用二项定理可以证明,这个表达式将永远是正整数。
\end{example}

\begin{example}
  讨论数列$a_1=1,a_2=1$,而以后的项由$a_{n+2}=a_{n+1}-a_n$决定,其特征方程是$x^2=x-1$,方程在复数范围内有两个根
  \[ \alpha = \frac{1+\sqrt{3}i}{2}, \  \beta = \frac{-1+\sqrt{3}i}{2} \]
  通项则有形式$a_n=c_1 \alpha^n + c_2 \beta^n$的形式,将$a_1$和$a_2$的值代入通项定出$c_1$和$c_2$后得
  \[ a_n = \frac{1}{\sqrt{3}i} \left( \frac{1+\sqrt{3}i}{2} \right)^{n} - \frac{1}{\sqrt{3}i} \left( \frac{-1+\sqrt{3}i}{2} \right)^n \]
 这时的通项,就不得不借助复数来表达了,实际上这还是一个周期数列。
\end{example}

\begin{example}
  我们已经知道,一个$m-2$阶等差数列的递推公式是
\[ C_m^0a_n-C_m^1a_{n-1}+\cdots+(-1)^mC_m^ma_{n-m} = 0 \]
这显然是一个线性递推数列,其特征方程是
\[ \sum_{i=0}^m(-1)^iC_m^ix^i = 0 \]
也就是$(1-x)^m = 0$,这方程只有一个$m$重根$x=1$,因而通项可表为$a_n=p(n) \cdot 1^n$,其中$p(n)$是一个$m-1$次多项式,这样就再次证明了\autoref{theorem:common-formular-for-high-level-common-difference-sequence}.
\end{example}



\subsection{分式型递推数列的通项}

%%% Local Variables:
%%% mode: latex
%%% TeX-master: "../../book"
%%% End:
