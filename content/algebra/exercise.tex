
\section{题集}
\label{sec:number-sequence-exercise}

\subsection{数列}
\label{sec:exercise-number-series}



\begin{exercise}
  已知各项都是正实数的数列$x_n$对一切正整数$n$都成立$x_n+\frac{1}{x_{n+1}}<2$,求证该数列所有项都满足$x_n<1$.
\end{exercise}
\begin{proof}[解答]
  如果用上极限理论,则可以很容易的得出它单调增加并以1为极限,结论不证自明,所以这里主要讨论的是初等证明。

因为
$$
x_n+\frac{1}{x_{n+1}}<2 \leqslant 
x_{n+1}+\frac{1}{x_{n+1}}
$$
所以$x_n<x_{n+1}$,即该数列单调增加。

又显然$x_n<2$,所以
$$
2>x_n+\frac{1}{x_{n+1}}>x_n+\frac{1}{2}
$$
于是$x_n<2-\frac{1}{2}$,我们得到一个更加好的上限,重复这个过程,我们由$x_n<y_m$就可以得到
$$
x_n<2-\frac{1}{y_m}
$$
所以我们作数列$y_m$,它由$y_1=2$和
$$
y_{m+1}=2-\frac{1}{y_m}
$$
来确定。

数列$y_m$的每一项都大过数列$x_n$的全部项,所以它的下标特意用$m$而不是$n$来表示,以示不相关。

现在来求$y_m$的通项公式,由于
$$
\frac{1}{y_{m+1}-1}=1+\frac{1}{y_{m}-1}
$$
因此数列$\frac{1}{y_m-1}$是等差数列,它的通项为$y_m=1+\frac{1}{m}$\footnote{这是分式型递推数列求通项的不动点解法。},于是$x_n<y_m$对一切正整数$n$和$m$都成立,所以必定有$x_n\leqslant 1$(用反证法),而$x_n$的单调性则保证了等号是不能取的。

下面关于$y_m$再给个不求通项的玩法\footnote{其实这是高数的玩法,就差提到确界二字了。},$y_m>1$这一点根据数学归纳法是明显成立的。下面证明它可以任意接近1,也就是要证明,对于无论多么小的正实数$\delta$,总存在$y_m$中的某一项$y_M$,使得$y_M<1+\delta$。
采用反证法,假定存在某个正实数$\delta$,使得$y_m$中的所有项都满足$y_m\geqslant 1+\delta$,则
$$
y_{m+1}-1=\frac{1}{y_m}(y_m-1)\leqslant 
\frac{1}{1+\delta}(y_m-1)
$$
于是
$$
y_m-1\leqslant \frac{1}{(1+\delta)^m}
$$
显然与假设矛盾,故得证。
\end{proof}

\begin{exercise}
  数列$a_n$满足:$a_1=\frac{1}{2}$,$a_{n+1}=\frac{na_n+a_n^2}{n+1}$,
  \begin{enumerate}
  \item 求证该数列是递减的。
  \item 求证 $a_n < \frac{7}{4n}$
  \end{enumerate}
\end{exercise}
\begin{proof}[解答]
  根据数列归纳法易知 $0<a_n<1$,所以
  \begin{equation*}
    a_{n+1}=\frac{na_n+a_n^2}{n+1} < \frac{na_n+a_n}{n+1} = a_n
  \end{equation*}
  所以数列递减,第二问,只要证明$n>3$时有如下更强的不等式即可(使用数学归纳法,过程略去)
  \begin{equation*}
    a_n \leqslant \frac{7}{4n}-\frac{4}{n^2}
  \end{equation*}
\end{proof}

\begin{exercise}
  记 $I_n=1-\frac{1}{2}+\frac{1}{3}-\frac{1}{4}+\cdots+\frac{1}{2n-1}-\frac{1}{2n}$,求证,在正整数$n\geqslant 100$时,有$0.68<I_{n}<0.7$.
\end{exercise}
\begin{proof}[证明]
  记$J_n=\sum_{k=1}^n(\frac{1}{2k}-\frac{1}{2k+1})$,由于$z_n=\frac{1}{n-1}-\frac{1}{n}=\frac{1}{n(n-1)}$是递减的,并且相邻两项也相差越来越小,所以有不等式$z_{2k}<\frac{1}{2}(z_{2k-1}+z_{2k+1})$,也就是如下的:
\begin{equation*}
  \frac{1}{2k-1} - \frac{1}{2k} < \frac{1}{2} \left[ \left( \frac{1}{2k-2} - \frac{1}{2k-1} \right) + \left( \frac{1}{2k} - \frac{1}{2k+1} \right) \right]
\end{equation*}
对上式左边进行累加,但从$k=3$到$k=n$使用右边放缩,得
\begin{equation*}
  I_n<\frac{1}{2}+\frac{1}{12}+\frac{1}{2} \left[ \left( J_n-\frac{1}{6}-\frac{1}{2n(2n+1} \right) + \left( J_n-\frac{1}{6}-\frac{1}{20} \right) \right]
\end{equation*}
化简
\begin{equation*}
 I_n<J_n+\frac{47}{120} - \frac{1}{4n(2n+1)}
\end{equation*}
利用$I_n+J_n=1-\frac{1}{2n+1}$从上式中换掉$J_n$得
\begin{equation}
  \label{eq:sign-sum-reciprocal-positive-integer-max}
  I_n<\frac{167}{240}-\frac{1}{2(2n+1)}-\frac{1}{8n(2n+1)}<\frac{167}{240}<\frac{168}{240}=0.7
\end{equation}
于是不等式的右边得证,接下来考虑左边不等式,同样因为$z_n$是递减的,有不等式$z_{2k}>\frac{1}{2}(z_{2k}+z_{2k+1})$,也就是
\begin{equation*}
  \frac{1}{2k-1} - \frac{1}{2k} > \frac{1}{2} \left[ \left( \frac{1}{2k-1} - \frac{1}{2k} \right) + \left( \frac{1}{2k} - \frac{1}{2k+1} \right) \right]
\end{equation*}
对左边进行累加,在$k \geqslant 3$ 时使用右边放缩,得到
\begin{equation*}
  I_n > \frac{1}{2} + \frac{1}{12} + \frac{1}{2} \left( \frac{1}{5} - \frac{1}{2n+1} \right)
\end{equation*}
也就是
\begin{equation}
  \label{eq:sign-sum-reciprocal-positive-integer-min}
 I_n > \frac{41}{60} -\frac{1}{2(2n+1)} 
\end{equation}
在$n \geqslant 100$时,有
\begin{equation*}
  I_{n} \geqslant I_{100} > \frac{41}{60} - \frac{1}{402} = 0.680845771... > 0.68
\end{equation*}
所以不等式左边得证.

其实证明左边所用的放缩是比较松的,实际上因为$\lim_{n\to\infty}I_n=\ln{2}=0.693147...$,所以左边不等式的放缩余地较大,所以这样的放缩也能达到要求,现在来尝试使用更强的放缩,看看能得到一个什么样的结果。

对于$z_{n}$,不等式$z_n>2z_{n+1}-z_{n+2}$将是一个更强的放缩,所以我们有$z_{2k}>2z_{2k+1}-z_{2k+2}$,也就是下面的不等式
\begin{equation*}
  \frac{1}{2k-1}-\frac{1}{2k} > 2 \left( \frac{1}{2k}-\frac{1}{2k+1} \right) - \left( \frac{1}{2k+1}-\frac{1}{2k+2} \right)
\end{equation*}
对上式左边进行累加,但只在$k\geqslant 3$时使用右边放缩,即得
\begin{equation*}
  I_n > \frac{1}{2}+\frac{1}{12} + 2 \left( J_n-\frac{1}{6}-\frac{1}{20} \right) - \left( I_n-\frac{1}{2}-\frac{1}{12}-\frac{1}{30}+\frac{1}{(2n+1)(2n+2)} \right)
\end{equation*}
将其中的$J_{n}$用$1-I_n-\frac{1}{2n+1}$替换掉,即得
\begin{equation*}
  I_n>\frac{83}{120}-\frac{1}{2(2n+1)}-\frac{1}{4(2n+1)(2n+2)}
\end{equation*}
因此在$n \geqslant 100$时,便有
\begin{equation*}
  I_n \geqslant I_{100} > \frac{83}{120}-\frac{1}{2(2\times 100+1)}-\frac{1}{4(2\times 100+1)(2 \times 100 + 2)} = 0.6891729471.....
\end{equation*}
这个值已经非常接近$0.69$了。
\end{proof}

\begin{exercise}
  已知数列$\{a_n\}$满足: $a_1=1$,$a_{n+1}=a_n+\frac{1}{a_n^2}$,求证$a_{2015}>18$.
\end{exercise}

\begin{proof}[证明]
  易见这是一个递增的正项数列,在递推式两边同时三次方:
  \begin{equation*}
    a_{n+1}^3=\left( a_n+\frac{1}{a_n^2} \right)^3 = a_n^3+3+\frac{3}{a_n^3}+\frac{1}{a_n^6}>a_n^3+3
  \end{equation*}
  所以$a_{2015}>a_1^3+3\times 2014=6043 > 5832 = 18^3$.

  遗留问题,如果要证明的是 $18.2<a_{2015}<18.3$呢(编程计算知这是成立的)?
\end{proof}

\begin{exercise}
  数列$a_n$满足$a_1=2$,$(n+1)a_{n+1}^2=na_n^2+a_n$,求证
  \[ \sum_{i=2}^n \frac{a_i^2}{i^2}<\frac{9}{5} \]
\end{exercise}

\begin{proof}[证明一]
  由数列归纳法易证$a_n>1$,所以
\[ a_{n+1}^2=\frac{na_n^2+a_n}{n+1} < \frac{na_n^2+a_n^2}{n+1} = a_n^2 \]
于是数列递减,所以当$n>1$时,$a_n < a_1=2$
\[ (n+1)a_{n+1}^2 = na_n^2+a_n < na_n^2+2\]
于是累加下去,就有
\[ na_n^2 < a_1^2+2(n-1)=2(n+1) \]
所以
\[ a_n^2 < 2 \left( 1+\frac{1}{n} \right) \]
于是
\[ \sum_{i=2}^n \frac{a_i^2}{i^2} <2 \left( \sum_{i=2}^n \frac{1}{i^2}+\sum_{i=2}^n \frac{1}{i^3} \right) \]
借用放缩
\[ \frac{1}{i^2}<\frac{1}{(i-1)i}=\frac{1}{i-1}+\frac{1}{i} \]
和
\[ \frac{1}{i^3}<\frac{1}{(i-1)i(i+1)} = \frac{1}{2} \left( \frac{1}{(i-1)i}-\frac{1}{i(i+1)} \right) \]
从$i \geqslant 4$开始放缩,累加即得
\begin{align*}
\sum_{i=2}^n \frac{a_i^2}{i^2} & < 2 \left( \frac{1}{4}+\frac{1}{9}+(\frac{1}{3}-\frac{1}{n})+\frac{1}{8}+\frac{1}{27}+\frac{1}{2}(\frac{1}{12}-\frac{n}{n+1}) \right) \\
& < 2 \left( \frac{1}{4}+\frac{1}{9}+\frac{1}{3}+\frac{1}{8}+\frac{1}{27}+\frac{1}{24} \right) \\
& = 2 \left( \frac{3}{4}+\frac{4}{27} \right)<\frac{9}{5}
\end{align*}
\end{proof}

\begin{proof}[证明二]
 不以要证的不等式为目标,研究下这个数列的性态,因为
\[ a_{n+1}^2=a_n \frac{na_n+1}{n+1} \]
显然$\frac{na_n+1}{n+1}$是$a_n$和1的加权平均,因为$a_n>1$有$\frac{na_n+1}{n+1}<a_n$,所以有
\begin{align*}
a_{n+1} &=\sqrt{a_n\cdot \frac{na_n+1}{n+1}} \\
& <\frac{1}{2} \left( a_n+\frac{na_n+1}{n+1} \right)  \\
& = \frac{2n+1}{2n+2}a_n+\frac{1}{2n+2}
\end{align*}
另一方面,由$\frac{na_n+1}{n+1}<a_n$,所以
\[ a_{n+1}^2=a_n \cdot  \frac{na_n+1}{n+1} > \left( \frac{na_n+1}{n+1} \right)^2 \]
所以
\[ a_{n+1}>\frac{n}{n+1}a_n+\frac{1}{n+1} \]
综合这两个估计,得到
\[ \frac{n}{n+1}a_n+\frac{1}{n+1} < a_{n+1} < \frac{2n+1}{2n+2}a_n+\frac{1}{2n+2} \]
左右都是$a_n$和1的加权平均,只是权重不同,上式改写为
\[ \frac{n}{n+1}(a_n-1) < a_{n+1}-1 < \frac{2n+1}{2n+2} (a_n-1) \]
所以最后就有估计式
\[ 1+\frac{1}{n} < a_n < 1 + \frac{1}{2} \cdot \frac{(2n-1)!!}{(2n)!!} \]
对于后面的双阶乘,由熟知的放缩
\begin{align*}
& \left( \frac{1}{2} \cdot \frac{3}{4} \cdots \frac{2n-1}{2n} \right)^2 \\
={} & \left( \frac{1}{2} \cdot \frac{1}{2} \right) \left(\frac{3}{4} \cdot \frac{3}{4} \right) \cdots \left( \frac{2n-1}{2n} \cdot \frac{2n-1}{2n} \right) \\
<{} & \left( \frac{1}{2} \cdot \frac{2}{3} \right) \left( \frac{3}{4} \cdot \frac{4}{5} \right) \cdots \left( \frac{2n-1}{2n} \cdot \frac{2n}{2n+1} \right) \\
={} & \frac{1}{2n+1}
\end{align*}
所以$a_n$的估计式两端都以1为极限,由夹逼定理,$a_n$极限为1.

而仍由那估计式,可以得出
\[ a_n^2< \left( 1+\frac{1}{2\sqrt{2n+1}} \right)^2 <2+\frac{1}{4} \frac{1}{2n+1} < 2+\frac{1}{8n} \]
由这不等式,仍同证明一中的放缩,同样可证得题目中的不等式。 
\end{proof}

\begin{exercise}\footnote{题目来自悠闲数学娱乐论坛.}
  数列$a_n$满足,$a_1=1$,$na_na_{n+1}=1$,求证: 
  \begin{enumerate}
  \item
    \[ \frac{a_{n+2}}{n} = \frac{a_n}{n+1} \]
  \item
    \[ 2(\sqrt{n+1}-1) \leqslant \frac{1}{2a_3}+\frac{1}{3a_4}+\cdots+\frac{1}{(n+1)a_{n+2}} \leqslant n \]
  \end{enumerate}
\end{exercise}

\begin{proof}[证明]
  首先在$a_{n+1}a_n=\frac{1}{n}$中将$n$替换为$n+1$得$a_{n+2}a_{n}=\frac{1}{n+1}$,这两式相除即得
\[ \frac{a_{n+2}}{n} = \frac{a_n}{n+1} \]
第一问得证。
第二问,通项
\[ \frac{1}{(k+1)a_{k+2}} = \frac{1}{ka_k} = a_{k+1} \]
所以只是要证明下式
\[ 2(\sqrt{n+1}-1) \leqslant a_2+a_3+\cdots+a_{n+1} \leqslant n \]
这只要能证明下面这个估计就能办到($n>1$)
\[ 2(\sqrt{n}-\sqrt{n-1}) \leqslant a_n \leqslant 1 \]
用数学归纳法试了一下,递推存在一点困难,倒是证明更强的放缩容易些($n>1$)
\[ \frac{1}{\sqrt{n-1}} \leqslant a_n \leqslant 1 \]
以下就来数学归纳法吧,计算得$a_2=1$,$a_3=\frac{1}{2}$,都符合这不等式,于是假定$a_n$满足这不等式,来看$a_{n+2}$的情况:
\[ a_{n+2} = \frac{n}{n+1}a_n < a_n \leqslant 1 \]
同时
\[ a_{n+2} = \frac{n}{n+1}a_n \geqslant \frac{n}{n+1} \cdot \frac{1}{\sqrt{n-1}} > \frac{1}{\sqrt{n+1}} \]
于是得证。
说明:这里数归是从$n$到$n+2$而不是$n+1$是因为,试算了前面几项,发现它是一个锯齿数列,也就是偶数项都向1靠近,奇数项都向0靠近,所以就分别考虑两个子列了。
\end{proof}

\begin{exercise}\footnote{来自kuing的悠闲数学娱乐论坛。}
  数列$a_n$满足: $a_1=1/3$,递推公式为$a_{n+1}=a_n+\frac{a_n^2}{n^2}$,求证:
  \[ \frac{n}{2n+1} \leqslant a_n \leqslant \frac{2n-1}{2n+1} \]
\end{exercise}

\begin{proof}[证明]
  这个递推公式比较有意思,在证出这题目后还可以进一步研究$a_1$的取值对数列的收敛性的影响,因为$a_n=n$是符号这递推式的,而题目的结论表明当$a_1$取$1/3$时数列有上界,显然数列又是递增的,所以是收敛的。

  左边不等式是很松的,因为数列是递增的,而
  \[ a_4=\frac{30760}{59049}=0.520923\cdots>\frac{1}{2}>\frac{n}{2n+1} \]
  右端尚未有思路,但是根据程序显示的结果,数列在$a_4$就已经超过0.57,但是直到$a_{100000}$都还没超过0.61,所以此数列增长极其缓慢,从要证的结论来看它有上界,因而有极限,猜测这极限应该小于1,若能证明这一点,则结论便能得到证明。
\end{proof}

\begin{exercise} 
  两个数列$a_n$和$b_n$满足: $a_1=0$, $b_1=\frac{1}{2}$,并有递推关系
  \[ a_{n+1}=\frac{a_n+b_n}{2}, \  b_{n+1}=\sqrt{a_{n+1}b_n} \]
  求证这两个数列有共同的极限,并求出这个极限。
\end{exercise}

题目来源: 悠闲数学娱乐论坛,原帖链接: \url{http://kuing.orzweb.net/viewthread.php?tid=4569&extra=page%3D1},这题目命题人(网友"其妙")给出了它的几何背景,是周长为2的正$2^n$边形的内切圆半径和外接圆半径.

\begin{proof}[证明一]
  (解答于 2017-05-02)
  因为
\[ \frac{a_{n+1}}{b_{n+1}}=\frac{\frac{a_n+b_n}{2}}{\sqrt{a_{n+1}b_n}} = \frac{\frac{a_n+b_n}{2}}{\sqrt{\frac{a_n+b_n}{2}b_n}} = \sqrt{\frac{1+\frac{a_n}{b_n}}{2}} \]
所以令$c_n=a_n / b_n$,就有
\[ c_{n+1}=\sqrt{\frac{1+c_n}{2}} \]
受半角公式
\[ \cos{\frac{\theta}{2}}=\sqrt{\frac{1+\cos{\theta}}{2}} \]
启发,令$c_n=\cos{\theta_n}$,则可取$\theta_{n+1}=\theta_n / 2$,结合$\cos{\theta_1}=c_1=a_1 / b_1=0$可取$\theta_1=\pi / 2$,于是$\theta_n=\pi / 2^n$,所以得到
\[ a_n=\cos\frac{\pi}{2^n}b_n \]
再回到$b_n$的递推式,有
\[ b_{n+1}^2=a_{n+1}b_n=\cos{\frac{\pi}{2^{n+1}}} b_{n+1}b_n\]
所以
\[ b_{n+1}=\cos{\frac{\pi}{2^{n+1}}} b_n \]
于是便不难求得
\[ b_n=b_1 \cos{\frac{\pi}{2^2}}\cos{\frac{\pi}{2^3}}\cdots \cos{\frac{\pi}{2^n}} \]
对这个余弦的连乘积,将它乘以$\sin{\frac{\pi}{2^n}}$再利用正弦的二倍角公式便会发生连锁反应,反应的结果便是这连乘积等于$\left( 2^{n-1}\sin{\frac{\pi}{2^n}} \right)^{-1}$,所以
\[ b_n=\frac{1}{2^n \sin{\frac{\pi}{2^n}}} \]
而
\[ a_n=\cos{\frac{\pi}{2^n}}b_n=\frac{1}{2^n}\cot{\frac{\pi}{2^n}} \]
由熟知的极限$\lim_{x \to 0} \frac{\sin{x}}{x}=1$便知$a_n$和$b_n$有共同的极限$\frac{1}{\pi}$.

补充说明:在令$c_n=\cos{\theta_n}$时尚需证明$|c_n| \leqslant 1$,这利用$c_n$的递推式和初始值$c_1=0$,由数学归纳法便知$0 \leqslant c_n \leqslant 1$,如果初始$c_1>1$,便不能使用余弦了,但却可以使用双曲函数$\cosh{x}=(e^x+e^{-x})/2$,双曲函数有着相同的倍半公式。
\end{proof}

\begin{proof}[证明二]
  (解答于 2012-05-18)
  首先在$a_n$的递推式两边取极限即知它俩如果收敛就必定收敛到相同的极限,将$b_n=2a_{n+1}-a_n$代入$b_n$的递推式中换掉$b_n$和$b_{n+1}$得到
  \[ (2a_{n+2}-a_{n+1})^{2} = a_{n+1}(2a_{n+1}-a_n) \]
  展开整理得
  \[ a_{n+2}^2-a_{n+2}a_{n+1} = \frac{1}{4}(a_{n+1}^2-a_{n+1}a_n) \]
  逐次把下标推下去,就有
  \[ a_{n+1}^2-a_{n+1}a_n = \frac{1}{4^{n+1}} \]
  把它视为关于$a_{n+1}$的二次方程,解之得
  \[ a_{n+1} = \frac{1}{2} \left( a_n+\sqrt{a_n^2+\frac{1}{4^n}} \right) \]
令$a_n=1/(2^n \tan{\theta_n})$,代入上式得$\tan{\theta_{n+1}}=\tan{\frac{\theta_n}{2}}$,所以可以取$\theta_{n+1}=\theta_n/2$,进而得$\theta_n=\pi / 2^n$,所以$n>1$时有
\[ a_n = \frac{1}{2^n \tan{\frac{\pi}{2^n}}} \]
比较$a_n$的递推式和定义式可得
\[ b_n = \sqrt{a_n^2+\frac{1}{4^n}} = \frac{1}{2^n \sin{\frac{\pi}{2^n}}} \]
由熟知的极限$\lim_{x \rightarrow 0} \frac{\sin{x}}{x}=1$即知$a_n$和$b_n$有共同的极限$1/\pi$.
\end{proof}

\subsection{不等式}
\label{sec:exercise-inequality}


\begin{exercise}
  已知三个正实数$a,b,c$满足
  \begin{equation*}
    \frac{1}{1+a} + \frac{1}{1+b} + \frac{1}{1+c} = 1
  \end{equation*}
  求证
  \begin{equation*}
    (a-1)(b-1)(c-1) \leqslant 1
  \end{equation*}
\end{exercise}

\begin{proof}[证明]
  作代换
  \begin{equation*}
    \frac{1}{1+a} = \frac{x}{x+y+z}, \  \frac{1}{1+b} = \frac{y}{x+y+z}, \  \frac{1}{1+c} = \frac{z}{x+y+z}
  \end{equation*}
  其中$x,y,z$是三个正实数,于是只要证明
  \begin{equation*}
    (y+z-x)(z+x-y)(x+y-z) \leqslant xyz
  \end{equation*}
  如果左边为负,不等式显然成立,否则可再作代换
  \begin{equation*}
    r = y+z-x, \  s = z+x-y, \  t = x+y-z
  \end{equation*}
  于是只要证明
  \begin{equation*}
    rst \leqslant \frac{s+t}{2} \frac{t+r}{2} \frac{r+s}{2}
  \end{equation*}
  由均值,这显然。
\end{proof}

\begin{exercise}\footnote{2008年IMO试题.}
  设$a$,$b$,$c$是三个互不相等的实数,求证
  \[ \left( \frac{a}{a-b} \right)^2 + \left( \frac{b}{b-c} \right)^2 + \left( \frac{c}{c-a} \right)^2 \geqslant 1 \]
\end{exercise}

\begin{proof}[证明]
  作代换
  \[ x=\frac{a}{a-b}, \  y=\frac{b}{b-c} \  z=\frac{c}{c-a} \]
  则只需证$x^2+y^2+z^2 \geqslant 1$,而在这代换下,易知
  \[ \left( 1-\frac{1}{x} \right) \left( 1-\frac{1}{y} \right) \left( 1-\frac{1}{z} \right) = 1 \]
  整理即为
  \[ x+y+z = (xy+yz+zx)+1 \]
  记等式左右两边的公共值为$t$,则
    \[ x^2+y^2+z^2 = (x+y+z)^2-2(xy+yz+zx) = t^2-2(t-1)=(t-1)^2+1 \geqslant 1 \]
  得证。
\end{proof}

\begin{exercise}\footnote{2017年德国数学奥林匹克试题.}
  已知非负实数$x$、$y$、$z$满足$x+y+z=1$,求证:
  \[ 1 \leqslant \frac{x}{1-yz}+\frac{y}{1-zx}+\frac{z}{1-xy} \leqslant \frac{9}{8} \]
\end{exercise}

\begin{proof}[证明]
  左边有最直接的证明,因为每个分母都小于等于1,所以有
  \[ \frac{x}{1-yz}+\frac{y}{1-zx}+\frac{z}{1-xy} \geqslant x+y+z = 1 \]
  此外我们还有以下的换元法证明:
  显见$0 \leqslant x,y,z \leqslant 1$,作代换
  \[ a=\frac{x}{1-yz}, \  b=\frac{y}{1-zx}, \  c=\frac{z}{1-xy} \]
  只需证明$1\leqslant a+b+c \leqslant 9/8$,在这代换下有$a,b,c \geqslant 0$,并且
  \begin{align*}
    a = {} & x+ayz \\
    b = {} & y+bzx \\
    c = {} & z+cxy
  \end{align*}
  三式相加并利用$x+y+z=1$得
  \[ a+b+c=1+ayz+bzx+cxy \]
  显然右边大于等于1,所以$a+b+c\geqslant 1$,原不等式左边得证。右边待证。
\end{proof}

\begin{exercise}
  (Sqing55) 已知实数$a,b,c>0$,且$a+b+c=1$,求证: 当$\lambda \geqslant \frac{1}{5}$时有下面不等式成立:
  \[ \frac{a+\lambda}{a+bc} + \frac{b+\lambda}{b+ca} + \frac{c+\lambda}{c+ab} \geqslant \frac{9}{4}(1+3\lambda) \]
  \begin{proof}[证明]
    (在 Kuing 的提示下完成),先分离出$\lambda$
    \[ f(\lambda) = \left( \frac{1}{a+bc}+\frac{1}{b+ca}+\frac{1}{c+ab}-\frac{27}{4} \right) \lambda + \frac{a}{a+bc}+\frac{b}{b+ca}+\frac{c}{c+ab} \geqslant \frac{9}{4} \]
    我们先证明$\lambda$的系数是非负的,这样就有$f(\lambda)\geqslant f\left( \frac{1}{5} \right)$,从而只要证明$f\left( \frac{1}{5} \right) \geqslant \frac{9}{4}$就可以了。

    因为$a+b+c=1$,所以
    \[ \frac{1}{a+bc} \geqslant \frac{1}{a+\left( \frac{b+c}{2} \right)^2} =
    \frac{1}{a+\left( \frac{1-a}{2} \right)^2} = \frac{4}{(a+1)^2} \]
  由切线法,易证
  \[ \frac{4}{(a+1)^2} \geqslant \frac{9}{4}-\frac{27}{8}\left( a-\frac{1}{3} \right) \]
  于是
  \[ \sum \frac{1}{a+bc} \geqslant \frac{27}{4} \]
  于是$f(\lambda)$是单调不减的,所以接下来证明$f\left( \frac{1}{5} \right) \geqslant \frac{9}{4}$,以下过程中$\lambda=1/5$,因为
  \[ \sum \frac{a+\lambda}{a+bc} = \sum \frac{a+\lambda}{(a+b)(a+c)} \]
  所以只需证明
  \[ \sum (a+\lambda)(b+c) \geqslant \frac{9}{4}(a+b)(b+c)(c+a) \]
  为齐次化,左边作处理
  \[ (a+\lambda)(b+c) = (a+\lambda(a+b+c))(b+c)(a+b+c) \]
  于是只要证明
  \[ (a+b+c)\sum (b+c)((1+\lambda)a+b+c) \geqslant \frac{9}{4}(a+b)(b+c)(c+a) \]
  暴力展开,整理后,即是要证
  \[ a^3+b^3+c^3 +3abc \geqslant a^2(b+c) + b^2(c+a)+c^2(a+b) \]
  这等价于熟知的
  \[ abc \geqslant (b+c-a)(c+a-b)(a+b-c) \]
  \end{proof}
\end{exercise}

\begin{exercise}
  设正实数$a$、$b$、$c$满足$a+b+c=1$,求证
  \[ \frac{1+a}{1-a} + \frac{1+b}{1-b} + \frac{1+c}{1-c} \leqslant 2 \left( \frac{a}{b} + \frac{b}{c} + \frac{c}{a} \right) \]
\end{exercise}

%%% Local Variables:
%%% mode: latex
%%% TeX-master: "../../book"
%%% End:

