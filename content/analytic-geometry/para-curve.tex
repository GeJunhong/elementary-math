
\section{抛物线}
\label{sec:para-curve}

\begin{property}
  如\autoref{fig:parabola-focus-chord-three-line-intersection}所示,过抛物线对称轴上焦点外侧任一点$D$,分别向过焦点$F$的弦$MN$的两个端点引直线,分别与准线相交于点$P$和$Q$,准线与对称轴的交点是$R$,那么: (1)直线$DR$、$PN$、$QM$三线共点(设为点$T$),(2)对称轴上的四点$R$、$T$、$F$、$D$构成调和点列。
\end{property}

\begin{figure}[htbp]
  \centering
\includegraphics{content/analytic-geometry/pic/parabola-focus-chord-three-line-intersection.pdf}
\caption{抛物线焦点弦的一个性质}
\label{fig:parabola-focus-chord-three-line-intersection}
\end{figure}

\begin{proof}[证明]
  (1) 由塞瓦定理,只需证
  \begin{equation}
    \label{eq:parabola-focus-chord-three-line-intersection-ceva}
  \frac{DN}{NQ} \cdot \frac{QR}{RP} \cdot \frac{PM}{MD} = 1 
  \end{equation}
  设点$N$在准线上的投影是$N_0$,则有$NF=NN_0$,再设点$N$在抛物线对称轴上的投影是$N_1$,则
  \[ \frac{DN}{NQ} = \frac{NN_1}{QN_0} = \frac{NN_1}{NF} \cdot \frac{NN_0}{QN_0} = \sin{\angle NFD} \cdot \tan{\angle DQP} \]
  同理可得
  \[ \frac{DM}{MP} = \sin{\angle MFR} \cdot \tan{\angle DPQ} \]
  而显然$\angle NFD = \angle MFR$,而且
  \[ \frac{QR}{RP} = \frac{\frac{DR}{RP}}{\frac{DR}{QR}} = \frac{\tan{\angle DPQ}}{\tan{\angle DQP}} \]
  因此式\ref{eq:parabola-focus-chord-three-line-intersection-ceva}成立,结论得证。

  (2) 对 $\triangle QRD$ 和截线$PTN$ 使用梅涅劳斯定理得
  \[ \frac{DN}{NQ} \cdot \frac{QP}{PR} \cdot \frac{RT}{TD} = 1 \]
  再对 $\triangle PRD$ 和截线 $QTM$ 使用梅涅劳斯定理得
  \[ \frac{DM}{MP} \cdot \frac{PQ}{QR} \cdot \frac{RT}{TD} = 1 \]
  由此二式便得
  \[ \frac{RT}{TD} \left( \frac{DN}{NQ} + \frac{DM}{MP} \right) = 1 \]
  记$\angle NFD=\theta$,$\angle DQP=\alpha$,$\angle DPQ=\beta$,则由(1)的证明过程中所得结果可得
  \[ \frac{DN}{NQ} + \frac{DM}{MP} = \sin{\theta} (\tan{\alpha} + \tan{\beta}) \]
  记$DF=d$,有熟知的焦半径
  \[ NF=\frac{p}{1-\cos{\theta}}, \  MF = \frac{p}{1+\cos{\theta}} \]
  由几何关系得
  \[ \tan{\alpha} = \tan{\angle DNN_1} = \frac{DN_1}{NN_1} = \frac{DF-NF\cos{\theta}}{NF \sin{\theta}} = \frac{d(1-\cos{\theta})-p\cos{\theta}}{p\sin{\theta}} \]
  同理可得
  \[ \tan{\beta} = \frac{d(1+\cos{\theta})+p\cos{\theta}}{p\sin{\theta}} \]
  由此二式便得
  \[ \sin{\theta}(\tan{\alpha} + \tan{\beta}) = \frac{2d}{p} \]
  即
  \[ \frac{DN}{NQ} + \frac{DM}{MP} = \frac{2d}{p} \]
  所以
  \[ \frac{RT}{TD} = \frac{p}{2d} \]
  由此便知,当抛物线和点$D$的位置确定后,点$T$的位置便固定下来,与焦点弦的具体位置无关,由上式不难求得
  \[ RT = \frac{p(p+d)}{p+2d}, \  TF=\frac{dp}{p+2d} \]
  于是
  \[ \frac{RT}{TF} = \frac{p+d}{d} = \frac{RD}{DF} \]
  所以这四点为调和点列.
\end{proof}

%%% Local Variables:
%%% TeX-master: "../../book"
%%% End:
