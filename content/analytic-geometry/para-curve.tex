
\section{抛物线}
\label{sec:para-curve}

\begin{property}
  如\autoref{fig:parabola-focus-chord-three-line-intersection}所示,过抛物线对称轴上焦点外侧任一点$D$,分别向过焦点$F$的弦$MN$的两个端点引直线,分别与准线相交于点$P$和$Q$,准线与对称轴的交点是$R$,那么有直线$DR$、$PN$、$QM$三线共点。
\end{property}

\begin{figure}[htbp]
  \centering
\includegraphics{content/analytic-geometry/pic/parabola-focus-chord-three-line-intersection.pdf}
\caption{抛物线焦点弦的一个性质}
\label{fig:parabola-focus-chord-three-line-intersection}
\end{figure}

\begin{proof}[证明]
  由塞瓦定理,只需证
  \begin{equation}
    \label{eq:parabola-focus-chord-three-line-intersection-ceva}
  \frac{DN}{NQ} \cdot \frac{QR}{RP} \cdot \frac{PM}{MD} = 1 
  \end{equation}
  设点$N$在准线上的投影是$N_0$,则有$NF=NN_0$,再设点$N$在抛物线对称轴上的投影是$N_1$,则
  \[ \frac{DN}{NQ} = \frac{NN_1}{QN_0} = \frac{NN_1}{NF} \cdot \frac{NN_0}{QN_0} = \sin{\angle NFD} \cdot \tan{\angle DQP} \]
  同理可得
  \[ \frac{DM}{MP} = \sin{\angle MFR} \cdot \tan{\angle DPQ} \]
  而显然$\angle NFD = \angle MFR$,而且
  \[ \frac{QR}{RP} = \frac{\frac{DR}{RP}}{\frac{DR}{QR}} = \frac{\tan{\angle DPQ}}{\tan{\angle DQP}} \]
  因此式\ref{eq:parabola-focus-chord-three-line-intersection-ceva}成立,结论得证。
\end{proof}

%%% Local Variables:
%%% TeX-master: "../../book"
%%% End:
