
\section{题集}
\label{sec:number-sequence-exercise}
\begin{exercise}
  已知各项都是正实数的数列$x_n$对一切正整数$n$都成立$x_n+\frac{1}{x_{n+1}}<2$,求证该数列所有项都满足$x_n<1$.
\end{exercise}
\begin{proof}[解答]
  如果用上极限理论,则可以很容易的得出它单调增加并以1为极限,结论不证自明,所以这里主要讨论的是初等证明。

因为
$$
x_n+\frac{1}{x_{n+1}}<2 \leqslant 
x_{n+1}+\frac{1}{x_{n+1}}
$$
所以$x_n<x_{n+1}$,即该数列单调增加。

又显然$x_n<2$,所以
$$
2>x_n+\frac{1}{x_{n+1}}>x_n+\frac{1}{2}
$$
于是$x_n<2-\frac{1}{2}$,我们得到一个更加好的上限,重复这个过程,我们由$x_n<y_m$就可以得到
$$
x_n<2-\frac{1}{y_m}
$$
所以我们作数列$y_m$,它由$y_1=2$和
$$
y_{m+1}=2-\frac{1}{y_m}
$$
来确定。

数列$y_m$的每一项都大过数列$x_n$的全部项,所以它的下标特意用$m$而不是$n$来表示,以示不相关。

现在来求$y_m$的通项公式,由于
$$
\frac{1}{y_{m+1}-1}=1+\frac{1}{y_{m}-1}
$$
因此数列$\frac{1}{y_m-1}$是等差数列,它的通项为$y_m=1+\frac{1}{m}$\footnote{这是分式型递推数列求通项的不动点解法。},于是$x_n<y_m$对一切正整数$n$和$m$都成立,所以必定有$x_n\leqslant 1$(用反证法),而$x_n$的单调性则保证了等号是不能取的。

下面关于$y_m$再给个不求通项的玩法\footnote{其实这是高数的玩法,就差提到确界二字了。},$y_m>1$这一点根据数学归纳法是明显成立的。下面证明它可以任意接近1,也就是要证明,对于无论多么小的正实数$\delta$,总存在$y_m$中的某一项$y_M$,使得$y_M<1+\delta$。
采用反证法,假定存在某个正实数$\delta$,使得$y_m$中的所有项都满足$y_m\geqslant 1+\delta$,则
$$
y_{m+1}-1=\frac{1}{y_m}(y_m-1)\leqslant 
\frac{1}{1+\delta}(y_m-1)
$$
于是
$$
y_m-1\leqslant \frac{1}{(1+\delta)^m}
$$
显然与假设矛盾,故得证。
\end{proof}

\begin{exercise}
  数列$a_n$满足:$a_1=\frac{1}{2}$,$a_{n+1}=\frac{na_n+a_n^2}{n+1}$,
  \begin{enumerate}
  \item 求证该数列是递减的。
  \item 求证 $a_n < \frac{7}{4n}$
  \end{enumerate}
\end{exercise}
\begin{proof}[解答]
  根据数列归纳法易知 $0<a_n<1$,所以
  \begin{equation*}
    a_{n+1}=\frac{na_n+a_n^2}{n+1} < \frac{na_n+a_n}{n+1} = a_n
  \end{equation*}
  所以数列递减,第二问,只要证明$n>3$时有如下更强的不等式即可(使用数学归纳法,过程略去)
  \begin{equation*}
    a_n \leqslant \frac{7}{4n}-\frac{4}{n^2}
  \end{equation*}
\end{proof}

\begin{exercise}
  记 $I_n=1-\frac{1}{2}+\frac{1}{3}-\frac{1}{4}+\cdots+\frac{1}{2n-1}-\frac{1}{2n}$,求证,在正整数$n\geqslant 100$时,有$0.68<I_{n}<0.7$.
\end{exercise}
\begin{proof}[证明]
  记$J_n=\sum_{k=1}^n(\frac{1}{2k}-\frac{1}{2k+1})$,由于$z_n=\frac{1}{n-1}-\frac{1}{n}=\frac{1}{n(n-1)}$是递减的,并且相邻两项也相差越来越小,所以有不等式$z_{2k}<\frac{1}{2}(z_{2k-1}+z_{2k+1})$,也就是如下的:
\begin{equation*}
  \frac{1}{2k-1} - \frac{1}{2k} < \frac{1}{2} \left[ \left( \frac{1}{2k-2} - \frac{1}{2k-1} \right) + \left( \frac{1}{2k} - \frac{1}{2k+1} \right) \right]
\end{equation*}
对上式左边进行累加,但从$k=3$到$k=n$使用右边放缩,得
\begin{equation*}
  I_n<\frac{1}{2}+\frac{1}{12}+\frac{1}{2} \left[ \left( J_n-\frac{1}{6}-\frac{1}{2n(2n+1} \right) + \left( J_n-\frac{1}{6}-\frac{1}{20} \right) \right]
\end{equation*}
化简
\begin{equation*}
 I_n<J_n+\frac{47}{120} - \frac{1}{4n(2n+1)}
\end{equation*}
利用$I_n+J_n=1-\frac{1}{2n+1}$从上式中换掉$J_n$得
\begin{equation}
  \label{eq:sign-sum-reciprocal-positive-integer-max}
  I_n<\frac{167}{240}-\frac{1}{2(2n+1)}-\frac{1}{8n(2n+1)}<\frac{167}{240}<\frac{168}{240}=0.7
\end{equation}
于是不等式的右边得证,接下来考虑左边不等式,同样因为$z_n$是递减的,有不等式$z_{2k}>\frac{1}{2}(z_{2k}+z_{2k+1})$,也就是
\begin{equation*}
  \frac{1}{2k-1} - \frac{1}{2k} > \frac{1}{2} \left[ \left( \frac{1}{2k-1} - \frac{1}{2k} \right) + \left( \frac{1}{2k} - \frac{1}{2k+1} \right) \right]
\end{equation*}
对左边进行累加,在$k \geqslant 3$ 时使用右边放缩,得到
\begin{equation*}
  I_n > \frac{1}{2} + \frac{1}{12} + \frac{1}{2} \left( \frac{1}{5} - \frac{1}{2n+1} \right)
\end{equation*}
也就是
\begin{equation}
  \label{eq:sign-sum-reciprocal-positive-integer-min}
 I_n > \frac{41}{60} -\frac{1}{2(2n+1)} 
\end{equation}
在$n \geqslant 100$时,有
\begin{equation*}
  I_{n} \geqslant I_{100} > \frac{41}{60} - \frac{1}{402} = 0.680845771... > 0.68
\end{equation*}
所以不等式左边得证.

其实证明左边所用的放缩是比较松的,实际上因为$\lim_{n\to\infty}I_n=\ln{2}=0.693147...$,所以左边不等式的放缩余地较大,所以这样的放缩也能达到要求,现在来尝试使用更强的放缩,看看能得到一个什么样的结果。

对于$z_{n}$,不等式$z_n>2z_{n+1}-z_{n+2}$将是一个更强的放缩,所以我们有$z_{2k}>2z_{2k+1}-z_{2k+2}$,也就是下面的不等式
\begin{equation*}
  \frac{1}{2k-1}-\frac{1}{2k} > 2 \left( \frac{1}{2k}-\frac{1}{2k+1} \right) - \left( \frac{1}{2k+1}-\frac{1}{2k+2} \right)
\end{equation*}
对上式左边进行累加,但只在$k\geqslant 3$时使用右边放缩,即得
\begin{equation*}
  I_n > \frac{1}{2}+\frac{1}{12} + 2 \left( J_n-\frac{1}{6}-\frac{1}{20} \right) - \left( I_n-\frac{1}{2}-\frac{1}{12}-\frac{1}{30}+\frac{1}{(2n+1)(2n+2)} \right)
\end{equation*}
将其中的$J_{n}$用$1-I_n-\frac{1}{2n+1}$替换掉,即得
\begin{equation*}
  I_n>\frac{83}{120}-\frac{1}{2(2n+1)}-\frac{1}{4(2n+1)(2n+2)}
\end{equation*}
因此在$n \geqslant 100$时,便有
\begin{equation*}
  I_n \geqslant I_{100} > \frac{83}{120}-\frac{1}{2(2\times 100+1)}-\frac{1}{4(2\times 100+1)(2 \times 100 + 2)} = 0.6891729471.....
\end{equation*}
这个值已经非常接近$0.69$了。
\end{proof}

\begin{exercise}
  已知数列$\{a_n\}$满足: $a_1=1$,$a_{n+1}=a_n+\frac{1}{a_n^2}$,求证$a_{2015}>18$.
\end{exercise}

\begin{proof}[证明]
  易见这是一个递增的正项数列,在递推式两边同时三次方:
  \begin{equation*}
    a_{n+1}^3=\left( a_n+\frac{1}{a_n^2} \right)^3 = a_n^3+3+\frac{3}{a_n^3}+\frac{1}{a_n^6}>a_n^3+3
  \end{equation*}
  所以$a_{2015}>a_1^3+3\times 2014=6043 > 5832 = 18^3$.

  遗留问题,如果要证明的是 $18.2<a_{2015}<18.3$呢(编程计算知这是成立的)?
\end{proof}

\begin{exercise}
  数列$a_n$满足$a_1=2$,$(n+1)a_{n+1}^2=na_n^2+a_n$,求证
  \[ \sum_{i=2}^n \frac{a_i^2}{i^2}<\frac{9}{5} \]
\end{exercise}

\begin{proof}[证明一]
  由数列归纳法易证$a_n>1$,所以
\[ a_{n+1}^2=\frac{na_n^2+a_n}{n+1} < \frac{na_n^2+a_n^2}{n+1} = a_n^2 \]
于是数列递减,所以当$n>1$时,$a_n < a_1=2$
\[ (n+1)a_{n+1}^2 = na_n^2+a_n < na_n^2+2\]
于是累加下去,就有
\[ na_n^2 < a_1^2+2(n-1)=2(n+1) \]
所以
\[ a_n^2 < 2(1+\frac{1}{n}) \]
于是
\[ \sum_{i=2}^n \frac{a_i^2}{i^2} <2 \left( \sum_{i=2}^n \frac{1}{i^2}+\sum_{i=2}^n \frac{1}{i^3} \right) \]
借用放缩
\[ \frac{1}{i^2}<\frac{1}{(i-1)i}=\frac{1}{i-1}+\frac{1}{i} \]
和
\[ \frac{1}{i^3}<\frac{1}{(i-1)i(i+1)} = \frac{1}{2} \left( \frac{1}{(i-1)i}-\frac{1}{i(i+1)} \right) \]
从$i \geqslant 4$开始放缩,累加即得
\begin{align*}
\sum_{i=2}^n \frac{a_i^2}{i^2} & < 2 \left( \frac{1}{4}+\frac{1}{9}+(\frac{1}{3}-\frac{1}{n})+\frac{1}{8}+\frac{1}{27}+\frac{1}{2}(\frac{1}{12}-\frac{n}{n+1}) \right) \\
& < 2 \left( \frac{1}{4}+\frac{1}{9}+\frac{1}{3}+\frac{1}{8}+\frac{1}{27}+\frac{1}{24} \right) \\
& = 2 \left( \frac{3}{4}+\frac{4}{27} \right)<\frac{9}{5}
\end{align*}
\end{proof}

\begin{proof}[证明二]
 不以要证的不等式为目标,研究下这个数列的性态,因为
\[ a_{n+1}^2=a_n \frac{na_n+1}{n+1} \]
显然$\frac{na_n+1}{n+1}$是$a_n$和1的加权平均,因为$a_n>1$有$\frac{na_n+1}{n+1}<a_n$,所以有
\begin{align*}
a_{n+1} &=\sqrt{a_n\cdot \frac{na_n+1}{n+1}} \\
& <\frac{1}{2} \left( a_n+\frac{na_n+1}{n+1} \right)  \\
& = \frac{2n+1}{2n+2}a_n+\frac{1}{2n+2}
\end{align*}
另一方面,由$\frac{na_n+1}{n+1}<a_n$,所以
\[ a_{n+1}^2=a_n \cdot  \frac{na_n+1}{n+1} > \left( \frac{na_n+1}{n+1} \right)^2 \]
所以
\[ a_{n+1}>\frac{n}{n+1}a_n+\frac{1}{n+1} \]
综合这两个估计,得到
\[ \frac{n}{n+1}a_n+\frac{1}{n+1} < a_{n+1} < \frac{2n+1}{2n+2}a_n+\frac{1}{2n+2} \]
左右都是$a_n$和1的加权平均,只是权重不同,上式改写为
\[ \frac{n}{n+1}(a_n-1) < a_{n+1}-1 < \frac{2n+1}{2n+2} (a_n-1) \]
所以最后就有估计式
\[ 1+\frac{1}{n} < a_n < 1 + \frac{1}{2} \cdot \frac{(2n-1)!!}{(2n)!!} \]
对于后面的双阶乘,由熟知的放缩
\begin{align*}
& \left( \frac{1}{2} \cdot \frac{3}{4} \cdots \frac{2n-1}{2n} \right)^2 \\
={} & \left( \frac{1}{2} \cdot \frac{1}{2} \right) \left(\frac{3}{4} \cdot \frac{3}{4} \right) \cdots \left( \frac{2n-1}{2n} \cdot \frac{2n-1}{2n} \right) \\
<{} & \left( \frac{1}{2} \cdot \frac{2}{3} \right) \left( \frac{3}{4} \cdot \frac{4}{5} \right) \cdots \left( \frac{2n-1}{2n} \cdot \frac{2n}{2n+1} \right) \\
={} & \frac{1}{2n+1}
\end{align*}
所以$a_n$的估计式两端都以1为极限,由夹逼定理,$a_n$极限为1.

而仍由那估计式,可以得出
\[ a_n^2< \left( 1+\frac{1}{2\sqrt{2n+1}} \right)^2 <2+\frac{1}{4} \frac{1}{2n+1} < 2+\frac{1}{8n} \]
由这不等式,仍同证明一中的放缩,同样可证得题目中的不等式。 
\end{proof}


%%% Local Variables:
%%% mode: latex
%%% TeX-master: "../../book"
%%% End:
