
\section{数列概要}
\label{sec:number-sequence-general}

\subsection{数列概念}

数列就是一个数的序列,可以是有限序列也可以是无限序列,按下标记法可以写成$a_1,a_2,\ldots$,有时为了方便也可以让下标从零开始$a_0,a_1,a_2,\ldots$,甚至作为无穷序列时,还可以把下标扩展到负整数:$\cdots,a_{-2},a_{-1},a_0,a_1,a_2,\cdots$,成为一个双向无穷数列,如果有一个公式可以把数列的每一项表成它的下标的函数,就称此公式为该数列的通项公式。数列本质上是定义在整数集或者正整数集上的函数。

\begin{example}
  任一实数在十进制下可以表为
  \[ a_na_{n-1}\cdots a_1a_0.a_{-1}a_{-2}\cdots \]
  其中$a_i$取值范围是$0,1,\ldots,9$,这便可以视为一个数列,而且是单侧无穷数列,而这个实数的值是
  \[ a_n\cdot 10^n + a_{n-1} \cdot 10^{n-1} + \cdots + a_1 \cdot 10 + a_0 \cdot 10^0 + a_{-1} \cdot 10^{-1} + a_{-2} \cdot 10^{-2} +\cdots \]
\end{example}

\subsection{等差数列与等比数列}

等差数列和等比数列是最常见的两种数列。

\begin{definition}
 如果一个数列$\ldots,a_{-2},a_{-1},a_0,a_1,a_2,\ldots$中,任一项减去它减一项所得差值都相同,即存在常数$d$,对任意整数$n$都有$a_n-a_{n-1}=d$,则称此数列为 \emph{等差数列},而常数$d$称为此等差数列的\emph{公差}.
\end{definition}

显然,正整数数列$\ldots,-2,-1,0,1,2,\ldots$是公差为1的等差数列,任何项都取相同值的\emph{常数数列}则是公差为零的等差数列。

\begin{theorem}
对于等差数列中的任意两项(无论顺序)有
\[ a_n-a_m=(n-m)d \]
\end{theorem}

\begin{proof}[证明]
证明是容易的,只要证明$n\geqslant m$的情况即可(否则可以反过来相减),首先可以验证$n$与$m$相等时该等式成立,而后有
\[ a_n-a_m=\sum_{k=m}^{n-1}(a_{k+1}-a_k)=(n-m)d \]
当然,对差值$n-m$施行归纳法也是可行的。
\end{proof}

假定我们知道$a_0$和公差$d$,在上式中取$m=0$,我们就得到等差数列的通项公式
\begin{theorem}
  等差数列的每一项可以由$a_0$及公差$d$表为
\[ a_n=a_0+nd \]
这便是等差数列的通项公式.
\end{theorem}

当然通项并不一定非得使用$a_0$来表示,$a_n=a_1+(n-1)d$,$a_n=a_{100}+(n-100)d$也都是通项。

易见这通项是关于$n$的一次函数,实际上容易证明,如果数列的通项可表成关于下标的一次函数,则它必然是等差数列。

显然考虑对等差数列进行某种求和,如果从$a_0$开始,向正下标或者负下标进行累加,$n$是任意一个整数(无论正负),记$S_n=\sum_{i=0}^{n}a_n$,则
\[ S_n=\sum_{i=0}^na_i=\sum_{i=0}^n(a_0+nd)=(n+1)a_0+d\sum_{i=0}^ni\]
于是这个求和就归结为对自然数序列$0,1,2,\ldots,n$进行求和,高斯曾经口算出$1+2+3+\cdots+100=5050$,他的计算方法是,将1与100相加得101,2与99相加得101,如此这般,最后50与51相加得101,于是$1+2+\cdots+100=50 \times 101=5050$,把这个方法略加改造便得出如下的计算$1+2+\cdots+n$的方法,为了避免$n$的奇偶性带来分组分不完从而还需要分奇偶讨论的问题,再拿一个式子$n+(n-1)+\cdots+2+1$与$1+2+\cdots+(n-1)+n$进行相加,则这两个式子对应位置上的两数之和都是$n+1$,所以最终得公式
\[ 1+2+\cdots+n = \frac{1}{2}n(n+1) \]
不必怀疑右端会有出现小数的可能,因为作为连续的两个自然数$n$和$n+1$中必定一奇一偶,所以右端永不会为小数。

把这公式应用到上面的式子便得
\begin{theorem}
  在公差为$d$的等差数列$a_n$中,有
  \[ S_n=\sum_{i=0}^na_i=(n+1)a_0+\frac{1}{2}n(n+1)d \]
  其中$n$可以是任意整数(包括负整数)。
\end{theorem}

上面求$1+2+\cdots+n$的方法称为\emph{倒序相加法},当然也可以直接把这方法应用到$S_n=a_0+a_1+\cdots+a_n$上,同样可得出上面的结果.

\begin{definition}
  如果一数列$\ldots,a_{-2},a_{-1},a_0,a_1,a_2,\ldots$中,任意一项与前一项的比值都是同一非零常数,即存在非零常数$q$,使得对于任意整数$n$都有$a_n / a_{n-1} = q$成立,则称此数列是 \emph{等比数列},而比值常数$q$称为它的 \emph{公比}.
\end{definition}

对于等比数列,如果它的各项都是正的,取对数即可变身为一个等差数列。对于所有的等比数列,同样可以得出类似于等差数列的结论来,即是

\begin{theorem}
对于公比为$q$的等比数列$\cdots,a_{-2},a_{-1},a_0,a_1,a_2,\cdots$,对任意整数$n$和任意整数$m$都成立
\[ \frac{a_n}{a_m}=q^{n-m} \]
\end{theorem}

并由此得出通项
\begin{theorem}
  公比为$q$的等比数列$\ldots,a_{-2},a_{-1},a_0,a_1,a_2,\ldots$的通项可表为
 \[ a_n=a_0q^n \]
\end{theorem}

关于等比数列的求和
\[ S_n = a_1+a_2+\cdots+a_n = a_1(1+q+\cdots+q^{n}) \]
这就归结为对$1+q+q^2+\cdots+q^{n}$求值,

由数学归纳法可得出下面的乘法公式,它也是平方差公式和立方差公式的自然推广
\begin{equation}
  \label{eq:a-power-n-substract-b-power-n}
 a^n-b^n = (a-b)(a^{n-1}+a^{n-2}b+\cdots+ab^{n-2}+b^{n-1}) 
\end{equation}
式中$n$是任意正整数,

在式中取$a=1,b=q$便可得出,对于任意正整数$n$,成立
\[ 1+q+\cdots+q^{n} = \frac{1-q^{n+1}}{1-q} \]
若$n$为负整数,则上式中以$1/q$代$q$,便知上式仍然成立,于是对于等比数列就有
\begin{theorem}
  在以$q$为公比的等比数列$\ldots,a_{-2},a_{-1},a_0,a_1,a_2,\ldots$中,对于任意整数$n$都有
  \[ S_n = \sum_{i=0}^na_i = \frac{a_0(1-q^{n+1})}{1-q} \]
\end{theorem}

还可以以另外一种方式得出这公式,因为
\[ S_n = a_0+a_1+\cdots+a_n \]
两端同乘以公比$q$,便可得出
\[ qS_n = q(a_0+a_1+\cdots+a_n) = a_1+a_2+\cdots+a_n+a_{n+1} \]
于是
\[ (1-q)S_n = a_0-a_{n+1} = a_0 \]
所以得出
\[ S_n = \frac{a_0-a_{n+1}}{1-q} = \frac{a_0(1-q^{n+1})}{1-q} \]
当然这过程中要求$q \neq 1$,而对于$q=1$的情况,显然公式也是对的。

这个方法称为 \emph{错位相减法},下面这个例子提示了这个方法的一些用途。

\begin{example}[错位相减法的一个用途]
  考虑对数列$a_n = n q^n(n=1,2,3,\ldots)$进行求和,这里要求$q \neq 1$,否则就变成早已解决过的问题了。同样有
  \[ S_n = a_1+a_2+\cdots+a_n = q+2q^2+3q^3 + \cdots + (n-1)q^{n-1} + nq^n \]
  两端同乘以$q$得
  \[ qS_n = q^2+2q^3+3q^4 + \cdots + (n-1)q^n + nq^{n+1} \]
  把以上两式相减,得
  \[ (1-q)S_n = (q + q^2 + \cdots + q^n) - n q^{n+1} \]
  显然上式右端括号中的部分是等比数列的求和,于是$S_n$便可以求出。

  接下来再考虑数列$a_n=n^mq^n(n=1,2,3,\ldots)$,其中$m$是个正整数,记
  \[ S_m(n) = \sum_{i=1}^na_i = \sum_{i=1}^n n^m q^n = q + 2^mq^2 + \cdots + (n-1)^mq^{n-1} + n^m q^m \]
  进行同样的过程有
  \[ qS_m(n) = q^2 + 2^mq^3 + \cdots + (n-1)^m q^n + n^m q^{n+1} \]
  两式相减
  \[ (1-q)S_m(n) = q +(2^m-1)q^2 + (3^m-2^m)q^3 + \cdots + (n^m-(n-1)^m)q^m - n^mq^{n+1}  \]
  而利用下面的公式(它可以由公式\autoref{eq:a-power-n-substract-b-power-n}得出,也可以由二项式定理得出)
  \[ (1+k)^n-k^n = 1+k+k^2 + \cdots + k^{n-1} \]
  上式可化为
  \begin{eqnarray*}
    (1-q)S_m(n) & = & \sum_{i=1}^n(i^m-(i-1)^m)q^i - n^mq^{n+1} \\
                & = & \sum_{i=0}^{n-1}((1+i)^m-i^m)q^{i+1} - n^mq^{n+1} \\
                & = & \sum_{i=0}^{n-1}\sum_{j=0}^{m-1}i^jq^{i+1} - n^mq^{n+1} \\
                & = & \sum_{j=0}^{m-1}\sum_{i=0}^{n-1}i^jq^{i+1} - n^mq^{n+1} \\
    & = & q\sum_{j=0}^{m-1}S_j(n-1) - n^mq^{n+1}
  \end{eqnarray*}
  这意味着$S_m(n)$可以用$S_0(n-1),S_1(n-1),\ldots,S_{m-1}(n-1)$来表示,即具有递推性,而$S_0(n-1)$显然就是等比数列求和,于是$S_m(n)$便可以求出来。

  进一步,便可以求形如$a_n=p(n)q^n$这样的数列的和,其中$p(n)$是关于$n$的多项式,只是这过程将越来越繁琐,如此机械化的计算方法,由计算机程序来进行是再合适不过了。
\end{example}

\begin{example}[自然数的幂和]
  在上面得出了公式
  \[ 1+2+\cdots+n = \frac{1}{2}n(n+1) \]
  现在就来讨论下一般的$S_m(n)=\sum_{i=1}^ni^m$的求和公式,仍然由
  \[ (1+k)^m-k^m = 1 + k + k^2 + \cdots + k^{m-1} \]
  对$k=1,2,\ldots,n$进行累加,便得
  \[ (1+n)^m-1 = S_0(n)+S_1(n)+S_2(n) + \cdots S_{m-1}(n) \]
  显然这便是联系着诸$S_m(n)$的递推关系,且有初始公式
  \[ S_0(n) = n \]
  由此出发便能求出任何$S_m(n)$的表达式来,比如说取$m=2$便得
  \[ (1+n)^2-1 = S_0(n)+S_1(n)+S_2(n) \]
  于是得出
  \[ S_2(n) = 1^2+2^2+\cdots+n^2 = \frac{1}{6}n(n+1)(2n+1) \]
  同样不必怀疑右端的整数性,因为$n$与$n+1$作为相邻两个正整数,必然有一个偶数,故$2 \mid n(n+1)$,还需证明$3 \mid n(n+1)(2n+1)$,若是$n$与$n+1$中有3的倍数,则自然不消说,若是$n$与$n+1$都不是3的倍数,则它俩被3除所得余数必然一个是1,另一个是2,于是$2n+1=n+(n+1)$便必然是3的倍数,于是2和3都能整除$n(n+1)(2n+1)$且2与3互素,所以6也能整除它。

  同样再取$m=3$,便可得出
  \[ S_3(n)= 1^3+2^3+\cdots+n^3 = \frac{1}{4}n^2(n+1)^2 \]
  一般的可以知道,$S_m(n)$是关于$n$的$m+1$次多项式。
\end{example}


\subsection{递推数列的通项}

\subsubsection{线性递推数列的通项}

\subsubsection{分式型递推数列的通项}

%%% Local Variables:
%%% mode: latex
%%% TeX-master: "../../book"
%%% End:
