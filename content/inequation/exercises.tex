
\section{题选}
\label{sec:inequality-exercises}

\begin{exercise}
  已知三个正实数$a,b,c$满足
  \begin{equation*}
    \frac{1}{1+a} + \frac{1}{1+b} + \frac{1}{1+c} = 1
  \end{equation*}
  求证
  \begin{equation*}
    (a-1)(b-1)(c-1) \leqslant 1
  \end{equation*}
\end{exercise}

\begin{proof}[证明]
  作代换
  \begin{equation*}
    \frac{1}{1+a} = \frac{x}{x+y+z}, \  \frac{1}{1+b} = \frac{y}{x+y+z}, \  \frac{1}{1+c} = \frac{z}{x+y+z}
  \end{equation*}
  其中$x,y,z$是三个正实数,于是只要证明
  \begin{equation*}
    (y+z-x)(z+x-y)(x+y-z) \leqslant xyz
  \end{equation*}
  如果左边为负,不等式显然成立,否则可再作代换
  \begin{equation*}
    r = y+z-x, \  s = z+x-y, \  t = x+y-z
  \end{equation*}
  于是只要证明
  \begin{equation*}
    rst \leqslant \frac{s+t}{2} \frac{t+r}{2} \frac{r+s}{2}
  \end{equation*}
  由均值,这显然。
\end{proof}

\begin{exercise}\footnote{2008年IMO试题.}
  设$a$,$b$,$c$是三个互不相等的实数,求证
  \[ \left( \frac{a}{a-b} \right)^2 + \left( \frac{b}{b-c} \right)^2 + \left( \frac{c}{c-a} \right)^2 \geqslant 1 \]
\end{exercise}

\begin{proof}[证明]
  作代换
  \[ x=\frac{a}{a-b}, \  y=\frac{b}{b-c} \  z=\frac{c}{c-a} \]
  则只需证$x^2+y^2+z^2 \geqslant 1$,而在这代换下,易知
  \[ \left( 1-\frac{1}{x} \right) \left( 1-\frac{1}{y} \right) \left( 1-\frac{1}{z} \right) = 1 \]
  整理即为
  \[ x+y+z = (xy+yz+zx)+1 \]
  记等式左右两边的公共值为$t$,则
    \[ x^2+y^2+z^2 = (x+y+z)^2-2(xy+yz+zx) = t^2-2(t-1)=(t-1)^2+1 \geqslant 1 \]
  得证。
\end{proof}

\begin{exercise}\footnote{2017年德国数学奥林匹克试题.}
  已知非负实数$x$、$y$、$z$满足$x+y+z=1$,求证:
  \[ 1 \leqslant \frac{x}{1-yz}+\frac{y}{1-zx}+\frac{z}{1-xy} \leqslant \frac{9}{8} \]
\end{exercise}

\begin{proof}[证明]
  左边有最直接的证明,因为每个分母都小于等于1,所以有
  \[ \frac{x}{1-yz}+\frac{y}{1-zx}+\frac{z}{1-xy} \geqslant x+y+z = 1 \]
  此外我们还有以下的换元法证明:
  显见$0 \leqslant x,y,z \leqslant 1$,作代换
  \[ a=\frac{x}{1-yz}, \  b=\frac{y}{1-zx}, \  c=\frac{z}{1-xy} \]
  只需证明$1\leqslant a+b+c \leqslant 9/8$,在这代换下有$a,b,c \geqslant 0$,并且
  \begin{align*}
    a = {} & x+ayz \\
    b = {} & y+bzx \\
    c = {} & z+cxy
  \end{align*}
  三式相加并利用$x+y+z=1$得
  \[ a+b+c=1+ayz+bzx+cxy \]
  显然右边大于等于1,所以$a+b+c\geqslant 1$,原不等式左边得证。右边待证。
\end{proof}

\begin{exercise}
  (Sqing) 已知实数$a,b,c>0$,且$a+b+c=1$,求证: 当$\lambda \geqslant \frac{1}{5}$时有下面不等式成立:
  \[ \frac{a+\lambda}{a+bc} + \frac{b+\lambda}{b+ca} + \frac{c+\lambda}{c+ab} \geqslant \frac{9}{4}(1+3\lambda) \]
  \begin{proof}[证明]
    (在 Kuing 的提示下完成),先分离出$\lambda$
    \[ f(\lambda) = \left( \frac{1}{a+bc}+\frac{1}{b+ca}+\frac{1}{c+ab}-\frac{27}{4} \right) \lambda + \frac{a}{a+bc}+\frac{b}{b+ca}+\frac{c}{c+ab} \geqslant \frac{9}{4} \]
    我们先证明$\lambda$的系数是非负的,这样就有$f(\lambda)\geqslant f\left( \frac{1}{5} \right)$,从而只要证明$f\left( \frac{1}{5} \right) \geqslant \frac{9}{4}$就可以了。

    因为$a+b+c=1$,所以
    \[ \frac{1}{a+bc} \geqslant \frac{1}{a+\left( \frac{b+c}{2} \right)^2} =
    \frac{1}{a+\left( \frac{1-a}{2} \right)^2} = \frac{4}{(a+1)^2} \]
  由切线法,易证
  \[ \frac{4}{(a+1)^2} \geqslant \frac{9}{4}-\frac{27}{8}\left( a-\frac{1}{3} \right) \]
  于是
  \[ \sum \frac{1}{a+bc} \geqslant \frac{27}{4} \]
  于是$f(\lambda)$是单调不减的,所以接下来证明$f\left( \frac{1}{5} \right) \geqslant \frac{9}{4}$,以下过程中$\lambda=1/5$,因为
  \[ \sum \frac{a+\lambda}{a+bc} = \sum \frac{a+\lambda}{(a+b)(a+c)} \]
  所以只需证明
  \[ \sum (a+\lambda)(b+c) \geqslant \frac{9}{4}(a+b)(b+c)(c+a) \]
  为齐次化,左边作处理
  \[ (a+\lambda)(b+c) = (a+\lambda(a+b+c))(b+c)(a+b+c) \]
  于是只要证明
  \[ (a+b+c)\sum (b+c)((1+\lambda)a+b+c) \geqslant \frac{9}{4}(a+b)(b+c)(c+a) \]
  暴力展开,整理后,即是要证
  \[ a^3+b^3+c^3 +3abc \geqslant a^2(b+c) + b^2(c+a)+c^2(a+b) \]
  这等价于熟知的
  \[ abc \geqslant (b+c-a)(c+a-b)(a+b-c) \]
  \end{proof}
\end{exercise}

%%% Local Variables:
%%% mode: latex
%%% TeX-master: "../../book"
%%% End:
