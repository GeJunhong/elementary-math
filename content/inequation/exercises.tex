
\section{题选}
\label{sec:inequality-exercises}

\begin{exercise}
  已知三个正实数$a,b,c$满足
  \begin{equation*}
    \frac{1}{1+a} + \frac{1}{1+b} + \frac{1}{1+c} = 1
  \end{equation*}
  求证
  \begin{equation*}
    (a-1)(b-1)(c-1) \leqslant 1
  \end{equation*}
\end{exercise}

\begin{proof}[证明]
  作代换
  \begin{equation*}
    \frac{1}{1+a} = \frac{x}{x+y+z}, \  \frac{1}{1+b} = \frac{y}{x+y+z}, \  \frac{1}{1+c} = \frac{z}{x+y+z}
  \end{equation*}
  其中$x,y,z$是三个正实数,于是只要证明
  \begin{equation*}
    (y+z-x)(z+x-y)(x+y-z) \leqslant xyz
  \end{equation*}
  如果左边为负,不等式显然成立,否则可再作代换
  \begin{equation*}
    r = y+z-x, \  s = z+x-y, \  t = x+y-z
  \end{equation*}
  于是只要证明
  \begin{equation*}
    rst \leqslant \frac{s+t}{2} \frac{t+r}{2} \frac{r+s}{2}
  \end{equation*}
  由均值,这显然。
\end{proof}

\begin{exercise}
\footnote{题目来源于重庆左文泽.}
  设$a$,$b$,$c$是三个互不相等的实数,求证
  \[ \left( \frac{a}{a-b} \right)^2 + \left( \frac{b}{b-c} \right)^2 + \left( \frac{c}{c-a} \right)^2 \geqslant 1 \]
\end{exercise}

\begin{proof}[证明]
  作代换
  \[ x=\frac{a}{a-b}, \  y=\frac{b}{b-c} \  z=\frac{c}{c-a} \]
  则只需证$x^2+y^2+z^2 \geqslant 1$,而在这代换下,易知
  \[ \left( 1-\frac{1}{x} \right) \left( 1-\frac{1}{y} \right) \left( 1-\frac{1}{z} \right) = 1 \]
  整理即为
  \[ x+y+z = (xy+yz+zx)+1 \]
  记等式左右两边的公共值为$t$,则
    \[ x^2+y^2+z^2 = (x+y+z)^2-2(xy+yz+zx) = t^2-2(t-1)=(t-1)^2+1 \geqslant 1 \]
  得证。
\end{proof}

%%% Local Variables:
%%% mode: latex
%%% TeX-master: "../../book"
%%% End:
