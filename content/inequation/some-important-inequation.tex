
\section{一些重要的不等式}
\label{sec:some-important-inequation}

本节叙述一些在初等数学领域非常重要的不等式,这些不等式应用广泛,刻画的不等式关系非常深刻。

\subsection{均值不等式}

\begin{theorem}[均值不等式]
  对于任意$n$个正实数$x_{i}(i=1,2,\ldots,n)$,记
  \begin{eqnarray}
    \label{eq:definition-of-average}
    H_n &=& \frac{n}{\sum_{i=1}^n \frac{1}{x_i}} \\
    G_n &=& (\prod_{i=1}^nx_{i})^{\frac{1}{n}} \\
    A_n &=& \frac{1}{n}\sum_{i=1}^{n}x_i \\
    Q_n &=& \sqrt{\frac{1}{n}\sum_{i=1}^nx_i^2}
  \end{eqnarray}
  称$H_{n}$为这$n$个正实数的 \emph{调和平均数},而$A_{n}$为 \emph{算术平均数},$G_n$为 \emph{几何平均数},$Q_{n}$为 \emph{平方平均数},则有下面不等式成立
  \begin{equation}
    \label{eq:mean-inequation-general}
    H_{n} \leqslant G_{n} \leqslant A_{n} \leqslant Q_{n}
  \end{equation}
  其中的等号只有在$x_{i}$全部都相等时才成立。
\end{theorem}
\begin{proof}[证明]
  先来证明$A_n\geqslant G_n$.

  我们在小节\ref{sec:mathematical-induction}中已经使用倒推形式的数学归纳法证明过这个不等式,所以在这一节我们使用第一数学归纳法。

  因为$\frac{x_1+x_2}{2}-\sqrt{x_1x_2}=\frac{1}{2}(\sqrt{x_1}-\sqrt{x_2})^2 \geqslant 0$,所以不等式$A_2 \geqslant G_2$成立。

  假定$A_n \geqslant G_n$,来尝试证明$A_{n+1} \geqslant G_{n+1}$,因为$A_{n+1}$是一个算术平均数,对一组数而言,如果新添加的数等于原来这些数的算术平均数,则算术平均数保持不变,于是有如下的策略
  \begin{eqnarray*}
    A_{n+1} &=& \frac{1}{n+1}\sum_{i=1}^{n+1}x_{i} \\
            &=& \frac{\sum_{i=1}^nx_i+[x_{n+1}+(n-1)A_{n+1}]}{2n} \\
            &=& \frac{1}{2} \left( \frac{x_1+\cdots+x_{n}}{n} + \frac{x_{n+1}+(n-1)A_{n+1}}{n} \right) \\
    & \geqslant & \frac{1}{2} \left( \sqrt[n]{x_1\cdots x_n} + \sqrt[n]{x_{n+1}A_{n+1}^{n-1}} \right) \\
            & \geqslant & \sqrt{\sqrt[n]{x_1\cdots x_n} \cdot \sqrt[n]{x_{n+1}A_{n+1}^{n-1}}} \\
    &=& \sqrt[2n]{x_1\cdots x_{n+1}A_{n+1}^{n-1}}
  \end{eqnarray*}
  也就是$A_{n+1}^{2n} \geqslant G_{n+1}^{n+1}A_{n+1}^{n-1}$,整理即得$A_{n+1} \geqslant G_{n+1}$,得证,等号成立的条件也是显而易见的。

  在$A_n \geqslant G_n$中,把每个数$x_i$都换成$\frac{1}{x_i}$,立即便得到$\frac{1}{H_n} \geqslant \frac{1}{G_n}$,所以$H_n \leqslant G_n$,至于$A_n \leqslant Q_n$,利用数学归纳法和基本不等式$2ab \leqslant a^2+b^2$就可以轻松获证,这里就不啰嗦了。

  下面再提供$A_n \geqslant G_n$的另一证法.

  这个不等式可以改写为如下形式
  \begin{equation*}
    \frac{x_1}{G_n} + \frac{x_2}{G_n} + \cdots + \frac{x_n}{G_n} \geqslant n
  \end{equation*}
  因为左边各项之积为1,所以我们只要能够证明:如果$n$个正实数$x_i(i=1,2,\ldots,n)$满足$x_1x_2\cdots x_n=1$,则$x_1+x_2+\cdots +x_n \geqslant n$ 就可以了。

  对于$n=2$的情况是很容易验证的,假定对于$n$个数也成立,现在来看$n+1$个数的情况,先把$x_nx_{n+1}$绑在一起视为一个数,这样$n+1$个数就成了$n$个数,根据归纳假设可得$x_1+\cdots + x_{n-1} + x_nx_{n+1} \geqslant n$,于是我们只要证明$x_n+x_{n+1} \geqslant x_nx_{n+1}+1$就可以了,也就是要证明$(x_n-1)(x_{n+1}-1)\leqslant 0$,这个似乎不一定成立,但是乘积为1的$n+1$个正实数,必定至少有一个大于等于1,同时至少有一个小于等于1,我们可以重新排序这些正实数,使得这两个数放在最后(或者干脆一开始绑定两个数的时候就绑这两个数),于是不等式$(x_n-1)(x_{n+1}-1)\leqslant 0$就成立了,所以得证。
\end{proof}

对于均值不等式中最基本的$G_n\leqslant A_n$,将每个数的权重一般化,得如下的加权形式
\begin{theorem}[加权平均值不等式]
  对于任意$n$个正实数$x_{i}(i=1,2,\ldots,n)$和一组权值$\alpha_i\geqslant 0(i=1,2,\ldots,n),\sum_{i=1}^n\alpha_i=1$,有如下的不等式
  \begin{equation}
    \label{eq:mean-inequation-with-weight}
    \sum_{i=1}^n\alpha_ix_i \geqslant \prod_{i=1}^nx_i^{\alpha_i}
  \end{equation}
  等号也只有在$x_i$全部都相等时才成立.
\end{theorem}
\begin{proof}[证明]
  先证$\alpha_i(i=1,2,\ldots,n)$都是有理数的情况,此时存在非负整数$p_i(i=1,2,\ldots,n)$使得$\alpha_i=\frac{p_i}{\sum_{k=1}^np_k}$,只要把$p_ix_i$看成$p_i$个$x_i$相加,就有
  \begin{eqnarray*}
    \sum_{i=1}^n\alpha_ix_i &=& \sum_{i=1}^n \frac{p_i}{\sum_{k=1}^np_k}x_i \\
                            &=& \frac{1}{\sum_{i=1}^np_i} \cdot \sum_{i=1}^np_ix_i \\
    & \geqslant & \prod_{i=1}^nx_i^{p_i}
  \end{eqnarray*}
  所以不等式得证。

  对于某个$\alpha_i$为无理数的情况,使用有理数序列去逼近它,再两边取极限即得证。
\end{proof}

\subsection{柯西(Cauchy)不等式}

\begin{theorem}[柯西不等式]
  对于任何两组实数$x_i(i=1,2,\ldots,n)$ 和$y_i(i=1,2,\ldots,n)$,有下面不等式成立
  \begin{equation}
    \label{eq:cauchy-inequation}
    \left( \sum_{i=1}^nx_iy_i \right)^{2} \leqslant \left( \sum_{i=1}^{n}x_i^2 \right) \left( \sum_{i=1}^ny_i^{2} \right)
  \end{equation}
  其中等号成立的唯一场景是:存在一个共同的实数$\lambda$使得$x_i=\lambda y_{i}(i=1,2,\ldots,n)$都成立。
\end{theorem}
等号成立的条件,通俗的说就是两组实数成比例,如果我们约定分数中的分母可以为零,并且此时分子也必须为零,则可以把取等条件改写为
\begin{equation*}
  \frac{x_1}{y_1} = \frac{x_2}{y_2} = \cdots = \frac{x_n}{y_n}
\end{equation*}
如果记向量$\vv{a}=(x_1,x_2,\ldots,x_n), \vv{b}=(y_1,y_2,\ldots,y_n)$,则柯西不等式即表明$|\vv{a} \cdot \vv{b}| \leqslant |\vv{a}| \cdot |\vv{b}|$.

把不等式的左边展开,再利用不等式$2xy \leqslant x^2+y^2$即可轻松证明柯西不等式,但更为流行的是下面的证明:

\begin{proof}[证明]
  构造二次函数
  \begin{equation*}
  f(t)=t^2\sum_{i=1}^nx_i^2+2t\sum_{i=1}^nx_iy_i+\sum_{i=1}^ny_i=\sum_{i=1}^n(x_it+y_i)^2 \geqslant 0
  \end{equation*}
 既然是恒为非负的二次函数,其判别式必小于等于零,于是即得柯西不等式,等号成立的条件就是二次函数有零点$t_0$,使得$x_it_0+y_i=0(i=1,2,\ldots,n)$,也就是两组数成比例。
\end{proof}

\subsection{排序不等式}

\begin{theorem}[排序不等式]
  对于任何两组按照相同的大小顺序排列好(均从小到大或者均从大到小)的两组实数$x_i(i=1,2,\ldots,n)$和$y_i(i=1,2,\ldots,n)$,有下面的不等式成立
  \begin{eqnarray}
    \label{eq:rearrangement inequality}
    & x_1y_n+x_2y_{n-1}+\cdots+x_ny_1 \nonumber \\
    \leqslant & x_1y_{r_1}+x_2y_{r_2}+\cdots+x_ny_{r_n} \nonumber \\
    \leqslant & x_1y_1+x_2y_2+\cdots+x_ny_n
  \end{eqnarray}
  其中$r_i(i=1,2,\ldots,n)$是$1,2,\ldots,n$的任意一个排列。
\end{theorem}
不等式的通俗说法就是,反序和最小,同序和最大,乱序和居中。

\subsection{切比雪夫(Chebyshev)不等式}

\begin{theorem}[切比雪夫不等式]
对于任意两组按照相同的大小顺序排列好的实数$x_i(i=1,2,\ldots,n)$和$y_i(i=1,2,\ldots,n)$,成立着下面不等式
\begin{equation}
  \label{eq:chebyshev-inequation}
  \frac{1}{n}\sum_{i=1}^nx_iy_i \leqslant \left( \frac{1}{n}\sum_{i=1}^nx_i \right) \left( \frac{1}{n}\sum_{i=1}^ny_i \right)
\end{equation}
其中的等号成立的唯一场景是:两组数中至少有一组数的值都相同。
\end{theorem}

\subsection{幂平均值不等式}

\begin{theorem}[幂平均值不等式]
  对于任意$n$个正实数$x_i(i=1,2,\ldots,n)$和两个正实数$0<p<q$,成立着不等式
  \begin{equation}
    \label{eq:pow-mean-inequation}
    \left( \frac{1}{n}\sum_{i=1}^nx_i^{p} \right)^{\frac{1}{p}}
    \leqslant \left( \frac{1}{n}\sum_{i=1}^nx_i^{q} \right)^{\frac{1}{q}}
  \end{equation}
  等号当且仅当$x_i(i=1,2,\ldots,n)$全部相等时成立。
\end{theorem}
取$p=1,q=2$,便可得算术平均数小于等于平方平均数。

将每个数的权重$\frac{1}{n}$一般化,得如下的加权形式
\begin{theorem}[加权幂平均值不等式]
  对任意$n$个正实数$x_i(i=1,2,\ldots,n)$和任意两个正实数$0<p<q$,以及一组权值$\alpha_i \geqslant 0(i=1,2,\ldots,n), \sum_{i=1}^n\alpha_i=1$,成立着下面不等式
  \begin{equation}
    \label{eq:pow-mean-inequation-with-weight}
    \left( \sum_{i=1}^n\alpha_ix_i^p \right)^{\frac{1}{p}}
    \leqslant \left( \sum_{i=1}^n\alpha_ix_i^q \right)^{\frac{1}{q}}
  \end{equation}
\end{theorem}

\subsection{赫尔德(Holder)不等式}

\begin{theorem}[赫尔德不等式]
  设$x_i> 0, y_i> 0 (i=1,2,\ldots,n)$且$\frac{1}{p}+\frac{1}{q}=1$,那么在$p>1$时成立下面的不等式
  \begin{equation}
    \label{eq:holder-inequation}
    \sum_{i=1}^nx_iy_i \leqslant \left( \sum_{i=1}^nx_i^p \right)^{\frac{1}{p}} \cdot \left( \sum_{i=1}^nx_i^q \right)^{\frac{1}{q}}
  \end{equation}
  在$p<1,p\neq 0$不等式反向,等号当且仅当向量$(x_1^p,\ldots,x_n^p)$和$(y_1^q,\ldots,y_n^q)$共线时成立。
\end{theorem}
在赫尔德不等式中,取$p=q=2$,即得柯西不等式。

\subsection{闵可夫斯基(Minkowski)不等式}


\begin{theorem}[闵可夫斯基不等式]
  设$x_i>0, y_i>0 (i=1,2,\ldots,n)$,那么在$p>1$时成立下面的不等式
  \begin{equation}
    \label{eq:holder-inequation}
    \left( \sum_{i=1}^n(x_i+y_i)^p \right)^{\frac{1}{p}} \leqslant \left( \sum_{i=1}^nx_i^p \right)^{\frac{1}{p}} + \left( \sum_{i=1}^ny_i^p \right)^{\frac{1}{p}}
  \end{equation}
  在$p<1$时不等式反向,等号当且仅当存在一个共同的实数$\lambda$使得$x_i=\lambda y_i(i=1,2,\ldots,n)$时成立。
\end{theorem}
在闵可夫斯基不等式中取$n=2,p=2$,便是三角形两边之和大于第三边的坐标表达。

\subsection{嵌入不等式}

\begin{theorem}[嵌入不等式]
  对$\triangle ABC$和任意的实数$x,y,z$均有下面不等式成立
  \begin{equation}
    \label{eq:embedding-inequation}
    x^2+y^2+z^2 \geqslant 2yz\cos{A}+2zx\cos{B}+2xy\cos{C}
  \end{equation}
  其中,当且仅当$x:y:z = \sin{A} : \sin{B} : \sin{C}$时等号成立。
\end{theorem}

%%% Local Variables:
%%% mode: latex
%%% TeX-master: "../../book"
%%% End:
