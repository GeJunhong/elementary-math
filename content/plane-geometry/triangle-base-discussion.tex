
\section{三角形基础}
\label{sec:triangle-base-discussion}

三角形是平面几何中最基本的图形之一,简单的图形却蕴含了丰富的内容,三角形的内容可以说是平面几何基础中最重要的内容。

\subsection{内外角平分线定理}
\label{sec:triangle-angle-bisector-theorem}

\begin{theorem}[三角形内外角平分线定理]
  三角形$ABC$中,角$A$的内角平分线交$BC$边于$E$,同时它的外角平分线交$BC$边延长线于$F$,那么有
  \begin{equation}
    \label{eq:triangle-angle-bisector-theorem}
    \frac{BE}{EC} = \frac{AB}{AC} = \frac{BF}{FC}
  \end{equation}
  反之,如果三角形$BC$边上和边的延长线上分别有两个点$E$和$F$能满足这比例式,那么$AE$和$AF$就分别是$\angle BAC$的内角平分线和外角平分线。
\end{theorem}

\begin{figure}[htbp]
\centering
\includegraphics{content/plane-geometry/pic/triangle-angle-bisector-theorem.pdf}
\caption{三角形内外角平分线定理}
\label{fig:triangle-angle-bisector-theorem}
\end{figure}

如果外角平分线与$BC$边平行,可以视为点$F$无穷远,此时右边的$BF : FC$按极限理解,等式也是成立的。

只证明定理本身,逆定理根据同一法是容易得到的。

\begin{proof}[证明一]
  过点$C$分别作内角平分线和外角平分线的平行线,分别与$AB$边及其延长线相交于点$P$和$Q$,那么易得$AC=AP$和$AC=AQ$,所以

\begin{figure}[htbp]
\centering
\includegraphics{content/plane-geometry/pic/triangle-angle-bisector-theorem-proof.pdf}
\caption{三角形内外角平分线定理的证明}
\label{fig:triangle-angle-bisector-theorem-proof}
\end{figure}

  \begin{equation*}
    \frac{BE}{EC} = \frac{BA}{AP} = \frac{AB}{AC}
  \end{equation*}
  和
  \begin{equation*}
    \frac{BF}{FC} = \frac{BA}{AQ} = \frac{AB}{AC}
  \end{equation*}
于是定理得证。
\end{proof}

\begin{proof}[证明二]
  辅助线的作法同证明一,对三角形$QBC$和截线$AE$应用梅涅劳斯定理可得$BE:EC=AB:AC$,对三角形$PBC$和截线$AF$应用梅涅劳斯定理可得$BF:FC=AB:AC$。
\end{proof}

\begin{example}[阿波罗尼奥斯圆]
  今来考虑平面上到两个定点的距离之比为常数的动点轨迹,记这两个定点为$A$和$B$,动点$P$在运动过程中始终满足$PA:PB=\lambda(\neq 1)$,这里限制常数不等于1是因为那是线段$AB$的垂直平分线。

\begin{figure}[htbp]
\centering
\includegraphics{content/plane-geometry/pic/apollonius-circle.pdf}
\caption{Apollonius圆}
\label{fig:apollonius-circle}
\end{figure}

 首先在直线$AB$上可以找到两个点$E$和$F$,使得$\vv{AE}=\lambda\vv{EB}$和$\vv{AF}=-\lambda\vv{FB}$,这里点$E$在线段$AB$上而点$F$在延长线上,动点$P$在运动过程中恒有$PA:PB=AE:EB$和$PA:PB=AF:FB$,当$P$不在直线$AB$上时,根据三角形的内外角平分线定理的逆定理,就有$PE$和$PF$分别是$\angle APB$的内角平分线和外角平分线,所以$PE \perp PF$,于是点$P$的轨迹就是以线段$EF$为直径的圆。这个结论最早是由古希腊几何学家阿波罗尼奥斯\footnote{参考文献\cite{conic-sections}的作者。}(Apollonius)发现的,所以称为阿波罗尼奥斯圆。

 值得一提的是,点$E$和$F$分线段$AB$的比为$\lambda$,则点$A$和$B$分线段$EF$的比也相同,这个比值是$|1-\lambda|/(1+\lambda)$,证明是很容易的,只要把线段$AE$、$EB$、$BF$、$AF$的长度都用$AB$的长度来表示就能得到这一点,这个结论是有用的,比如说任给一个圆和一个不等于1的正实数,我们可以反过来找出它是关于哪两个点的阿波罗尼斯圆,当然这两个点不是唯一的,在每一条直径上都能找出两对。
\end{example}

\subsection{正弦定理和余弦定理}
\label{sec:sine-theorem-and-cosine-theorem}

\subsection{外心}
\label{sec:triangle-excenter}

\begin{property}
  \label{ppty:triangle-circumradius-area-sides}
  三角形的外接圆半径$R$,面积$S$,和三边之间有如下的关系式
  \begin{equation}
    \label{eq:triangle-circumradius-area-sides}
    abc = 4RS
  \end{equation}
\end{property}

\begin{proof}[证明]
  由正弦定理和海伦公式,过程中$p=\frac{1}{2}(a+b+c)$为半周长
  \begin{eqnarray*}
    2R & = & \frac{a}{\sin{A}} \\
       & = & \frac{a}{\sqrt{1-\cos^2{A}}} \\
       & = & \frac{a}{\sqrt{1-\left( \frac{b^2+c^2-a^2}{2bc} \right)^2}} \\
       & = & \frac{2abc}{\sqrt{4b^2c^2-(b^2+c^2-a^2)^2}} \\
       & = & \frac{2abc}{\sqrt{((b+c)^2-a^2)(a^2-(b-c)^2)}} \\
       & = & \frac{2abc}{\sqrt{(a+b+c)(b+c-a)(a+b-c)(c+a-b)}} \\
       & = & \frac{abc}{2 \sqrt{p(p-a)(p-b)(p-c)}} \\
    & = & \frac{abc}{2S}
  \end{eqnarray*}
\end{proof}

这性质意味着,如果以三角形为底面,外接圆半径为高作三棱柱,其体积与以三边为棱长的长方体的体积之间,有着固定的比例关系,不知这能否有构造法说明这一点?

\subsection{内心}
\label{sec:triangle-incenter}

\begin{property}
  三角形的内切圆半径$r$,外接圆半径$R$,以及半周长$s$,与三边之间有如下关系
  \begin{equation}
    \label{eq:triangle-abc-equal-4Rrs}
    abc = 4Rrs
  \end{equation}
\end{property}

\begin{proof}[证明]
 由性质\ref{ppty:triangle-circumradius-area-sides},结合面积$S=\frac{1}{2}(a+b+c)r=sr$即得证。
\end{proof}

\subsection{重心}
\label{sec:triangle-centroid}

\subsection{垂心}
\label{sec:triangle-orthocentre}

\subsection{旁心}
\label{sec:escenter}



%%% Local Variables:
%%% TeX-master: "../../book"
%%% End:
