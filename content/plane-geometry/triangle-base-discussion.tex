
\section{三角形}
\label{sec:triangle-base-discussion}

三角形是平面几何中最基本的图形之一,简单的图形却蕴含了丰富的内容,三角形的内容可以说是平面几何基础中最重要的内容。

\subsection{内外角平分线定理}
\label{sec:triangle-angle-bisector-theorem}

\begin{theorem}[三角形内外角平分线定理]
  三角形$ABC$中,角$A$的内角平分线交$BC$边于$E$,同时它的外角平分线交$BC$边延长线于$F$,那么有
  \begin{equation}
    \label{eq:triangle-angle-bisector-theorem}
    \frac{BE}{EC} = \frac{AB}{AC} = \frac{BF}{FC}
  \end{equation}
  反之,如果三角形$BC$边上和边的延长线上分别有两个点$E$和$F$能满足这比例式,那么$AE$和$AF$就分别是$\angle BAC$的内角平分线和外角平分线。
\end{theorem}

\begin{figure}[htbp]
\centering
\includegraphics{content/plane-geometry/pic/triangle-angle-bisector-theorem.pdf}
\caption{三角形内外角平分线定理}
\label{fig:triangle-angle-bisector-theorem}
\end{figure}

如果外角平分线与$BC$边平行,可以视为点$F$无穷远,此时右边的$BF : FC$按极限理解,等式也是成立的。

只证明定理本身,逆定理根据同一法是容易得到的。

\begin{proof}[证明一]
  过点$C$分别作内角平分线和外角平分线的平行线,分别与$AB$边及其延长线相交于点$P$和$Q$,那么易得$AC=AP$和$AC=AQ$,所以

\begin{figure}[htbp]
\centering
\includegraphics{content/plane-geometry/pic/triangle-angle-bisector-theorem-proof.pdf}
\caption{三角形内外角平分线定理的证明}
\label{fig:triangle-angle-bisector-theorem-proof}
\end{figure}

  \begin{equation*}
    \frac{BE}{EC} = \frac{BA}{AP} = \frac{AB}{AC}
  \end{equation*}
  和
  \begin{equation*}
    \frac{BF}{FC} = \frac{BA}{AQ} = \frac{AB}{AC}
  \end{equation*}
于是定理得证。
\end{proof}

\begin{proof}[证明二]
  辅助线的作法同证明一,对三角形$QBC$和截线$AE$应用梅涅劳斯定理可得$BE:EC=AB:AC$,对三角形$PBC$和截线$AF$应用梅涅劳斯定理可得$BF:FC=AB:AC$。
\end{proof}

\begin{example}[阿波罗尼奥斯圆]
  今来考虑平面上到两个定点的距离之比为常数的动点轨迹,记这两个定点为$A$和$B$,动点$P$在运动过程中始终满足$PA:PB=\lambda(\neq 1)$,这里限制常数不等于1是因为那是线段$AB$的垂直平分线。

\begin{figure}[htbp]
\centering
\includegraphics{content/plane-geometry/pic/apollonius-circle.pdf}
\caption{Apollonius圆}
\label{fig:apollonius-circle}
\end{figure}

 首先在直线$AB$上可以找到两个点$E$和$F$,使得$\vv{AE}=\lambda\vv{EB}$和$\vv{AF}=-\lambda\vv{FB}$,这里点$E$在线段$AB$上而点$F$在延长线上,动点$P$在运动过程中恒有$PA:PB=AE:EB$和$PA:PB=AF:FB$,当$P$不在直线$AB$上时,根据三角形的内外角平分线定理的逆定理,就有$PE$和$PF$分别是$\angle APB$的内角平分线和外角平分线,所以$PE \perp PF$,于是点$P$的轨迹就是以线段$EF$为直径的圆。这个结论最早是由古希腊几何学家阿波罗尼奥斯\footnote{参考文献\cite{conic-sections}的作者。}(Apollonius)发现的,所以称为阿波罗尼奥斯圆。

 值得一提的是,点$E$和$F$分线段$AB$的比为$\lambda$,则点$A$和$B$分线段$EF$的比也相同,这个比值是$|1-\lambda|/(1+\lambda)$,证明是很容易的,只要把线段$AE$、$EB$、$BF$、$AF$的长度都用$AB$的长度来表示就能得到这一点,这个结论是有用的,比如说任给一个圆和一个不等于1的正实数,我们可以反过来找出它是关于哪两个点的阿波罗尼斯圆,当然这两个点不是唯一的,在每一条直径上都能找出两对。
\end{example}

\subsection{正弦定理和余弦定理}
\label{sec:sine-theorem-and-cosine-theorem}

\subsection{海伦公式}
\label{sec:helen-area-formula}

\begin{theorem}[海伦(Helen)公式]
  三角形面积$S$可表为(式中$s$为半周长)
  \begin{equation}
    \label{eq:helen-area-formula}
    S = \sqrt{s(s-a)(s-b)(s-c)}
  \end{equation}
\end{theorem}

\begin{proof}[证明]
  由面积公式
  \begin{eqnarray*}
    S & = & \frac{1}{2} bc \sin{A} \\
      & = & \frac{1}{2} bc \sqrt{1-\cos^2{A}} \\
      & = & \frac{1}{2} bc \sqrt{1- \left( \frac{b^2+c^2-a^2}{2bc} \right)^2} \\
      & = & \frac{1}{4} \sqrt{4b^2c^2-(b^2+c^2-a^2)^2} \\
      & = & \frac{1}{4} \sqrt{((b+c)^2-a^2)(a^2-(b-c)^2)} \\
      & = & \frac{1}{4} \sqrt{(a+b+c)(b+c-a)(a+b-c)(a+c-b)} \\
    & = & \sqrt{s(s-a)(s-b)(s-c)}
  \end{eqnarray*}
\end{proof}

  这与我国古代数学家秦九韶在《数书九章》中提出的三斜求积术是等价的(见推导过程):
  \begin{equation}
    \label{eq:three-inclined-quadrature-operation}
    S_{\triangle ABC} = \frac{1}{2} \sqrt{a^2b^2- \left( \frac{a^2+b^2-c^2}{2} \right)^2}
  \end{equation}
\end{example}

\subsection{一些三角恒等式}
\label{sec:some-equation-in-triangle}

\begin{property}
  \label{property:triangle-sinA-sinB-sinC-cosA-cosB-cosC}
  在三角形$ABC$中,有如下恒等式成立:
  \begin{eqnarray}
    \label{eq:triangle-sinA-sinB-sinC-cosA-cosB-cosC}
   \sin{A} + \sin{B} + \sin{C} & = & 4\cos{\frac{A}{2}}\cos{\frac{B}{2}}\cos{\frac{C}{2}} \\ 
   \cos{A} + \cos{B} + \cos{C} & = & 1 + 4 \sin{\frac{A}{2}}\sin{\frac{B}{2}}\sin{\frac{C}{2}}  
  \end{eqnarray}
\end{property}

\begin{proof}[证明]
  只证明第一个,第二个也是类似的。
  \begin{eqnarray*}
    \sin{A} + \sin{B} + \sin{C} & = & (\sin{A}+\sin{B})+\sin{(A+B)} \\
                                & = & 2 \sin{\frac{A+B}{2}} \cos{\frac{A-B}{2}} + 2 \sin{\frac{A+B}{2}} \cos{\frac{A+B}{2}} \\
                                & = & 4\sin{\frac{A+B}{2}} \cos{\frac{A}{2}} \cos{\frac{B}{2}} \\
    & = & 4\cos{\frac{A}{2}} \cos{\frac{B}{2}} \cos{\frac{C}{2}}
  \end{eqnarray*}
\end{proof}

\subsection{外接圆与外心}
\label{sec:triangle-excenter}

\begin{property}
  \label{ppty:triangle-circumradius-area-sides}
  三角形的外接圆半径$R$,面积$S$,和三边之间有如下的关系式
  \begin{equation}
    \label{eq:triangle-circumradius-area-sides}
    abc = 4RS
  \end{equation}
\end{property}

这性质意味着,如果以三角形为底面,外接圆半径为高作三棱柱,其体积与以三边为棱长的长方体的体积之间,有着固定的比例关系,不知这能否有构造法说明这一点?

\begin{proof}[证明一]
  由正弦定理得$2R=\frac{a}{\sin{A}}$,再由面积公式得$S=\frac{1}{2}bc\sin{A}$,两式相乘即得结论。
\end{proof}

\begin{proof}[证明二]
  由正弦定理和海伦公式,过程中$p=\frac{1}{2}(a+b+c)$为半周长
  \begin{eqnarray*}
    2R & = & \frac{a}{\sin{A}} \\
       & = & \frac{a}{\sqrt{1-\cos^2{A}}} \\
       & = & \frac{a}{\sqrt{1-\left( \frac{b^2+c^2-a^2}{2bc} \right)^2}} \\
       & = & \frac{2abc}{\sqrt{4b^2c^2-(b^2+c^2-a^2)^2}} \\
       & = & \frac{2abc}{\sqrt{((b+c)^2-a^2)(a^2-(b-c)^2)}} \\
       & = & \frac{2abc}{\sqrt{(a+b+c)(b+c-a)(a+b-c)(c+a-b)}} \\
       & = & \frac{abc}{2 \sqrt{p(p-a)(p-b)(p-c)}} \\
    & = & \frac{abc}{2S}
  \end{eqnarray*}
\end{proof}
证明二实际上是又把海伦公式推了一遍。


\subsection{内切圆与内心}
\label{sec:triangle-incenter}

\begin{property}
  \label{property:triangle-incenter-ki-kb-kc-equal}
  三角形$ABC$中,设顶点$A$与内心$I$所在直线与三角形外接圆相交于另一点$K$,则$KB=KI=KC$。
\end{property}

\begin{figure}[htbp]
\centering
\includegraphics{content/plane-geometry/pic/triangle-incenter-ki-kb-kc-equal.pdf}
\caption{}
\label{fig:triangle-incenter-ki-kb-kc-equal}
\end{figure}

\begin{proof}[证明]
  $KB=KC$是显然的,只证明$KI=KB$,易得$\angle KIB=\angle KAB + \angle IBA = (\angle BAC + \angle ABC)/2$,同时$\angle KBI = \angle KBC + \angle IBC = \angle KAC + \angle IBC = (\angle BAC + \angle ABC)/2$,所以$KB=KI$。
\end{proof}

\begin{property}
  \label{property:ration-inradius-and-circumradius-of-triangle}
  三角形的内切圆半径$r$和外接圆半径$R$之间有如下等量关系
  \begin{equation}
    \label{eq:inradius-and-circumradius-ration}
    \frac{r}{R} = 4 \sin{\frac{A}{2}} \sin{\frac{B}{2}} \sin{\frac{C}{2}}
  \end{equation}
\end{property}

\begin{proof}[证明]
  因为$r=\frac{1}{2}(b+c-a)\tan{\frac{A}{2}}$,$R=a/(2\sin{A})$,所以
  \begin{eqnarray*}
    \frac{r}{R} & = & \frac{b+c-a}{a} \cdot \tan{\frac{A}{2}}\sin{A}\\
                & = & \frac{\sin{B} + \sin{C} - \sin{A}}{\sin{A}} \cdot \tan{\frac{A}{2}}\sin{A} \\
                & = & (\sin{B} + \sin{C} - \sin{A}) \tan{\frac{A}{2}} \\
                & = & (\sin{B} + \sin{C} - \sin{(B+C)}) \tan{\frac{A}{2}} \\
                & = & \left( 2\sin{\frac{B+C}{2}}\cos{\frac{B-C}{2}}-2\sin{\frac{B+C}{2}}\cos{\frac{B+C}{2}} \right) \tan{\frac{A}{2}} \\
                & = & 4\sin{\frac{B+C}{2}}\sin{\frac{B}{2}}\sin{\frac{C}{2}}\tan{\frac{A}{2}} \\
    & = & 4 \sin{\frac{A}{2}}\sin{\frac{B}{2}}\sin{\frac{C}{2}}
  \end{eqnarray*}
\end{proof}


\begin{inference}
  三角形的外心到三边的距离的代数和是$R+r$,代数和是指:对于某一边,如果外心和另一个顶点在此边所在直线同侧,则外心到这边的距离取正值,否则取负值,$R$和$r$分别指代外接圆半径和内切圆半径。
\end{inference}

\begin{proof}[证明]
  由外心性质,容易得这个距离和是$R(\cos{A}+\cos{B}+\cos{C})$,结合性质\ref{property:triangle-sinA-sinB-sinC-cosA-cosB-cosC}中的三角恒等式和性质\ref{property:ration-inradius-and-circumradius-of-triangle},即可得证。
\end{proof}

\begin{property}
  三角形的内切圆半径$r$,外接圆半径$R$,以及半周长$s$,与三边之间有如下关系
  \begin{equation}
    \label{eq:triangle-abc-equal-4Rrs}
    abc = 4Rrs
  \end{equation}
\end{property}

\begin{proof}[证明]
 由性质\ref{ppty:triangle-circumradius-area-sides},结合面积$S=\frac{1}{2}(a+b+c)r=sr$即得证。
\end{proof}

\subsection{欧拉不等式}
\label{sec:euler-inequality}

\begin{theorem}[欧拉(Euler)不等式]
  三角形的外接圆半径$R$和内切圆半径$r$满足不等式$R \geqslant 2r$,等号仅当三角形为正三角形时取得。
\end{theorem}

\begin{proof}[证明一]
  由性质\ref{ppty:triangle-circumradius-area-sides}和面积公式$S=sr$,这里$s$为半周长,欲证不等式等价于
  \begin{equation*}
    \frac{abc}{4S} \geqslant \frac{2S}{s}
  \end{equation*}
  即要证$8S^2 \leqslant sabc$,由海伦公式,只需证
  \begin{eqnarray*}
    & \Longleftrightarrow & 8s(s-a)(s-b)(s-c) \leqslant sabc \\
    & \Longleftrightarrow & (a+b-c)(b+c-a)(c+a-b) \leqslant abc
  \end{eqnarray*}
  作代换 $x=a+b-c$,$y=b+c-a$,$z=c+a-b$,则又等价于
  \begin{equation*}
    xyz \leqslant \frac{x+z}{2} \cdot \frac{y+x}{2} \cdot \frac{z+y}{2}
  \end{equation*}
  由均值不等式,这显然成立,易见取等条件是三角形为正三角形。
\end{proof}

\begin{proof}[证明二]
  由$2R=\frac{a}{\sin{A}}$及$r=\frac{1}{2}(b+c-a)\tan{\frac{A}{2}}$得
  \begin{eqnarray*}
    \frac{r}{R} & = & \frac{(b+c-a)}{a}\sin{A}\tan{\frac{A}{2}} \\
                & = & \frac{(b+c-a)}{a} \cdot 2\sin^2{\frac{A}{2}} \\
                & = & \frac{(b+c-a)}{a} (1-\cos{A}) \\
                & = & \frac{(b+c-a)}{a} \left( 1-\frac{b^2+c^2-a^2}{2bc} \right) \\
                & = & \frac{(b+c-a)[a^2-(b-c)^2]}{2abc} \\
    & = & \frac{(b+c-a)(c+a-b)(a+b-c)}{2abc}
  \end{eqnarray*}
  因此要证的不等式等价于
  \[
    (b+c-a)(c+a-b)(a+b-c) \leqslant abc
  \]
  以下同证明一。
\end{proof}

\begin{proof}[证明三]
  由性质\ref{property:ration-inradius-and-circumradius-of-triangle},及均值不等式和正弦函数在区间$(0,\pi)$上的上凸性质,有
  \begin{eqnarray*}
    \frac{r}{R} & = & 4\sin{\frac{A}{2}}\sin{\frac{B}{2}}\sin{\frac{C}{2}} \\
                & \leqslant & 4 \left( \frac{\sin{\frac{A}{2}}+\sin{\frac{B}{2}}+\sin{\frac{C}{2}}}{3} \right)^3 \\
                & \leqslant & 4 \left( \sin{\frac{\frac{A}{2}+\frac{B}{2}+\frac{C}{2}}{3}} \right)^3 \\
    & = & \frac{1}{2}
  \end{eqnarray*}
\end{proof}

\subsection{重心}
\label{sec:triangle-centroid}

\begin{theorem}
  三角形的三条中线相交于一点。
\end{theorem}


三条中线的交点称为三角形的 \emph{重心}。

\begin{figure}[htbp]
\centering
\includegraphics{content/plane-geometry/pic/centroid-of-triangle.pdf}
\caption{三角形的重心}
\label{fig:centroid-of-triangle}
\end{figure}

根据塞瓦定理,这是显然的,在例\ref{example:barycentric-of-triangle-vector}中也给出了向量方法的证明,这里再给出一个纯面积方法的证明:
\begin{proof}[证明]
  先由$AB$边上的中线$CE$和$AC$边上的中线$BF$相交得交点$G$,易得$S_{\triangle AGC} = S_{\triangle BGC}$ 和 $S_{\triangle AGB} = S_{\triangle CGB}$,从而得到$S_{\triangle AGB} = S_{\triangle AGC}$,所以$AG$延长后与$BC$的交点$D$必是$BC$边中点,得证。
\end{proof}

\subsection{垂心}
\label{sec:triangle-orthocentre}

\begin{property}
  \label{property:circumcenter-orthocenter-ah-equal-2od}
  在三角形$ABC$中,$O$和$H$分别为外心和垂心,$D$是$BC$边中点,则有$OD \parallel AH$并且$OD = \frac{1}{2} AH$。
\end{property}

\begin{figure}[htbp]
\centering
\includegraphics{content/plane-geometry/pic/property-circumcenter-orthocenter-oh-equal-2od.pdf}
\caption{外心和垂心的一个性质}
\label{fig:property-circumcenter-orthocenter-oh-equal-2od}
\end{figure}

\begin{proof}[证明]
  平行是明显的,只证明2倍比例关系,如图\ref{fig:property-circumcenter-orthocenter-oh-equal-2od-proof},作出外接圆和过$B$的直径$BP$,连接$AP$、$CP$,那么显然有$OD \parallel CP$并且$OD=\frac{1}{2}CP$,又容易证得四边形$AHCP$为平行四边形,所以$AH$与$CP$平行且相等,故得证。

\begin{figure}[htbp]
\centering
\includegraphics{content/plane-geometry/pic/property-circumcenter-orthocenter-oh-equal-2od-proof.pdf}
\caption{}
\label{fig:property-circumcenter-orthocenter-oh-equal-2od-proof}
\end{figure}

\end{proof}

\subsection{欧拉(Euler)定理与欧拉公式}
\label{sec:euler-theorem-and-euler-formula}

\begin{theorem}[欧拉(Euler)定理]
  三角形$ABC$的外心$O$,重心$G$,垂心$H$共线且$OG=\frac{1}{2}GH$。
\end{theorem}

定理中的这条直线称为三角形的 \emph{欧拉线}。

\begin{figure}[htbp]
\centering
\includegraphics{content/plane-geometry/pic/euler-circumcenter-orthocenter-barycenter-line.pdf}
\caption{}
\label{fig:euler-circumcenter-orthocenter-barycenter-line}
\end{figure}

\begin{proof}[证明]
  作出中线$AD$,假如直线$OG$与直线$AH$相交于点$H'$,那么由重心性质,$AG:GD=2:1$,而$OD \parallel AH$,所以$AH'=2OD$,根据性质\ref{property:circumcenter-orthocenter-ah-equal-2od},$H'$必是垂心,得证。
\end{proof}

\begin{theorem}
  三角形的外接圆圆心和内切圆圆心的距离是 $d=\sqrt{R(R-2r)}$,其中$R$和$r$分别指代外接圆半径和内切圆半径。
\end{theorem}

\begin{proof}[证明一]
  如图\ref{fig:triangle-distance-of-incenter-circumcenter},设外心和内心连线所在直线与外接圆相交于$U$、$V$两点,角$A$的平分线与外接圆相交于另一点$K$,由相交弦定理有$AI \cdot IK = UI \cdot IV$,下面来计算各条线段长度。
 
\begin{figure}[htbp]
\centering
\includegraphics{content/plane-geometry/pic/triangle-distance-of-incenter-circumcenter.pdf}
\caption{}
\label{fig:triangle-distance-of-incenter-circumcenter}
\end{figure}

由性质\ref{property:triangle-incenter-ki-kb-kc-equal},有$KI=KB=2R\sin{\frac{A}{2}}$,$AI=r/ \sin{\frac{A}{2}}$,记$OI=d$则$UI\cdot VI=(R+d)(R-d)$,于是$R^2-d^2=2Rr$,得证。
\end{proof}

\begin{proof}[证明二]
  在三角形$AOI$中,对$\angle OAI$使用余弦定理,由$AO=R$, $AI=r/\sin{\frac{A}{2}}$,$\angle OAI = \left| \angle OAB - \angle IAB \right| = \left| \left( \frac{\pi}{2}-B \right) - \frac{A}{2} \right| = \left| \frac{\pi}{2} - \left( B+\frac{A}{2} \right) \right|$,再利用性质\ref{property:ration-inradius-and-circumradius-of-triangle},有
  \begin{eqnarray*}
    d^2 & = & R^2+\left( \frac{r}{\sin{\frac{A}{2}}} \right)^2 - 2R \cdot \frac{r}{\sin{\frac{A}{2}}} \cdot \cos{\left( \frac{\pi}{2}-\left( B+\frac{A}{2} \right) \right)} \\
        & = & R^2 + \frac{r^2}{\sin^2 \frac{A}{2}} -2Rr \frac{\sin{\left( B+\frac{A}{2} \right)}}{\sin{\frac{A}{2}}} \\
    & = & R^2 + 2Rr \cdot \frac{2\sin{\frac{B}{2}}\sin{\frac{C}{2}}}{\sin{\frac{A}{2}}}-2Rr \cdot \frac{\sin{\left( B+\frac{A}{2} \right)}}{\sin{\frac{A}{2}}} \\
        & = & R^2 + 2Rr \left( \frac{2\sin{\frac{B}{2}}\sin{\frac{C}{2}}}{\sin{\frac{A}{2}}} - \frac{\sin{\left( B+\frac{A}{2} \right)}}{\sin{\frac{A}{2}}} \right) \\
        & = & R^2 + 2Rr \left( \frac{2\sin{\frac{B}{2}}\sin{\frac{C}{2}}}{\sin{\frac{A}{2}}} - \frac{\cos{\left( \frac{B-C}{2} \right)}}{\sin{\frac{A}{2}}} \right) \\
        & = & R^2 - 2Rr \cdot \frac{\cos{\left( \frac{B+C}{2} \right)}}{\sin{\frac{A}{2}}} \\
    & = & R^2 - 2Rr
  \end{eqnarray*}
\end{proof}

\subsection{旁切圆与旁心}
\label{sec:escenter}



%%% Local Variables:
%%% TeX-master: "../../book"
%%% End:
