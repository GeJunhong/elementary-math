
\section{几个重要的定理}
\label{sec:some-import-plane-geometry-theorem}

本节讲述几个在平面几何中占有重要地位的定理。

\subsection{张角定理}
\label{sec:spread-angle-theorem}

\begin{theorem}[张角定理]
  在三角形$ABC$中,$D$是$BC$边上任意一点,有下式成立:
  \begin{equation}
    \label{eq:spread-angle-theorem}
    \frac{\sin{\angle{BAC}}}{AD} = \frac{\sin{\angle{CAD}}}{AB} + \frac{\sin{\angle{BAD}}}{AC}
  \end{equation}
\end{theorem}

张角定理的逆也是成立的,即如果平面上四个点满足定理中的等式,就有$B$、$C$、$D$三点共线,这也可以用来处理共线问题。

  \begin{figure}[htbp]
  \centering
\includegraphics{content/plane-geometry/pic/spread-angle-theorem.pdf}
\caption{张角定理}
\label{fig:spread-angle-theorem}
\end{figure}

\begin{proof}[证明一]
  因为$S_{\triangle ABC} = S_{\triangle ABD} + S_{\triangle ACD}$,所以
  \begin{equation*}
    \frac{1}{2}AB \cdot AC \cdot \sin{\angle BAC} =
    \frac{1}{2}AB \cdot AD \cdot \sin{\angle BAD} +
    \frac{1}{2}AC \cdot AD \cdot \sin{\angle CAD} 
  \end{equation*}
  两边同除以$\frac{1}{2}AB \cdot AC \cdot AD$即得证。
\end{proof}

\begin{proof}[证明二]
  由正弦定理有 $AB \sin{\angle BAC} = BC \sin{\angle ACB}$以及$AD \sin{\angle CAD} = CD \sin{\angle ACD}$,两式相除得
  \[
    \frac{\sin{\angle CAD}}{\sin{\angle BAC}} \cdot \frac{AD}{AB} = \frac{CD}{BC}
  \]
  同理也有
  \[
    \frac{\sin{\angle BAD}}{\sin{\angle BAC}} \cdot \frac{AD}{AC} = \frac{BD}{BC}
  \]
  两式相加再整理即得定理中等式。
\end{proof}

\subsection{梅涅劳斯(Menelaus)定理}
\label{sec:menelaus-theorem}

\begin{theorem}[梅涅劳斯定理]
  设$X$、$Y$、$Z$是三角形$ABC$的三边$AB$、$BC$、$CA$或其延长线上的点,其中有奇数个点在边的延长线上,则此三点共线的充分必要条件是下式成立:
  \begin{equation}
    \label{eq:menelaus-theorem}
    \frac{AX}{XB} \cdot \frac{BY}{YC} \cdot \frac{CZ}{ZA} = 1
 \end{equation}
\end{theorem}
 
\begin{figure}[htbp]
\centering
\includegraphics{content/plane-geometry/pic/menelaus-theorem.pdf}
\includegraphics{content/plane-geometry/pic/menelaus-theorem2.pdf}
\caption{梅涅劳斯定理}
\label{fig:menelaus-theorem}
\end{figure}
 
%\begin{figure}[htbp]
%\centering
%\includegraphics{content/plane-geometry/pic/menelaus-theorem2.pdf}
%\caption{梅涅劳斯定理:三个点都在边的延长线上的情形}
%\label{fig:menelaus-theorem2}
%\end{figure}

只证明必要性,得证之后利用同一法即可轻松得到充分性的证明。
最精巧的是下面这个辅助线的证明
\begin{proof}[证明一]
  过$C$作$XYZ$的平行线,与边$AB$相交于点$P$,那么有
 
\begin{figure}[htbp]
\centering
\includegraphics{content/plane-geometry/pic/menelaus-theorem-proof.pdf}
\includegraphics{content/plane-geometry/pic/menelaus-theorem-proof2.pdf}
\caption{梅涅劳斯定理的证明}
\label{fig:menelaus-theorem-proof}
\end{figure}
 
%\begin{figure}[htbp]
%\centering
%\includegraphics{content/plane-geometry/pic/menelaus-theorem-proof2.pdf}
%\caption{梅涅劳斯定理的证明}
%\label{fig:menelaus-theorem-proof2}
%\end{figure}

  \begin{equation*}
    \frac{AX}{XB} \cdot \frac{BY}{YC} \cdot \frac{CZ}{ZA} =
    \frac{AX}{XB} \cdot \frac{BX}{XP} \cdot \frac{PX}{XA} = 1
  \end{equation*}
所以必要性成立,充分性利用同一法即可得证,略去。
\end{proof}

另外一种辅助线的作法是分别过三角形的三个顶点向直线$XYZ$引垂线,剩下的过程与上面类似,此处就不再详述了。

面积方法仍然是行之有效的手段:
\begin{proof}[证明二]
  \begin{equation*}
    \frac{AX}{XB} \cdot \frac{BY}{YC} \cdot \frac{CZ}{ZA} =
    \frac{S_{\triangle AYX}}{S_{\triangle BYX}} \cdot
    \frac{S_{\triangle BYX}}{S_{\triangle CYX}} \cdot
    \frac{S_{\triangle CYX}}{S_{\triangle AYX}} = 1
  \end{equation*}
\end{proof}

再给一个向量方法的证明:
\begin{proof}[证明三]
  记$\vv{AX}=\lambda_1\vv{XB}$, $\vv{BY}=\lambda_2\vv{YC}$, $\vv{CZ}=\lambda_3\vv{ZA}$,则有
  \begin{equation*}
    \vv{XZ} = \vv{AZ}-\vv{AX}=\frac{1}{1+\lambda_3}\vv{AC}-\frac{\lambda_1}{1+\lambda_1}\vv{AB}
  \end{equation*}
  同时
  \begin{eqnarray*}
    \vv{XY} &=& \vv{AY} - \vv{AX} = \left( \frac{1}{1+\lambda_2}\vv{AB}+\frac{\lambda_2}{1+\lambda_2}\vv{AC} \right) - \frac{\lambda_1}{1+\lambda_1}\vv{AB} \\
    &=& \left( \frac{1}{1+\lambda_2}-\frac{\lambda_1}{1+\lambda_1} \right)\vv{AB}+\frac{\lambda_2}{1+\lambda_2}\vv{AC} 
  \end{eqnarray*}
这两个向量共线的充分必要条件是上面这两组系数对应成比例,经过计算即为等式$\lambda_1\lambda_2\lambda_3=-1$,因为是有奇数个点在边的延长线上,所以它等价于定理中的等式。
\end{proof}

从向量形式中我们得到了梅涅劳斯定理的向量表达: $\lambda_1\lambda_2\lambda_3=-1$,而且向量形式的证明,充分性与必要性是统一的。

\begin{example}
  \label{example:external-common-tangent-for-3-circle}
  考虑三个半径两两不同的圆,有三组外公切线,每两个圆的一组外公切线都有一个交点,这样的交点有三个,今将说明这三点共线\footnote{这个例子及证法来自于参考文献\cite{kuing-problem-book}}。
 
\begin{figure}[htbp]
\centering
\includegraphics{content/plane-geometry/pic/external-common-tangent-for-3-circle.pdf}
\caption{三个半径两两不同的圆的三组外公切线交点共线}
\label{fig:external-common-tangent-for-3-circle}
\end{figure}

如图\ref{fig:external-common-tangent-for-3-circle},易知$P$、$Q$、$R$三点在三个圆心形成的三角形的三条边的延长线上,而$\frac{PA}{PB}=\frac{r_A}{r_B}$,仿此有另外两式,三式相乘即知这满足梅涅劳斯定理中的等式,所以这三点共线。

 这结论还有另外一种解释,把这三个圆看成三个球,这三个球有公切面(实际上有两个),这三个点$P$、$Q$、$R$将同时位于这公切面和三个球心所决定的平面上,从而位于两个平面的交线上(三个球半径两两不等,所以这两个平面必定相交)。那这三个点为什么在公切面上呢,可以这样理解,每两个球有一个公切圆锥,三个公切圆锥的顶点就是这三个点,而这三个圆锥与前述公切面必定都相切于一条圆锥的母线,所以作为圆锥顶点必定在这条母线上,也就在公切面上。
\end{example}

\subsection{塞瓦(Ceva)定理}
\label{sec:cevian-theorem}

\begin{theorem}[塞瓦定理]
  在三角形$ABC$中,点$X$、$Y$、$Z$分别是三边$AB$、$BC$、$CA$或其延长线上的点,其中有偶数个点在边的延长线上,则三条线段$AY$、$BZ$、$CX$交于一点$P$或者相互平行的充分必要条件是
  \begin{equation}
    \label{eq:cevian-theorem}
    \frac{AX}{XB} \cdot \frac{BY}{YC} \cdot \frac{CZ}{ZA} = 1
  \end{equation}
\end{theorem}
 
\begin{figure}[htbp]
\centering
\includegraphics[scale=0.8]{content/plane-geometry/pic/cevian-theorem.pdf}
\includegraphics[scale=0.8]{content/plane-geometry/pic/cevian-theorem2.pdf}
\includegraphics[scale=0.8]{content/plane-geometry/pic/cevian-theorem3.pdf}
\caption{塞瓦定理}
\label{fig:cevian-theorem}
\end{figure}

三线平行也可以视为相交于无穷远的点。

同样只证明必要性,第一个证明是直接使用了梅涅劳斯定理:
\begin{proof}[证明一]
  对三角形$ABY$和截线$XPC$使用梅涅劳斯定理得
  \begin{equation*}
    \frac{AX}{XB} \cdot \frac{BC}{CY} \cdot \frac{YP}{PA} = 1
  \end{equation*}
  再对三角形$ACY$和截线$BPZ$使用梅涅劳斯定理得
  \begin{equation*}
    \frac{YB}{BC} \cdot \frac{CZ}{ZA} \cdot \frac{AP}{PY} = 1
  \end{equation*}
  两式相乘即得塞瓦定理。
\end{proof}

面积方法仍然是最直接的:
\begin{proof}[证明二]
  \begin{equation*}
    \frac{AX}{XB} \cdot \frac{BY}{YC} \cdot \frac{CZ}{ZA} =
    \frac{S_{\triangle CPA}}{S_{\triangle CPB}} \cdot
    \frac{S_{\triangle APB}}{S_{\triangle APC}} \cdot
    \frac{S_{\triangle BPC}}{S_{\triangle BPA}} = 1
  \end{equation*}
\end{proof}

同样有向量方法
\begin{proof}[证明三]
  仍然记$\vv{AX}=\lambda_1\vv{XB}$,$\vv{BY}=\lambda_2\vv{YC}$,$\vv{CZ}=\lambda_3\vv{ZA}$,有
  \begin{equation*}
    \vv{AX}=\frac{\lambda_1}{1+\lambda_{1}}\vv{AB}, \ \vv{AZ}=\frac{1}{1+\lambda_3}\vv{AC}
  \end{equation*}
  因为点$P$是$BZ$与$CX$的交点,所以存在两个实数$s$和$t$,使得
  \begin{eqnarray*}
    \vv{AP} &=& (1-s)\vv{AB}+s\vv{AZ} = (1-s)\vv{AB}+\frac{s}{1+\lambda_3}\vv{AC} \\
    \vv{AP} &=& (1-t)\vv{AC}+t\vv{AX} =  \frac{t\lambda_1}{1+\lambda_{1}}\vv{AB}+(1-t)\vv{AC}
  \end{eqnarray*}
  于是得方程组
  \begin{eqnarray}
    1-s &=&  \frac{t\lambda_1}{1+\lambda_{1}} \\
    \frac{s}{1+\lambda_3} &=& 1-t
  \end{eqnarray}
  解之得
  \begin{equation*}
    t = 1- \frac{1}{(1+\lambda_1)(1+\lambda_3)-\lambda_1}
  \end{equation*}
  所以
  \begin{equation*}
    \vv{AP}
    = \left( 1- \frac{1}{(1+\lambda_1)(1+\lambda_3)-\lambda_1} \right) \cdot \frac{\lambda_1}{1+\lambda_{1}} \vv{AB}+\frac{1}{(1+\lambda_1)(1+\lambda_3)-\lambda_1}\vv{AC}
  \end{equation*}
  另一方面有
  \begin{equation*}
    \vv{AY}=\frac{1}{1+\lambda_2}\vv{AB}+\frac{\lambda_2}{1+\lambda_2}\vv{AC}
  \end{equation*}
  于是$\vv{AP}$与$\vv{AY}$共线的充分必要条件是两组系数成比例,经简单计算知上面两个式子中的系数比分别为$\lambda_1\lambda_3 : 1$和$1 : \lambda_2$,所以得到$\lambda_1\lambda_2\lambda_3=1$,即得证。
\end{proof}

\begin{example}[梅涅劳斯定理与塞瓦定理的统一]
根据这证明过程,实际上梅涅劳斯定理和塞瓦定理可以统一起来:假定$X$、$Y$、$Z$是三角形$ABC$三边$AB$、$BC$、$CA$或它们的延长线上的点,有分比$\vv{AX}=\lambda_1\vv{XB}$,$\vv{BY}=\lambda_2\vv{YC}$,$\vv{CZ}=\lambda_3\vv{ZA}$,那么这三点共线的充分必要条件是$\lambda_1\lambda_2\lambda_3=-1$(梅涅劳斯定理),而$AY$、$BZ$、$CX$三线共点的充分必要条件是$\lambda_1\lambda_2\lambda_3=1$(塞瓦定理)。
\end{example}

\begin{example}[三角形的几个心]
 利用塞瓦定理,我们知道三角形的三条中线是交于同一点的,该点称为三角形的 \emph{重心}。

 同样,利用塞瓦定理和三角形内角平分线定理,还可以知道三角形的三条内角平分线也是交于同一点的,这点称为三角形的 \emph{内心},它是三角形内切圆圆心。

 对于三角形的一个角的内角平分线和另外两个角的外角平分线,根据三角形的内外角平分线定理,可得这三线也是相交于同一点的,这一点称为三角形的 \emph{旁心},它是旁切圆的圆心。

 对于三条高线,假定$BC$边上的高线垂足是$D$,则有$\frac{BD}{CD}=\frac{c \cos{B}}{b \cos{C}}$,仿此有另外两式,三式相乘并根据塞瓦定理即知三条高也相交于同一点,称该点为三角形的 \emph{垂心}。

这四个心再加上三角形的\emph{外心}(三边的中垂线交点),合称为三角形的五心。 
\end{example}

\begin{example}
  仿照例\ref{example:external-common-tangent-for-3-circle},把那里的外公切线改为内公切线,那么图\ref{fig:internal-common-tangent-for-3-circle}中的三线共点。
 
\begin{figure}[htbp]
\centering
\includegraphics{content/plane-geometry/pic/internal-common-tangent-for-3-circle.pdf}
\caption{三个半径两两不同的圆的三组内公切线交点形成的塞瓦三线}
\label{fig:internal-common-tangent-for-3-circle}
\end{figure}

因为$\frac{AP}{PB}=\frac{r_A}{r_B}$,同样有另外两式,三式相乘后由塞瓦定理知三线共点。
\end{example}

\begin{example}
  如图\ref{fig:ceva-circle},圆周上有顺次六个点$A$、$B$、$C$、$D$、$E$、$F$,那么三条线段$AD$、$BE$、$CF$交于一点的充分必要条件是
  \begin{equation*}
    \frac{AB}{BC} \cdot \frac{CD}{DE} \cdot \frac{EF}{FA} = 1
  \end{equation*}
 
\begin{figure}[htbp]
\centering
\includegraphics{content/plane-geometry/pic/ceva-circle.pdf}
\caption{}
\label{fig:ceva-circle}
\end{figure}

证明是很容易的,连接$AC$、$CE$、$EA$,对三角形$ACE$使用塞瓦定理就有
\begin{equation*}
  \frac{AX}{XC} \cdot \frac{CY}{YE} \cdot \frac{EZ}{ZA} = 1
\end{equation*}
但是又有
\begin{equation*}
  \frac{AX}{XC} = \frac{AB \sin{\angle ABX}}{BC \sin{\angle CBX}}
\end{equation*}
仿此有另外两式,三式相乘并利用 $\angle ABX = \angle EDY$等一系列等角,即可得结论的必要性,而充分性仍然由同一法得出,略去。
\end{example}

\begin{example}
  如图\ref{fig:example-for-cevian-ap-perp-bc},在$\triangle ABC$中,点$D$是角$BAC$的平分线与$BC$边的交点,过$D$向$AB$和$AC$引垂线,垂足分别是$E$、$F$,连接$BF$和$CE$相交于点$P$,今来证明$AP \perp BC$。

\begin{figure}[htbp]
\centering
\includegraphics{content/plane-geometry/pic/example-for-cevian-ap-perp-bc.pdf}
\caption{}
\label{fig:example-for-cevian-ap-perp-bc}
\end{figure}

延长$AP$与$BC$边相交于点$Q$,那么由塞瓦定理有
\begin{equation*}
  \frac{AE}{EB} \cdot \frac{BQ}{QC} \cdot \frac{CF}{FA} = 1
\end{equation*}
另外显然有$AE=AF$,所以由上式得$\frac{BQ}{QC}=\frac{EB}{FC}$,而由几何关系和角平分线定理得
\begin{equation*}
  \frac{EB}{FC} = \frac{BD \cos{B}}{CD \cos{C}} = \frac{AB \cos{B}}{AC \cos{C}}
\end{equation*}
所以
\begin{equation*}
  \frac{BQ}{QC} = \frac{AB \cos{B}}{AC \cos{C}}
\end{equation*}
这说明点$Q$与$BC$边上的高线垂足重合,所以$AP$就是高线。
\end{example}

\begin{example}
 如图\ref{fig:example-for-cevian-ade-equal-adf},在$\triangle ABC$中,$AD$是$BC$边上的高线,点$P$是这高线上任一点,直线$CP$、$FP$分别与边$AB$、$AC$相交于$E$、$F$,连接$DE$和$DF$,求证$\angle ADE = \angle ADF$。

\begin{figure}[htbp]
\centering
\includegraphics{content/plane-geometry/pic/example-for-cevian-ade-equal-adf.pdf}
\caption{}
\label{fig:example-for-cevian-ade-equal-adf}
\end{figure}

这是某年的全国高中联赛试题,也是一道名题,证法很多,从纯几何方法到坐标方法都有,这里只提供两个利用塞瓦定理的证明。

\begin{proof}[证明一]
由塞瓦定理有
\begin{equation*}
  \frac{AE}{EB} \cdot \frac{BD}{DC} \cdot \frac{CF}{FA} = 1
\end{equation*}
另一方面有
\begin{equation*}
  \frac{AE}{EB} = \frac{S_{\triangle AED}}{S_{\triangle BED}} = \frac{AD \sin{\angle ADE}}{BD \sin{\angle BDE}} = \frac{AD \sin{\angle ADE}}{BD \cos{\angle ADE}}
\end{equation*}
同理有
\begin{equation*}
  \frac{AF}{FC} = \frac{AD \sin{\angle ADF}}{CD \cos{\angle ADF}}
\end{equation*}
这两式相除得
\begin{equation*}
  \frac{AE}{EB} \cdot \frac{BD}{DC} \cdot \frac{CF}{FA} = \frac{\sin{\angle ADE} \cos{\angle ADF}}{\cos{\angle ADE} \sin{\angle ADF}} 
\end{equation*}
将这式子与由塞瓦定理所得式子相比较得$\sin{\angle ADE} \cos{\angle ADF} = \cos{\angle ADE} \sin{\angle ADF}$,于是$\sin{(\angle ADE - \angle ADF)} = 0$,而这两个都是锐角,所以$\angle ADE = \angle ADF$。
\end{proof}

\begin{proof}[证明二]
  如图\ref{fig:example-for-cevian-ade-equal-adf},过点$A$作$BC$边的平行线,分别与$DE$和$DF$和延长线相交于$T$、$S$,由塞瓦定理
  \begin{equation*}
    \frac{AE}{EB} \cdot \frac{BD}{DC} \cdot \frac{CF}{FA} = 1
  \end{equation*}
  但由于
  \begin{equation*}
    \frac{AE}{EB} = \frac{AT}{BD}, \  \frac{CF}{FA} = \frac{CD}{AS}
  \end{equation*}
  所以得到$AT = AS$,结合$TS \perp AD$知$\angle ADE = \angle ADF$。
\end{proof}
\end{example}

\subsection{斯特瓦尔特(Stewart)定理}
\label{sec:stewart-theorem}

\begin{theorem}[斯特瓦尔特定理]
  在三角形$ABC$中,点$D$是$BC$边上任一点,有下式成立:
  \begin{equation}
    \label{eq:stewart-theorem}
    CD \cdot AB^2 + BD \cdot AC^2 = BC \cdot AD^2 + BD \cdot CD \cdot BC
  \end{equation}
\end{theorem}

\begin{figure}[htbp]
\centering
\includegraphics{content/plane-geometry/pic/stewart-theorem.pdf}
\caption{斯特瓦尔特定理}
\label{fig:stewart-theorem}
\end{figure}

这定理中的等式较长,不太容易记住,写成如下的含比例的式子或许会有所帮助:
\begin{equation}
  \label{eq:stewart-theorem-rate}
  \frac{CD}{BC}\cdot AB^2 + \frac{BD}{BC} \cdot AC^2 = AD^2 + BD \cdot CD
\end{equation}

\begin{proof}[证明一]
  由余弦定理
  \begin{eqnarray*}
    \cos{\angle ADB} & = & \frac{AD^2+BD^2-AB^2}{2 AD \cdot BD} \\
    \cos{\angle ADC} & = & \frac{AD^2+CD^2-AC^2}{2 AD \cdot CD} 
  \end{eqnarray*}
  因为$\angle ADB$与$\angle ADC$互补,所以$\cos{\angle ADB} + \cos{\angle ADC} = 0$,将上两式相加化简就得定理中的等式。
\end{proof}

\begin{proof}[证明二]
  设$\vv{BD}=\lambda \vv{DC}$,则
  \begin{equation*}
    \vv{AD}=\frac{1}{1+\lambda}\vv{AB}+\frac{\lambda}{1+\lambda}\vv{AC}
  \end{equation*}
  将$\lambda$换为$\frac{BD}{DC}$,就是
  \begin{equation*}
    \vv{AD}=\frac{CD}{BC}\vv{AB}+\frac{BD}{BC}\vv{AC}
  \end{equation*}
  两边平方
  \begin{equation*}
    AD^2=\frac{CD^2}{BC^2} \cdot AB^2 + \frac{BD^2}{BC^2} \cdot AC^2 + \frac{ BD \cdot CD}{BC^2} \cdot 2 AB \cdot AC \cdot \cos{\angle BAC}
  \end{equation*}
  利用余弦定理将上式中最后一个余弦值换掉,化简就得定理中的等式。
\end{proof}

\begin{example}
  \label{example:ptolemy-inequality-for-triangle}
  在这个例子中,我们来建立一个有用的不等式,在三角形$ABC$中,点$D$是$BC$边上的异于端点的任意一点,那么有如下不等式成立:
  \begin{equation}
    \label{eq:definite-proportion-inequality}
    AD < \frac{CD}{BC} \cdot AB + \frac{BD}{BC} \cdot AC
  \end{equation}
  或者写成乘积的形式
  \begin{equation*}
    BC \cdot AD < CD \cdot AB + BD \cdot AC
  \end{equation*}
  
  这个不等式与四边形中的托勒密不等式极为相似,实际上它就是四边形退化为三角形的情形,这个例子就是说托勒密不等式在这种情况下仍然成立。

  最简单的是利用向量的证明:
  \begin{proof}[证明一]
  设$\vv{BD}=\lambda\vv{DC}$,那么
  \begin{eqnarray*}
    \vv{AD} & = & \frac{1}{1+\lambda}\vv{AB}+\frac{\lambda}{1+\lambda}\vv{AC} \\
   & = & \frac{CD}{BC}\vv{AB}+\frac{BD}{BC}\vv{AC} 
  \end{eqnarray*}
  由此
  \begin{eqnarray*}
    |\vv{AD}| & = & \left| \frac{CD}{BC}\vv{AB}+\frac{BD}{BC}\vv{AC} \right| \\
              & < & \left| \frac{CD}{BC}\vv{AB} \right| + \left| \frac{BD}{BC}\vv{AC} \right| \\
              & = & \frac{CD}{BC} | \vv{AB} | + \frac{BD}{BC} |\vv{AC} |
  \end{eqnarray*}
  即得证。
  \end{proof}
  
  \begin{proof}[证明二]
    如图\ref{fig:definite-proportion-inequality},过点$D$分别作边$AC$、$AB$的平行线,分别与$AB$、$AC$相交于点$E$、$F$,有

  \begin{figure}[htbp]
  \centering
\includegraphics{content/plane-geometry/pic/definite-proportion-inequality.pdf}
\caption{}
\label{fig:definite-proportion-inequality}
\end{figure}

\begin{equation*}
  \frac{DC}{BC}AB + \frac{BD}{BC}AC = \frac{AE}{AB} AB + \frac{AF}{AC} AC = AE + AF = AE + ED > AD
\end{equation*}
所以不等式得证。
  \end{proof}

  \begin{proof}[证明三]
  第三个证明就是利用斯特瓦尔特定理,在定理中的等式两边同时除以$BC$,改为如下含比例的等式:
  \begin{eqnarray*}
    AD^2 & = & \frac{CD}{BC} \cdot AB^2 + \frac{BD}{BC} \cdot AC^2 - BD \cdot CD \\
    & = & \frac{CD \cdot BC}{BC^2} \cdot AB^2 + \frac{BD \cdot BC}{BC^2} \cdot AC^2 - BD \cdot CD \\
    & = & \frac{CD \cdot (BD+CD)}{BC^2} \cdot AB^2 + \frac{BD \cdot (BD+CD)}{BC^2} \cdot AC^2 - BD \cdot CD \\
    & = & \frac{CD^2}{BC^2} \cdot AB^2 + \frac{BD^2}{BC^2} \cdot AC^2 + \frac{BD \cdot CD}{BC^2}(AB^2+AC^2-BC^2) \\
    & = & \frac{CD^2}{BC^2} \cdot AB^2 + \frac{BD^2}{BC^2} \cdot AC^2 + \frac{BD \cdot CD}{BC^2} \cdot 2 \cdot AB \cdot AC \cdot \cos{\angle BAC} \\
    & < & \frac{CD^2}{BC^2} \cdot AB^2 + \frac{BD^2}{BC^2} \cdot AC^2 + \frac{BD \cdot CD}{BC^2} \cdot 2 \cdot AB \cdot AC \\
    & = & \left( \frac{CD}{BC}\cdot AB + \frac{BD}{BC} \cdot AC \right)^2
  \end{eqnarray*}
  所以不等式成立。
  \end{proof}

  当点$D$在$BC$边的延长线上时,比如说在$C$这一侧,有什么样的结论呢,这时把点$C$看成在三角形$ABD$的边$BD$上,应用刚证明的不等式,就有
  \begin{equation*}
    BD \cdot AC < CD \cdot AB + BC \cdot AD
  \end{equation*}
  所以
  \begin{equation*}
    AD > -\frac{CD}{BC} AB + \frac{BD}{BC} AC
  \end{equation*}
  这个结果看起来难以记忆,但实际上那两个系数跟点$D$分$\vv{BC}$的比例有关,我们把这结果统一成下面这样:

  \begin{statement}
    在三角形$ABC$中,点$D$是$BC$边所在直线上不与$B$和$C$重合的任一点,设$\vv{BD}=\lambda\vv{DC}$,那么在$\lambda>0$时成立不等式
    \begin{equation}
      \label{eq:definite-proportion-inequality2}
      AD < \frac{1}{1+\lambda} AB + \frac{\lambda}{1+\lambda} AC
    \end{equation}
    在$\lambda<0$时不等式反向。
  \end{statement}
\end{example}

\begin{example}
  在这个例子中,我们利用斯特瓦尔特定理来建立三角形中的中线长公式、角平分线长公式、高线长公式。

  三角形$ABC$的$BC$边上的中线长是
  \begin{equation}
    \label{eq:midline-length-triangle}
    l_{a}^2 = \frac{1}{4}(2b^2+2c^2-a^2)
  \end{equation}
  由此可以继续推得一个不等式$l_a < \frac{1}{2}(b+c)$。

  利用内角平分线定理可求得角平分线分对边的比例,于是可以得出角$A$的平分线长度
  \begin{equation}
    \label{eq:bisec-angle-length-triangle}
    \omega_a^2 = bc \left( 1 - \left( \frac{a}{b+c} \right)^2 \right)
  \end{equation}
  由此可得不等式$\omega < \sqrt{bc}$。

  用斯特瓦尔特定理求高线长比较麻烦,倒是直接使用余弦定理方便:
  \begin{eqnarray*}
    h_a & = & b \sin{C} \\
       & = & b \sqrt{1-\cos^2{C}} \\
       & = & b \sqrt{1-\left( \frac{a^2+b^2-c^2}{2ab} \right)^2} \\
    & = & \frac{1}{a} \sqrt{a^2b^2-\frac{(a^2+b^2-c^2)^2}{4}} \\
    & = & \frac{1}{2a} \sqrt{(a+b+c)(a+b-c)(b+c-a)(c+a-b)}
  \end{eqnarray*}
  这就是高线长公式。

  由于$S_{\triangle ABC} = \frac{1}{2} a \cdot h_a$,再记$p=\frac{1}{2}(a+b+c)$为半周长,那么我们实际上得出了著名的 \emph{海伦公式}:
  \begin{equation*}
    S_{\triangle ABC} = \sqrt{p(p-a)(p-b)(p-c)}
  \end{equation*}

%%% Local Variables:
%%% TeX-master: "../../book"
%%% End:
