
\section{几个重要的定理}
\label{sec:some-import-plane-geometry-theorem}

本节讲述几个在平面几何中占有重要地位的定理。

\subsection{张角定理}
\label{sec:spread-angle-theorem}

\begin{theorem}[张角定理]
  在三角形$ABC$中,$D$是$BC$边上任意一点,有下式成立:
  \begin{equation}
    \label{eq:spread-angle-theorem}
    \frac{\sin{\angle{BAC}}}{AD} = \frac{\sin{\angle{CAD}}}{AB} + \frac{\sin{\angle{BAD}}}{AC}
  \end{equation}
\end{theorem}

张角定理的逆也是成立的,即如果平面上四个点满足定理中的等式,就有$B$、$C$、$D$三点共线,这也可以用来处理共线问题。

  \begin{figure}[htbp]
  \centering
\includegraphics{content/plane-geometry/pic/spread-angle-theorem.pdf}
\caption{张角定理}
\label{fig:spread-angle-theorem}
\end{figure}

\begin{proof}[证明一]
  因为$S_{\triangle ABC} = S_{\triangle ABD} + S_{\triangle ACD}$,所以
  \begin{equation*}
    \frac{1}{2}AB \cdot AC \cdot \sin{\angle BAC} =
    \frac{1}{2}AB \cdot AD \cdot \sin{\angle BAD} +
    \frac{1}{2}AC \cdot AD \cdot \sin{\angle CAD} 
  \end{equation*}
  两边同除以$\frac{1}{2}AB \cdot AC \cdot AD$即得证。
\end{proof}

\begin{proof}[证明二]
  采用分析法,要证的等式
  \begin{equation*}
    \Longleftrightarrow AB \cdot AC \sin{\angle BAC} =
      AD \cdot (AB \sin{\angle BAD + AC \sin{\angle CAD}})
  \end{equation*}
  根据正弦定理有$AB \sin{\angle BAD} = BD \sin{\angle ADB}$,$AC \sin{\angle CAD} = CD \sin{\angle ADC}$,所以欲证式又等价于
  \begin{eqnarray*}
    & \Longleftrightarrow & AB \cdot AC \sin{\angle BAC} =
                            AD \cdot (BD \sin{\angle ADB} + CD \sin{\angle ADC}) \\
    & \Longleftrightarrow & AB \cdot AC \sin{\angle BAC} = AD \cdot BC \sin{\angle ADP}
  \end{eqnarray*}
  此式两边都等于$2S_{\triangle ABC}$,所以成立,于是得证。
\end{proof}


\subsection{梅涅劳斯(menelaus)定理}
\label{sec:menelaus-theorem}

\begin{theorem}[梅涅劳斯定理]
  设$X$、$Y$、$Z$是三角形$ABC$的三边$AB$、$BC$、$CA$或其延长线上的点,其中有奇数个点在边的延长线上,则此三点共线的充分必要条件是下式成立:
  \begin{equation}
    \label{eq:menelaus-theorem}
    \frac{AX}{XB} \cdot \frac{BY}{YC} \cdot \frac{CZ}{ZA} = 1
 \end{equation}
\end{theorem}
 
\begin{figure}[htbp]
\centering
\includegraphics{content/plane-geometry/pic/menelaus-theorem.pdf}
\includegraphics{content/plane-geometry/pic/menelaus-theorem2.pdf}
\caption{梅涅劳斯定理}
\label{fig:menelaus-theorem}
\end{figure}
 
%\begin{figure}[htbp]
%\centering
%\includegraphics{content/plane-geometry/pic/menelaus-theorem2.pdf}
%\caption{梅涅劳斯定理:三个点都在边的延长线上的情形}
%\label{fig:menelaus-theorem2}
%\end{figure}

只证明必要性,得证之后利用同一法即可轻松得到充分性的证明。
最精巧的是下面这个辅助线的证明
\begin{proof}[证明一]
  过$C$作$XYZ$的平行线,与边$AB$相交于点$P$,那么有
 
\begin{figure}[htbp]
\centering
\includegraphics{content/plane-geometry/pic/menelaus-theorem-proof.pdf}
\includegraphics{content/plane-geometry/pic/menelaus-theorem-proof2.pdf}
\caption{梅涅劳斯定理的证明}
\label{fig:menelaus-theorem-proof}
\end{figure}
 
%\begin{figure}[htbp]
%\centering
%\includegraphics{content/plane-geometry/pic/menelaus-theorem-proof2.pdf}
%\caption{梅涅劳斯定理的证明}
%\label{fig:menelaus-theorem-proof2}
%\end{figure}

  \begin{equation*}
    \frac{AX}{XB} \cdot \frac{BY}{YC} \cdot \frac{CZ}{ZA} =
    \frac{AX}{XB} \cdot \frac{BX}{XP} \cdot \frac{PX}{XA} = 1
  \end{equation*}
所以必要性成立,充分性利用同一法即可得证,略去。
\end{proof}

面积方法仍然是行之有效的手段:
\begin{proof}[证明二]
  \begin{equation*}
    \frac{AX}{XB} \cdot \frac{BY}{YC} \cdot \frac{CZ}{ZA} =
    \frac{S_{\triangle AYX}}{S_{\triangle BYX}} \cdot
    \frac{S_{\triangle BYX}}{S_{\triangle CYX}} \cdot
    \frac{S_{\triangle CYX}}{S_{\triangle AYX}} = 1
  \end{equation*}
\end{proof}

再给一个向量方法的证明:
\begin{proof}[证明三]
  记$\vv{AX}=\lambda_1\vv{XB}$, $\vv{BY}=\lambda_2\vv{YC}$, $\vv{CZ}=\lambda_3\vv{ZA}$,则有
  \begin{equation*}
    \vv{XZ} = \vv{AZ}-\vv{AX}=\frac{1}{1+\lambda_3}\vv{AC}-\frac{\lambda_1}{1+\lambda_1}\vv{AB}
  \end{equation*}
  同时
  \begin{eqnarray*}
    \vv{XY} &=& \vv{AY} - \vv{AX} = \left( \frac{1}{1+\lambda_2}\vv{AB}+\frac{\lambda_2}{1+\lambda_2}\vv{AC} \right) - \frac{\lambda_1}{1+\lambda_1}\vv{AB} \\
    &=& \left( \frac{1}{1+\lambda_2}-\frac{\lambda_1}{1+\lambda_1} \right)\vv{AB}+\frac{\lambda_2}{1+\lambda_2}\vv{AC} 
  \end{eqnarray*}
这两个向量共线的充分必要条件是上面这两组系数对应成比例,经过计算即为等式$\lambda_1\lambda_2\lambda_3=-1$,因为是有奇数个点在边的延长线上,所以它等价于定理中的等式。
\end{proof}

从向量形式中我们得到了梅涅劳斯定理的向量表达: $\lambda_1\lambda_2\lambda_3=-1$,而且向量形式的证明,充分性与必要性是统一的。

\begin{example}
  \label{example:external-common-tangent-for-3-circle}
  考虑三个半径两两不同的圆,有三组外公切线,每两个圆的一组外公切线都有一个交点,这样的交点有三个,今将说明这三点共线\footnote{这个例子及证法来自于参考文献\cite{kuing-problem-book}}。
 
\begin{figure}[htbp]
\centering
\includegraphics{content/plane-geometry/pic/external-common-tangent-for-3-circle.pdf}
\caption{三个半径两两不同的圆的三组外公切线交点共线}
\label{fig:external-common-tangent-for-3-circle}
\end{figure}

如图\ref{fig:external-common-tangent-for-3-circle},易知$P$、$Q$、$R$三点在三个圆心形成的三角形的三条边的延长线上,而$\frac{PA}{PB}=\frac{r_A}{r_B}$,仿此有另外两式,三式相乘即知这满足梅涅劳斯定理中的等式,所以这三点共线。

这个结论还有另外一个解释,将三个圆看成三个球,那么每两个球有一个公切圆锥,这三个公切圆锥有一个公切面,那么这三个圆锥的顶点都在这个公切面上,同时它们也在这三个球心所在平面上,所以它们都在这两个平面的交线上。
  
\end{example}

\subsection{塞瓦(cevian)定理}
\label{sec:cevian-theorem}

\begin{theorem}[塞瓦定理]
  在三角形$ABC$中,点$X$、$Y$、$Z$分别是三边$AB$、$BC$、$CA$或其延长线上的点,其中有偶数个点在边的延长线上,则三条线段$AY$、$BZ$、$CX$交于一点$P$的充分必要条件是
  \begin{equation}
    \label{eq:cevian-theorem}
    \frac{AX}{XB} \cdot \frac{BY}{YC} \cdot \frac{CZ}{ZA} = 1
  \end{equation}
\end{theorem}
 
\begin{figure}[htbp]
\centering
\includegraphics{content/plane-geometry/pic/cevian-theorem.pdf}
\includegraphics{content/plane-geometry/pic/cevian-theorem2.pdf}
\caption{塞瓦定理}
\label{fig:cevian-theorem}
\end{figure}

同样只证明必要性,第一个证明是直接使用了梅涅劳斯定理:
\begin{proof}[证明一]
  对三角形$ABY$和截线$XPC$使用梅涅劳斯定理得
  \begin{equation*}
    \frac{AX}{XB} \cdot \frac{BC}{CY} \cdot \frac{YP}{PA} = 1
  \end{equation*}
  再对三角形$ACY$和截线$BPZ$使用梅涅劳斯定理得
  \begin{equation*}
    \frac{YB}{BC} \cdot \frac{CZ}{ZA} \cdot \frac{AP}{PY} = 1
  \end{equation*}
  两式相乘即得塞瓦定理。
\end{proof}

面积方法仍然是最直接的:
\begin{proof}[证明二]
  \begin{equation*}
    \frac{AX}{XB} \cdot \frac{BY}{YC} \cdot \frac{CZ}{ZA} =
    \frac{S_{\triangle CPA}}{S_{\triangle CPB}} \cdot
    \frac{S_{\triangle APB}}{S_{\triangle APC}} \cdot
    \frac{S_{\triangle BPC}}{S_{\triangle BPA}} = 1
  \end{equation*}
\end{proof}

同样有向量方法
\begin{proof}[证明三]
  仍然记$\vv{AX}=\lambda_1\vv{XB}$,$\vv{BY}=\lambda_2\vv{YC}$,$\vv{CZ}=\lambda_3\vv{ZA}$,有
  \begin{equation*}
    \vv{AX}=\frac{\lambda_1}{1+\lambda_{1}}\vv{AB}, \ \vv{AZ}=\frac{1}{1+\lambda_3}\vv{AC}
  \end{equation*}
  因为点$P$是$BZ$与$CX$的交点,所以存在两个实数$s$和$t$,使得
  \begin{eqnarray*}
    \vv{AP} &=& (1-s)\vv{AB}+s\vv{AZ} = (1-s)\vv{AB}+\frac{s}{1+\lambda_3}\vv{AC} \\
    \vv{AP} &=& (1-t)\vv{AC}+t\vv{AX} =  \frac{t\lambda_1}{1+\lambda_{1}}\vv{AB}+(1-t)\vv{AC}
  \end{eqnarray*}
  于是得方程组
  \begin{eqnarray}
    1-s &=&  \frac{t\lambda_1}{1+\lambda_{1}} \\
    \frac{s}{1+\lambda_3} &=& 1-t
  \end{eqnarray}
  解之得
  \begin{equation*}
    t = 1- \frac{1}{(1+\lambda_1)(1+\lambda_3)-\lambda_1}
  \end{equation*}
  所以
  \begin{equation*}
    \vv{AP}
    = \left( 1- \frac{1}{(1+\lambda_1)(1+\lambda_3)-\lambda_1} \right) \cdot \frac{\lambda_1}{1+\lambda_{1}} \vv{AB}+\frac{1}{(1+\lambda_1)(1+\lambda_3)-\lambda_1}\vv{AC}
  \end{equation*}
  另一方面有
  \begin{equation*}
    \vv{AY}=\frac{1}{1+\lambda_2}\vv{AB}+\frac{\lambda_2}{1+\lambda_2}\vv{AC}
  \end{equation*}
  于是$\vv{AP}$与$\vv{AY}$共线的充分必要条件是两组系数成比例,经简单计算知上面两个式子中的系数比分别为$\lambda_1\lambda_3 : 1$和$1 : \lambda_2$,所以得到$\lambda_1\lambda_2\lambda_3=1$,即得证。
\end{proof}

\begin{example}[梅涅劳斯定理与塞瓦定理的统一]
根据这证明过程,实际上梅涅劳斯定理和塞瓦定理可以统一起来:假定$X$、$Y$、$Z$是三角形$ABC$三边$AB$、$BC$、$CA$或它们的延长线上的点,有分比$\vv{AX}=\lambda_1\vv{XB}$,$\vv{BY}=\lambda_2\vv{YC}$,$\vv{CZ}=\lambda_3\vv{ZA}$,那么这三点共线的充分必要条件是$\lambda_1\lambda_2\lambda_3=-1$(梅涅劳斯定理),而$AY$、$BZ$、$CX$三线共点的充分必要条件是$\lambda_1\lambda_2\lambda_3=1$(塞瓦定理)。
\end{example}

\begin{example}[三角形的几个心]
 利用塞瓦定理,我们知道三角形的三条中线是交于同一点的,该点称为三角形的 \emph{重心}。

 同样,利用塞瓦定理和三角形内角平分线定理,还可以知道三角形的三条内角平分线也是交于同一点的,这点称为三角形的 \emph{内心},它是三角形内切圆圆心。

 对于三角形的一个角的内角平分线和另外两个角的外角平分线,根据三角形的内外角平分线定理,可得这三线也是相交于同一点的,这一点称为三角形的 \emph{旁心},它是旁切圆的圆心。

 对于三条高线,假定$BC$边上的高线垂足是$D$,则有$\frac{BD}{CD}=\frac{c \cos{B}}{b \cos{C}}$,仿此有另外两式,三式相乘并根据塞瓦定理即知三条高也相交于同一点,称该点为三角形的 \emph{垂心}。

这四个心再加上三角形的\emph{外心}(三边的中垂线交点),合称为三角形的五心。 
\end{example}

\begin{example}
  仿照例\ref{example:external-common-tangent-for-3-circle},把那里的外公切线改为内公切线,那么图\ref{fig:internal-common-tangent-for-3-circle}中的三线共点。
 
\begin{figure}[htbp]
\centering
\includegraphics{content/plane-geometry/pic/internal-common-tangent-for-3-circle.pdf}
\caption{三个半径两两不同的圆的三组内公切线交点形成的塞瓦三线}
\label{fig:internal-common-tangent-for-3-circle}
\end{figure}

因为$\frac{AP}{PB}=\frac{r_A}{r_B}$,同样有另外两式,三式相乘后由塞瓦定理知三线共点。
\end{example}

%%% Local Variables:
%%% TeX-master: "../../book"
%%% End:
