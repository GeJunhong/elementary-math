
\section{几个重要的定理}
\label{sec:some-import-plane-geometry-theorem}

本节讲述几个在平面几何中占有重要地位的定理。

\subsection{张角定理}
\label{sec:spread-angle-theorem}

\begin{theorem}[张角定理]
  在三角形$ABC$中,$D$是$BC$边上任意一点,有下式成立:
  \begin{equation}
    \label{eq:spread-angle-theorem}
    \frac{\sin{\angle{BAC}}}{AD} = \frac{\sin{\angle{CAD}}}{AB} + \frac{\sin{\angle{BAD}}}{AC}
  \end{equation}

  \begin{figure}[htbp]
  \centering
\includegraphics{content/plane-geometry/pic/spread-angle-theorem.pdf}
\caption{张角定理}
\label{fig:spread-angle-theorem}
\end{figure}

\end{theorem}

\begin{proof}[证明]
  因为$S_{\triangle ABC} = S_{\triangle ABD} + S_{\triangle CBD}$,所以
  \begin{equation*}
    \frac{1}{2}AB \cdot AC \cdot \sin{\angle BAC} =
    \frac{1}{2}AB \cdot AD \cdot \sin{\angle BAD} +
    \frac{1}{2}AC \cdot AD \cdot \sin{\angle CAD} 
  \end{equation*}
  两边同除以$\frac{1}{2}AB \cdot AC \cdot AD$即得证。
\end{proof}

张角定理的逆也是成立的,即如果平面上四个点满足定理中的等式,就有$B$、$C$、$D$三点共线,这也可以用来处理共线问题。

\subsection{梅涅劳斯(menelaus)定理}
\label{sec:menelaus-theorem}

\begin{theorem}[梅涅劳斯定理]
  设$X$、$Y$、$Z$是三角形$ABC$的三边$BC$、$CA$、$AB$或其延长线上的点,其中有奇数个点在边的延长线上,则此三点共线的充分必要条件是下式成立:
  \begin{equation}
    \label{eq:menelaus-theorem}
    \frac{AZ}{ZB} \cdot \frac{BX}{XC} \cdot \frac{CY}{YA} = 1
 \end{equation}
 
  \begin{figure}[htbp]
  \centering
\includegraphics{content/plane-geometry/pic/menelaus-theorem.pdf}
\caption{梅涅劳斯定理}
\label{fig:menelaus-theorem}
\end{figure}

\end{theorem}

最精巧的是下面这个辅助线的证明
\begin{proof}[证明]
  先证必要性,只就有一个点在边的延长线上进行证明,三个点的情况是类似的,如图\ref{fig:menelaus-theorem-proof},点$X$在$BC$边延长线上,过$C$作$XYZ$的平行线,与边$AB$相交于点$P$,那么有
 
  \begin{figure}[htbp]
  \centering
\includegraphics{content/plane-geometry/pic/menelaus-theorem-proof.pdf}
\caption{梅涅劳斯定理的证明}
\label{fig:menelaus-theorem-proof}
\end{figure}

  \begin{equation*}
    \frac{AZ}{ZB} \cdot \frac{BX}{XC} \cdot \frac{CY}{YA} =
    \frac{AZ}{ZB} \cdot \frac{BZ}{ZP} \cdot \frac{PZ}{ZA} = 1
  \end{equation*}
所以必要性成立,充分性利用同一法即可得证,略去
\end{proof}

%%% Local Variables:
%%% TeX-master: "../../book"
%%% End:
