
\section{题集}

\begin{exercise}
\label{exercise:2016-06-24-1}
  如图,在三角形$ABC$中,$AO$是$BC$边上的高,$D$是$AO$的中点,点$E$是其内切圆$\odot I$与$BC$边的切点,连接$ED$并延长与内切圆相交于另一点$F$,过点$C$并与$BC$垂直的直线与$BI$的延长线相交于$G$,求证:(1) $EF \perp FG$,(2)$EF$平分角$BFC$。
\end{exercise}

\begin{figure}[htbp]
  \centering
\includegraphics{content/plane-geometry/pic/2016-06-24-1.pdf}
\caption{}
\label{fig:2016-06-24-1}
\end{figure}

\begin{exercise}
 (姚佳斌供题)如\autoref{fig:exercise-ahk-coline},在$\triangle ABC$中,点$H$是垂心,$AD$是$BC$边上的中线,过$H$作$AD$的垂线,分别与$AC$、$AB$边相交于$E$、$F$,连结$BE$、$CF$相交于点$K$,求证 $A$、$H$、$K$三点共线。
\end{exercise}

\begin{figure}[htbp]
  \centering
\includegraphics{content/plane-geometry/pic/exercise-ahk-coline.pdf}
\caption{}
\label{fig:exercise-ahk-coline}
\end{figure}

\begin{proof}[证明]
  只要证明$BE$、$CF$、$AH$三线共点就可以了,设$BC$边上的高线垂足是$M$,由塞瓦定理,只需证
  \[ \frac{AF}{FB} \cdot \frac{BM}{MC} \cdot \frac{CE}{EA} = 1 \]
  延长$FE$与$BC$边延长线相交于点$P$,再设$EF$与$AD$交点为$L$,则对$\triangle ABC$和截线$FEP$应用梅涅劳斯定理得
  \[ \frac{AF}{FB} \cdot \frac{BP}{PC} \cdot \frac{CE}{EA} = 1 \]
  所以只需要证明下式就可以了:
  \[ \frac{BM}{MC} = \frac{BP}{PC} \]
  现在就来证明这一点,易知$L$、$D$、$M$、$H$四点共圆,所以
  \[ AL \cdot AD = AH \cdot AM \]
  又易知$A$、$L$、$M$、$P$四点共圆,所以
  \[ DL \cdot DA = DM \cdot DP \]
  这两式相加得
  \[ AD^2 = AH \cdot AM + DM \cdot DP \]
  以下就来从这式中求出$DP$,易知
  \begin{eqnarray*}
    AD^2 & = & \frac{1}{2}(b^2+c^2)-\frac{1}{4}c^2 \\
    AH & = & \frac{\cos{A}}{\sin{B}}b \\
    AM & = & c \sin{B} \\
    DM & = & \frac{1}{2}a-b \cos{C}
  \end{eqnarray*}
  全部代入前式中即得
  \[ DP = \frac{a^2}{2(a-2b\cos{C})} \]
  进而(结合余弦定理)
  \begin{eqnarray*}
  BP & = & DP+\frac{a}{2} = \frac{a^2+c^2-b^2}{2(a-2b\cos{C})}  \\
  PC & = & DP-\frac{a}{2} = \frac{a^2+b^2-c^2}{2(a-2b\cos{C})} 
  \end{eqnarray*}
  于是
  \[ \frac{BP}{PC} = \frac{a^2+c^2-b^2}{a^2+b^2-c^2} = \frac{2ac\cos{B}}{2ab\cos{C}} = \frac{c\cos{B}}{b\cos{C}} = \frac{BM}{MC} \]
  得证。
\end{proof}


\begin{exercise}
  \label{ec:2017-02-09-01}
  如图\ref{fig:2017-02-09-01-01},分别以三角形$ABC$的边$AB$和$AC$向形外作正方形,两个正方形的中心分别记为$O_{1}$和$O_{2}$,$X$是$BC$边中点,其余各点的意义如图所示,求证:
  \begin{enumerate}
  \item 三角形$O_{1}XO_{2}$是等腰直角三角形.
  \item $ST \parallel PQ$.
  \end{enumerate}
\end{exercise}

\begin{figure}[htbp]
  \centering
\includegraphics{content/plane-geometry/pic/ec-2017-02-09-01.pdf}
\caption{}
\label{fig:2017-02-09-01-01}
\end{figure}
  
\begin{proof}[证明]
  易见$\triangle ACE \cong \triangle AGB$,而$O_{1}X \parallel EC$并且$|O_{1}X|=\frac{1}{2}|EC|$,同样$O_2X \parallel BG$并且$|O_2X|=\frac{1}{2}|BG|$,但是$EC=BG$,并且$EC \perp BG$,所以$O_1X=O_2X$并且$O_1X \perp O_2X$,所以$\triangle O_{1}XO_{2}$是等腰直角三角形.

  第二问的最简单证法是设$DE$延长线与$CA$延长线交于$U$,$FG$延长线与$BA$延长线交于$V$,则有 $\triangle AEU \sim \triangle AGV$,于是
  \begin{equation*}
    \frac{AS}{SP}=\frac{UE}{ED}=\frac{UE}{EA}=\frac{VG}{GA}=\frac{VG}{GF}=\frac{AT}{TQ}
  \end{equation*}
  所以$ST \parallel PQ$.

\begin{figure}[htbp]
  \centering
\includegraphics{content/plane-geometry/pic/2017-02-09-01-02.pdf}
\caption{}
\label{fig:2017-02-09-01-02}
\end{figure}

  如果注意到图中包含着两个同样的基本图形,则可以有如下的证明,先提一个引理:如图\ref{fig:2017-02-09-01-02},设点$P$为正方形$ABCD$的边$AB$外侧一点,$PC$与$PD$分别与边$AB$相交于$S$和$T$,记$\angle PAB=\alpha, \angle PBA=\beta$,则有
  \begin{equation*}
    AT : TS : SB = \cos{\alpha}\sin{\beta} : \sin{\alpha}\sin{\beta} : \sin{\alpha}\cos{\beta}
  \end{equation*}
  利用此引理知,$AT:TS$仅与角$\alpha$有关,因此在原题中立刻便有$\frac{AS}{SP}=\frac{AT}{TQ}$,因此两线平行,而这个引理的证明也很简单,记点$P$在边$AB$上的投影为$R$,并假定正方形边长为1,并记$PB=a, PA=b$,根据正弦定理可得 
  \begin{equation*}
    \frac{a}{\sin{\alpha}}=\frac{1}{\sin{\alpha+\beta}}=\frac{b}{\sin{\beta}}
  \end{equation*}
  于是
  \begin{equation*}
    a=\frac{\sin{\alpha}}{\sin{\alpha+\beta}}, b=\frac{\sin{\beta}}{\sin{\alpha+\beta}}
  \end{equation*}
  根据几何关系有
  \begin{equation*}
    AR=b\cos{\alpha}=\frac{\cos{\alpha}\sin{\beta}}{\sin{\alpha+\beta}}, PR=b\sin{\alpha}=\frac{\sin{\beta}}{\sin{\alpha+\beta}}
  \end{equation*}
  而由$\frac{TR}{AT}=\frac{PR}{AD}$得
  \begin{eqnarray*}
    AT &=& \frac{AT}{AT+TR}AR=\frac{1}{1+\frac{TR}{AT}}AR 
    = \frac{1}{1+\frac{\sin{\alpha}\sin{\beta}}{\sin{\alpha+\beta}}} \cdot \frac{\cos{\alpha}\sin{\beta}}{\sin{\alpha+\beta}} \\
    &=& \frac{\cos{\alpha}\sin{\beta}}{\sin{\alpha}\sin{\beta}+\sin{\alpha+\beta}}
  \end{eqnarray*}
  根据对称性,互换$\alpha$和$\beta$可得
  \begin{equation*}
    BS = \frac{\sin{\alpha}\cos{\beta}}{\sin{\alpha}\sin{\beta}+\sin{\alpha+\beta}}
  \end{equation*}
  于是
  \begin{equation*}
    TS = \frac{\sin{\alpha}\sin{\beta}}{\sin{\alpha}\sin{\beta}+\sin{\alpha+\beta}}
  \end{equation*}
  于是引理得证。
\end{proof}

\begin{exercise}\footnote{2017年北京大学中学生数学奖个人能力挑战赛试题.}
  如\autoref{fig:exercise-pi-perp-mn},三角形$ABC$的内切圆$\odot I$与三边分别相切于$D$、$E$、$F$,线段$BE$与线段$CF$相交于点$P$,直线$DE$与直线$BA$相交于点$M$,直线$DF$与直线$CA$相交于点$N$,求证: $PI \perp MN$.
 
\begin{figure}[htbp]
  \centering
\includegraphics{content/plane-geometry/pic/exercise-pi-perp-mn.pdf}
\caption{}
\label{fig:exercise-pi-perp-mn}
\end{figure}

\end{exercise}

\begin{exercise}
 若$\triangle ABC$中成立$b+c=2a$,求证$OI \perp AI$,其中$O$和$I$分别表外心和内心。
\end{exercise}

\begin{proof}[证明]
  只需证明$AO^2=OI^2+AI^2$,记外接圆半径和内切圆半径分别为$R$和$r$,由几何关系及欧拉公式有
  \[ AO=R, \  OI^2=R^2-2Rr, \  AI=\frac{r}{\sin{\frac{A}{2}}} \]
  以上三式代入勾股式后知只需证
  \[ \frac{r}{R}=2\sin^2{\frac{A}{2}} \]
  而
  \[ R=\frac{a}{2\sin{A}} \]
  和
  \[ r=\frac{1}{2}(b+c-a)\tan{\frac{A}{2}}=\frac{1}{2}a\tan{\frac{A}{2}} \]
  由此二式便知要证的等式成立。
\end{proof}

%%% Local Variables:
%%% mode: latex
%%% TeX-master: "../../book"
%%% End:
