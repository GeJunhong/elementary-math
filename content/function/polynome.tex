
\section{多项式}
\label{sec:polynome}

这一节是\cite{advanced-algebra}的学习笔记,讨论了多项式的整除理论与及因式分解相关内容。

\subsection{数域}
\label{sec:number-field}

域本来是抽象代数中的一个概念,但早点接触它还是有好处的。

\begin{definition}
  如果一个数集包含了0和1,并且对集合中任意两数(可以相同)进行加减乘除运算所得的结果都仍然在这集合中(称为对这四种运算具有封闭性),则称该数集为一个 \emph{数域}.
\end{definition}

易知任何数域都包含有理数集作为它的一个子集,常见的有理数集、实数集、复数集都是数域,但数域是一个更宽泛的概念,例如下面这个例子。

\begin{example}
  设$\pi$是一个无理数,则定义数集 $A_{\pi} = \{x|x=a+b\pi,a,b\in Q\}$,这里$Q$是有理数集,易证这是一个数域,其中除了有理数外,还包含一些无理数,这些无理数都跟一个确定的无理数$\pi$有关,这个数集可以称为是由无理数$\pi$生成的 \emph{最小数域},这类数域在构造某些特例时是有用的。
\end{example}

\subsection{一元多项式}
\label{sec:polynome-with-one-variable}



\subsection{整除的概念与带余除法}
\label{sec:polynome-integer-division-and-devision-with-remainder}

\begin{theorem}[带余除法]
  对数域$P$上的任意两个非零多项式$f(x)$和$g(x)$,存在数域$P$上的另外两个多项式$q(x)$和$r(x)$,其中$r(x)$次数低于$g(x)$或者是零多项式,使得下式成立
  \[ f(x) = q(x) g(x) + r(x) \]
\end{theorem}

\begin{proof}[证明]
  如果$f(x)$的次数低于$g(x)$,则取$u(x)=0$, $r(x)=f(x)$就可以了,以下证明$f(x)$比$g(x)$次数高的情况,设
  \[ f(x)=\sum_{i=0}^na_ix^i, \   g(x)=\sum_{i=0}^nb_ix^i \]
  其中$n>m>0$,则首先用
  \[ u_0(x)=\frac{a_n}{b_m}x^{n-m} \]
  与$g(x)$相乘得出一个$n$次多项式,于是这个多项式与$f(x)$的差就是一个次数低于$n$的多项式,即
  \[ f(x) = u_0(x) g(x) + f_1(x) \]
  如果$f_1(x)$的次数仍然高于或者等于$g(x)$,则我们再对$f_1(x)$施以同样的手法可得出第二个等式
  \[ f_1(x) = u_1(x)g(x) + f_2(x) \]
  依次下去,必定能够在有限步之内(最多$n-m+1$步)使得某个$f_r(x)$的次数低于$g(x)$或者成为零多项式,这时便有
  \[ f(x) = g(x) \sum_{i=0}^{r-1}u_i(x) + f_r(x) \]
  于是取
  \[ u(x)=\sum_{i=0}^{r-1}u_i(x), \  r(x)=f_r(x) \]
  就可以满足定理要求。
\end{proof}

这个证明过程实际上就是多项式除法的运算过程。

\subsection{最大公因式与辗转相除法}
\label{sec:greatest-common-divisor-and-euclidean-division}

\subsection{因式分解定理}
\label{sec:factoring-theorem}

(包含复系数与实系数多项式的因式分解)

\subsection{重因式}
\label{sec:mulitple-factor}

\subsection{多项式函数}
\label{sec:polynome-function}

在此讨论一下有理系数多项式的有理根的问题,因为有理系数多项式方程总可以化为一个整系数多项式方程,所以只要讨论整系数多项式的根就行了,这时我们有以下定理
\begin{theorem}
  整系数$n$次多项式
  \[ a_nx^n+a_{n-1}x^{n-1}+\cdots+a_1x+a_0 \]
  如果有一个有理根$r/s$($r$、$s$互素),则$r \mid a_0$,$s \mid a_n$,在最高次项系数$a_n=1$的特殊情况下,它的有理数都只能是整数根,而且这些根都是常数项$a_0$的因数。
\end{theorem}

\begin{proof}[证明]
  由条件得
  \[ a_n \left( \frac{r}{s} \right)^{n} + a_{n-1}\left( \frac{r}{s} \right)^{n-1} + \cdots + a_1 \frac{r}{s} + a_0 = 0 \]
  整理得
  \[ a_nr^n + a_{n-1}r^{n-1}s + \cdots + a_1rs^{n-1} + a_0s^n = 0 \]
  左边除最后一项外都能被$r$整除,所以$r \mid a_0s^{n-1}$,而$r$与$s$互素,所以$r \mid a_0$,同样,左边除第一项外都能被$s$整除,所以$s \mid a_nr^n$,从而$s \mid a_n$.
\end{proof}

\subsection{插值多项式}
\label{sec:interpolation-polynome}






%%% Local Variables:
%%% mode: latex
%%% TeX-master: "../../book"
%%% End:
