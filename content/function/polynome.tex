
\section{多项式}
\label{sec:polynome}

这一节是\cite{advanced-algebra}的学习笔记,讨论了多项式的整除理论与及因式分解相关内容。

\subsection{数域}
\label{sec:number-field}

域本来是抽象代数中的一个概念,但早点接触它还是有好处的。

\begin{definition}
  如果一个数集包含了0和1,并且对集合中任意两数(可以相同)进行加减乘除运算所得的结果都仍然在这集合中(称为对这四种运算具有封闭性),则称该数集为一个 \emph{数域}.
\end{definition}

易知任何数域都包含有理数集作为它的一个子集,常见的有理数集、实数集、复数集都是数域,但数域是一个更宽泛的概念,例如下面这个例子。

\begin{example}
  设$\pi$是一个无理数,则定义数集 $A_{\pi} = \{x|x=a+b\pi,a,b\in Q\}$,这里$Q$是有理数集,易证这是一个数域,其中除了有理数外,还包含一些无理数,这些无理数都跟一个确定的无理数$\pi$有关,这个数集可以称为是由无理数$\pi$生成的 \emph{最小数域},这类数域在构造某些特例时是有用的。
\end{example}

\subsection{一元多项式}
\label{sec:polynome-with-one-variable}



\subsection{整除的概念与带余除法}
\label{sec:polynome-integer-division-and-devision-with-remainder}

\begin{theorem}[带余除法]
  对数域$P$上的任意两个非零多项式$f(x)$和$g(x)$,存在数域$P$上的另外两个多项式$q(x)$和$r(x)$,其中$r(x)$次数低于$g(x)$或者是零多项式,使得下式成立
  \[ f(x) = q(x) g(x) + r(x) \]
  并且$q(x)$及$r(x)$是唯一的。
\end{theorem}

\begin{proof}[证明]
  先证明存在性。
  
  如果$f(x)$的次数低于$g(x)$,则取$u(x)=0$, $r(x)=f(x)$就可以了,以下证明$f(x)$比$g(x)$次数高的情况,设
  \[ f(x)=\sum_{i=0}^na_ix^i, \   g(x)=\sum_{i=0}^nb_ix^i \]
  其中$n>m>0$,则首先用
  \[ u_0(x)=\frac{a_n}{b_m}x^{n-m} \]
  与$g(x)$相乘得出一个$n$次多项式,于是这个多项式与$f(x)$的差就是一个次数低于$n$的多项式,即
  \[ f(x) = u_0(x) g(x) + f_1(x) \]
  如果$f_1(x)$的次数仍然高于或者等于$g(x)$,则我们再对$f_1(x)$施以同样的手法可得出第二个等式
  \[ f_1(x) = u_1(x)g(x) + f_2(x) \]
  依次下去,必定能够在有限步之内(最多$n-m+1$步)使得某个$f_r(x)$的次数低于$g(x)$或者成为零多项式,这时便有
  \[ f(x) = g(x) \sum_{i=0}^{r-1}u_i(x) + f_r(x) \]
  于是取
  \[ q(x)=\sum_{i=0}^{r-1}u_i(x), \  r(x)=f_r(x) \]
  就可以满足定理要求。

  再证明唯一性,设有两组符合定理中等式的$q(x)$和$r(x)$,即有
  \[ f(x) = q_1(x)g(x)+r_1(x) = q_2(x)g(x)+r_2(x) \]
  于是便有
  \[ (q_1(x)-q_2(x)) g(x) = r_1(x)-r_2(x) \]
  上式右端的次数低于$g(x)$的次数,因此左边必须有$q_1(x)=q_2(x)$,这时也就必然有$r_1(x)=r_2(x)$,唯一性得证。
\end{proof}

这个证明过程实际上就是多项式除法的运算过程,定理中的$q(x)$称为$f(x)$除以$g(x)$的\emph{商},而$r(x)$则为\emph{余式}.

\begin{definition}
  对于多项式$f(x)$和$g(x)$,如果存在多项式$h(x)$,使得$f(x)=g(x)h(x)$,则称$g(x)$能够\emph{整除}$f(x)$,记为$g(x) \mid f(x)$,此时称$g(x)$是$f(x)$的\emph{因式},而$f(x)$则称为$g(x)$的\emph{倍式}.
\end{definition}

多项式整除有以下性质.
\begin{property}
  如果多项式$f(x)$和$g(x)$能够互相整除,则它俩只相差一个常数因子,即$f(x)=cg(x)$,$c$为常数。
\end{property}

\begin{proof}[证明]
  因为存在多项式$h_1(x)$和$h_2(x)$,使得$f(x)=g(x)h_1(x)$和$g(x)=f(x)h_2(x)$,所以$h_1(x)h_2(x)=1$,于是$h_1(x)$和$h_2(x)$都只能是常数,得证。
\end{proof}

还有以下两个性质,证明略。
\begin{property}
  如果$f(x) \mid g(x)$且$g(x) \mid h(x)$,则$f(x) \mid h(x)$.
\end{property}

\begin{property}
  如果$f(x) \mid g(x)(i=1,2,\ldots,n)$,则有
  \[ f(x) \mid \sum_{i=1}^n u_i(x)g_i(x) \]
  式中$u_i(x)(i=1,2,\ldots,n)$是任意多项式。
\end{property}

\begin{theorem}
  多项式$g(x)$能够整除多项式$f(x)$的充分必要条件是,在带余除法等式
  \[ f(x)=q(x)g(x)+r(x) \]
  中的余式$r(x)$是零多项式。
\end{theorem}

\begin{proof}[证明]
  充分性显然,只证必要性,由$g(x) \mid f(x)$,知$g(x)$能够整除带余除法等式中三个项的其中两个项,因而也能整除另外一项,即$g(x) \mid r(x)$,但$r(x)$的次数低于$g(x)$,所以只能是零多项式,必要性成立。
\end{proof}

\subsection{最大公因式与辗转相除法}
\label{sec:greatest-common-divisor-and-euclidean-division}

\begin{definition}
  如果$d(x) \mid f(x)$且$d(x) \mid g(x)$,则称$d(x)$是$f(x)$与$g(x)$的一个 \emph{公因式}.
\end{definition}

显然常数因子是任何两个多项式的公因式,因而公因式是存在的。

\begin{definition}
  多项式$f(x)$与$g(x)$的诸公因式中,次数最高的那一个称为这两个多项式的\emph{最大公因式}.
\end{definition}

这个定义的问题是,有没有可能同时存在多个次数都最高的公因式(只相差一个常数因子的视为同一个),也就是最大公因式的唯一性问题。

而流行的定义是:

\begin{definition}
  设$d(x)$是$f(x)$与$g(x)$的一个公因式,如果$f(x)$与$g(x)$的所有公因式都是$d(x)$的因式,则称$d(x)$是$f(x)$与$g(x)$的\emph{最大公因式}.
\end{definition}

这个定义也有个问题,在这诸公因式中,次数最高的那个,是否能被别的所有公因式都整除呢,所以无论采用哪一种定义,都有些问题要留待对最大公因式有一定程度的讨论后才能解决,这就是最大公因式的性质定理:

\begin{theorem}[最大公因式性质定理]
  \label{theorem:greatest-factor-polynome-property}
  设$d(x)$是$f(x)$与$g(x)$的最大公因式,则存在多项式$u(x)$和$v(x)$,使得
  \[ d(x) = u(x)f(x) + v(x)g(x) \]
\end{theorem}

为证明这个定理,先提出如下引理
\begin{lemma}
  如果有等式
  \[ f(x)=q(x)g(x)+r(x) \]
  则$f(x)$与$g(x)$的公因式,也必然是$g(x)$与$r(x)$的公因式.
\end{lemma}
由整除性质,引理是显然的。

现在证明\autoref{theorem:greatest-factor-polynome-property}:
\begin{proof}[证明]
  根据带余除法,存在多项式$q_1(x)$及$r_1(x)$(次数低于$g(x)$或者是零多项式),使
  \[ f(x)=q_1(x)g(x)+r_1(x) \]
  如果$r_1(x)$不是零多项式,则再继续拿$g(x)$除以$r_1(x)$,得出
  \[ g(x) = q_2r_1(x)+r_2(x) \]
  如此反复下去,因为每进行一次,$r_i(x)$的次数至少减少一,因此必然在有限步之后,$r_i(x)$成为零次多项式即为常数,再进行一次之后,$r_i(x)$便成为零多项式,把这过程写成等式序列,并记$r_{-1}(x)=f(x)$,$r_0(x)=g(x)$,便是
  \begin{eqnarray*}
    r_{-1}(x) & = & q_1(x)r_0(x)+r_1(x) \\
    r_0(x) & = & q_2(x)r_1(x)+r_2(x) \\
    \cdots \\
    r_{k-1}(x) & = & q_{k+1}(x)r_k(x)+r_{k+1}(x) \\
    \cdots \\
    r_{s-2}(x) & = & q_s(x)r_{s-1}(x) + r_s(x) \\
    r_{s-1}(x) & = & q_{s+1}(x)r_s(x) + 0
  \end{eqnarray*}
  按照上述引理,$f(x)$与$g(x)$的公因式,必然也是$g(x)$与$r_1(x)$的公因式,也就必然是$r_1(x)$和$r_2(x)$的公因式,依次推下去,最后必然也是$r_s(x)$与0的公因式,从而就必然是$r_s(x)$的因式,于是$r_s(x)$本身便是最大公因式(无论按照前面的两种定义的那一种)。

  从上面倒数第二个等式可以看出,$r_s(x)$可以用$r_{s-1}(x)$与$r_{s-2}(x)$表示成各自与某个多项式相乘后再相加的形式,而$r_{s-1}(x)$又可以用$r_{s-2}(x)$和$r_{s-3}(x)$用相同的形式表示出来,依次倒着推回去,最后$r_s(x)$便必定可以用$f(x)$与$g(x)$各自与某个多项式相乘后相加的形式表示出来,当然这也可以用数学归纳法的形式进行严格叙述,略去。
\end{proof}

定理证明过程中的这个反复做带余除法的过程称为 \emph{辗转相除法}。

\subsection{因式分解定理}
\label{sec:factoring-theorem}

(包含复系数与实系数多项式的因式分解)

\subsection{重因式}
\label{sec:mulitple-factor}

\subsection{多项式函数}
\label{sec:polynome-function}

在此讨论一下有理系数多项式的有理根的问题,因为有理系数多项式方程总可以化为一个整系数多项式方程,所以只要讨论整系数多项式的根就行了,这时我们有以下定理
\begin{theorem}
  整系数$n$次多项式
  \[ a_nx^n+a_{n-1}x^{n-1}+\cdots+a_1x+a_0 \]
  如果有一个有理根$r/s$($r$、$s$互素),则$r \mid a_0$,$s \mid a_n$,在最高次项系数$a_n=1$的特殊情况下,它的有理数都只能是整数根,而且这些根都是常数项$a_0$的因数。
\end{theorem}

\begin{proof}[证明]
  由条件得
  \[ a_n \left( \frac{r}{s} \right)^{n} + a_{n-1}\left( \frac{r}{s} \right)^{n-1} + \cdots + a_1 \frac{r}{s} + a_0 = 0 \]
  整理得
  \[ a_nr^n + a_{n-1}r^{n-1}s + \cdots + a_1rs^{n-1} + a_0s^n = 0 \]
  左边除最后一项外都能被$r$整除,所以$r \mid a_0s^{n-1}$,而$r$与$s$互素,所以$r \mid a_0$,同样,左边除第一项外都能被$s$整除,所以$s \mid a_nr^n$,从而$s \mid a_n$.
\end{proof}

\subsection{插值多项式}
\label{sec:interpolation-polynome}






%%% Local Variables:
%%% mode: latex
%%% TeX-master: "../../book"
%%% End:
