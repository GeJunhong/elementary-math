
\section{三角函数}
\label{sec:triangle-function}

\subsection{三倍角公式}
\label{sec:triple-angle-formula}

\subsection{第一类切比雪夫多项式}
\label{sec:first-chebyshev-polynome}

我们知道余弦的二倍角公式和三倍角公式
\begin{eqnarray*}
  \cos{2\theta} & = & 2\cos^2{\theta}-1 \\
  \cos{3\theta} & = & 4\cos^3{\theta}-3\cos{\theta}
\end{eqnarray*}
于是$\cos{3\theta}$和$\cos{4\theta}$分别表成了$\cos{\theta}$的二次多项式和三次多项式,那么对于一般形式的$\cos{n\theta}$,情况又如何呢?

由和差化积公式可得
\[ \cos{(n+1)\theta}+\cos{(n-1)\theta}=2\cos{n\theta}\cos{\theta} \]
所以
\[ \cos{(n+1)\theta}=2\cos{n\theta}\cos{\theta}-\cos{(n-1)\theta} \]
假如$\cos{n\theta}$和$\cos{(n-1)\theta}$分别是$\cos{\theta}$的$n$次多项式和$n-1$次多项式,那么$\cos{(n+1)\theta}$显然也是$\cos{\theta}$的一个次数不超过$n+1$的多项式,而易见最高次系数$a_n$满足递归关系$a_{n+1}=2a_n$,而$a_2=2,a_3=4$,所以$a_n=2^{n-1}$,最高次项系数恒正,从而$\cos{(n+1)\theta}$表成了$\cos{\theta}$的$n+1$次多项式。这结果就是如下的定理:

\begin{theorem}[切比雪夫多项式定理]
  对于任意正整数$n$,$\cos{n\theta}$可表为$\cos{\theta}$的$n$次多项式,这多项式的最高次项系数是$2^{n-1}$,这多项式称为\emph{第一类切比雪夫(Chebyshev)多项式}。
\end{theorem}

第一类切比雪零头多项式以$T_n(x)$标记,按定义有$\cos{n\theta}=T_n(\cos{\theta})$。

容易得$T_0(x)=1,T_1(x)=x$并有如下递推式
\begin{equation}
  \label{eq:first-chebyshev-polynome-recursion}
  T_{n+1}(x)=2xT_n(x)-T_{n-1}(x)
\end{equation}
当然也可以用这个递归式来定义第一类切比雪夫多项式,只是少了三角函数的背景,成了为了定义而定义了。

前几个多项是如下的表:
\begin{eqnarray*}
  T_0(x) & = & 1 \\
  T_1(x) & = & x \\
  T_2(x) & = & 2x^2-1 \\
  T_3(x) & = & 4x^3-3x \\
  T_4(x) & = & 8x^4-4x^2+1 \\
  T_5(x) & = & 16x^5-20x^3+5x \\
  T_6(x) & = & 32x^6-48x^4+18x^2-1
\end{eqnarray*}

在讨论这些多项式的性质之前,我们再从另外一个角度来引导出第一类切比雪夫多项。由复数的乘幂公式,复数$z=r(\cos{\theta}+i\sin{\theta})$的乘方是
\begin{equation*}
  z^n=[r(\cos{\theta}+i\sin{\theta})]^n = r^n(\cos{n\theta}+i\sin{n\theta})
\end{equation*}
在上式中将$\theta$换成$-\theta$得
\begin{equation*}
  [r(\cos{\theta}-i\sin{\theta})]^n = r^n(\cos{n\theta}-i\sin{n\theta})
\end{equation*}
两式相加,并按二项定理将左边展开,所有求和指标为奇数的项都被消去,得
\begin{eqnarray*}
  \cos{n\theta} & = & \sum_{0 \leqslant i \leqslant n/2}C_n^{2i}(-1)^i\cos^{n-2i}{\theta}\sin^{2i}{\theta} \\
  & = & \sum_{0 \leqslant i \leqslant n/2}C_n^{2i}(-1)^i\cos^{n-2i}{\theta}(1-\cos^2{\theta})^i \\
\end{eqnarray*}
上式中右边每一项都是$\cos{\theta}$的$n$次多项式,所有最高次项式的和是$C_n^0+C_n^2+C_n^4+\cdots$,这是对上标为偶数的二次项系数求和,由组合数性质,这个和是$2^{n-1}$,于是$\cos{n\theta}$就是$\cos{\theta}$的$n$次多项式,最高次项系数是$2^{n-1}$,这样也引出了第一类的切比雪夫多项式,还得出了它的具体表达式:
\begin{equation*}
  T_n(x) = \sum_{0 \leqslant i \leqslant n/2}C_n^{2i}(-1)^ix^{n-2i}(1-x^2)^{2i}
\end{equation*}

以下我们来推证这些多项式的一些性质,我们先由公式$\cos{n\theta}=T_n(\cos{\theta})$得出结论,再尝试利用递推关系来加以证明。

\begin{property}
  对于任意满足$|x|\leqslant 1$的实数$x$,都有$|T_n(x)| \leqslant 1$.
\end{property}



\subsection{第二类切比雪夫多项式}
\label{sec:second-chebyshev-polynome}

第一类切比雪夫多项式刻画的是$\cos{n\theta}$展开为$\cos{\theta}$的多项式,那么正弦的情况又如何呢,还是先观察一下二倍角公式和三倍角公式
\begin{eqnarray*}
  \sin{2\theta} & = & 2\sin{\theta}\cos{\theta} \\
  \sin{3\theta} & = & 3\sin{\theta}-4\sin^3{\theta}
\end{eqnarray*}
不难发现,$\sin{n\theta}$无法表示为$\sin{\theta}$的多项式,但是$\sin{n\theta}$似乎可以表达为$\sin{\theta}$与一个关于$\cos{\theta}$的多项式的乘积,也就是说,$\sin{n\theta}/\sin{\theta}$可表为$\cos{\theta}$的多项式。那么对于一般情况,仍然由和差化积
\begin{equation*}
  \sin{(n+1)\theta}=2\sin{n\theta}\cos{\theta}-\sin{(n-1)\theta}
\end{equation*}
可见,$\sin{(n+1)\theta}$也能表为$\sin{\theta}$与一个关于$\cos{\theta}$的$n$次多项式之积,于是得到
\begin{theorem}
  对于任意正整数$n$,$\sin{(n+1)\theta}/\sin{\theta}$可表为一个关于$\cos{\theta}$的$n$次多项式,其最高次项系数是$2^n$,这多项式称为\emph{第二类切比雪夫多项式}。
\end{theorem}

仍然有递推公式$U_0(x)=1, U_1(x)=2x$和
\begin{equation*}
  U_{n+1}(x)=2xU_n(x)-U_{n-1}(x)
\end{equation*}
可以发现,两类切比雪夫多项的递推公式是相同的,不同的是两个初始值。

第二类切比雪夫多项式以$U_n(x)$标记,有
\begin{equation*}
  \frac{\sin{(n+1)\theta}}{\sin{\theta}} = U_n(\cos{\theta})
\end{equation*}

同样利用复数的乘方公式,可以得到第二类切比雪夫多项式的具体表达:
\begin{equation*}
  U_n(x) = \sum_{0 \leqslant 2i+1 \leqslant n+1}(-1)^iC_n^{2i+1}\cos^{n-2k}{\theta}(1-\cos^2{\theta})^{k}
\end{equation*}
其最高次项系数是$C_{n+1}^1+C_{n+1}^3+C_{n+1}^5+\cdots=2^n$。

前几个多项式是
\begin{eqnarray*}
  U_0(x) & = & 1 \\
  U_1(x) & = & 2x \\
  U_2(x) & = & 4x^2-1 \\
  U_3(x) & = & 8x^3-4x \\
  U_4(x) & = & 16x^4-12x^2+1 \\
  U_5(x) & = & 32x^5-32x^3+6x \\
  U_6(x) & = & 64x^6-80x^4+24x^2-1
\end{eqnarray*}

%%% Local Variables:
%%% mode: latex
%%% TeX-master: "../../book"
%%% End:
